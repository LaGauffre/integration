\documentclass[11pt, addpoints, answers]{exam}

\usepackage[utf8]{inputenc}
\usepackage[T1]{fontenc}
\usepackage[margin  = 1in]{geometry}
\usepackage{amsmath, amscd, amssymb, amsthm, verbatim}
\usepackage{mathabx}
\usepackage{setspace}
\usepackage{float}
\usepackage{color}
\usepackage{graphicx}
\usepackage[colorlinks=true]{hyperref}
\usepackage{tikz}
\usetikzlibrary{trees}

\shadedsolutions
\definecolor{SolutionColor}{RGB}{214,240,234}

\newcommand{\bbC}{{\mathbb C}}
\newcommand{\R}{\mathbb{R}}            % real numbers
\newcommand{\bbR}{{\mathbb R}}
\newcommand{\Z}{\mathbb{Z}}            % integers
\newcommand{\No}{\mathbb{N}}            % integers
\newcommand{\bbZ}{{\mathbb Z}}
\newcommand{\bx}{\mathbf x}            % boldface x
\newcommand{\by}{\mathbf y}            % boldface y
\newcommand{\bz}{\mathbf z}            % boldface z
\newcommand{\bn}{\mathbf n}            % boldface n
\newcommand{\br}{\mathbf r}            % boldface r
\newcommand{\bc}{\mathbf c}            % boldface c
\newcommand{\be}{\mathbf e}            % boldface e
\newcommand{\bE}{\mathbb E}            % blackboard E
\newcommand{\bP}{\mathbb P}            % blackboard P

\newcommand{\ve}{\varepsilon}          % varepsilon
\newcommand{\avg}[1]{\left< #1 \right>} % for average
%\renewcommand{\vec}[1]{\mathbf{#1}} % bold vectors
\newcommand{\grad}{\nabla }
\newcommand{\lb}{\langle }
\newcommand{\rb}{\rangle }

\def\Bin{\operatorname{Bin}}
\def\Var{\operatorname{Var}}
\def\Geom{\operatorname{Geom}}
\def\Pois{\operatorname{Pois}}
\def\Exp{\operatorname{Exp}}
\newcommand{\Ber}{\operatorname{Ber}}
\def\Unif{\operatorname{Unif}}
\def\No{\operatorname{N}}
\newcommand{\E}{\mathbb E}            % blackboard E
\def\th{\theta }            % theta shortcut
\def\V{\operatorname{Var}}
\def\Var{\operatorname{Var}}
\def\Cov{\operatorname{Cov}}
\def\Corr{\operatorname{Corr}}
\newcommand{\epsi}{\varepsilon}            % epsilon shortcut

\providecommand{\norm}[1]{\left\lVert#1\right\rVert} %norm
\providecommand{\abs}[1]{\left \lvert#1\right \rvert} %absolute value

\DeclareMathOperator{\lcm}{lcm}
\newcommand{\ds}{\displaystyle}	% displaystyle shortcut

\def\semester{2019-2020 }
\def\course{Théorie de la mesure et intégration}
\def\title{\MakeUppercase{Examen deuxième session}}
\def\name{Pierre-O. Goffard}
%\def\name{Professor Wildman}

\setlength\parindent{0pt}

\cellwidth{.35in} %sets the minimum width of the blank cells to length
\gradetablestretch{2.5}

%\bracketedpoints
%\pointsinmargin
%\pointsinrightmargin

\begin{document}


\runningheader{\course  \vspace*{.25in}}{}{\title \vspace*{.25in}}
%\runningheadrule
\runningfooter{}{Page \thepage\ of \numpages}{}

% \firstpageheader{Name:\enspace\hbox to 2.5in{\hrulefill}\\  \vspace*{2em} Section: (circle one) TR: 3-3:50 \textbar\, TR: 5-5:50 \textbar\,  TR: 6-6:50(Xu) \textbar\,  TR: 6-6:50 }{}{Perm \#: \enspace\hbox to 1.5in{\hrulefill}\\ \vspace*{2em} Score:\enspace\hbox to .6in{\hrulefill} $/$\numpoints}
\extraheadheight{.25in}

\hrulefill

\vspace*{1em}

% Heading
{\center \textsc{\Large\title}\\
	\vspace*{1em}
	\course -- \semester\\
	Pierre-O Goffard\\
}
\vspace*{1em}

\hrulefill

\vspace*{2em}

\noindent {\bf\em Instructions:} On éteint et on range son téléphone.
\begin{itemize}
	\item La calculatrice et les appareils éléctroniques ne sont pas autorisés.
	\item Vous devez justifier vos réponses de manière claire et concise.
	\item Vous devez écrire de la manière la plus lisible possible. Souligner ou encadrer votre réponse finale.
\end{itemize}

\begin{center}
	\gradetable[h]
\end{center}

\smallskip

\begin{questions}
\question 
\begin{parts}
\part[1] Soient $A_1,\ldots, A_n\subset\Omega$, montrer que 
$$
\bigcup_{n\geq1}A_n^c  =\left(\bigcap_{n\geq1}A_n\right)^c 
$$ 
\begin{solution}
On procède par double inclusion, 
\begin{itemize}
\item Supposons que $x\in \bigcup_{n\geq1}A_n^c $, alors $\exists n\geq1$ tel que $x\in A_n^c$ puis $x\notin A_n$ et donc $x\notin \bigcap_{n\geq1}A_n$, puis finalement $x\in\left(\bigcap_{n\geq1}A_n\right)^c$ 
\item Supposons que $x\in\left(\bigcap_{n\geq1}A_n\right)^c $ alors $x\notin \bigcap_{n\geq1}A_n$ alors $\exists n\geq1$ tel que $x\notin A_n$, donc $x\in A_n^c$ et finalement $x\in\bigcup_{n\geq1}A_n^c$.
\end{itemize}
\end{solution}
\part[1] Soit $\mu$ une mesure sur l'espace mesurable $(\R,\mathcal{B}_\R)$ de densité 
$$
	f_{\mu}(x) = xe^{-x^2/2}\mathbb{I}_{x>0}
$$ 
par rapport à la mesure de Lebesgue. Calculer $\mu([0,1])$.
\begin{solution}
$1-e^{-1/2}$
\end{solution}
\part[1] Montrer que 
$$
	0\leq \int_{0}^{2}\frac{e^{-x^2/2}}{\sqrt{2\pi}}\text{d}{x}\leq \frac{2}{\sqrt{2\pi}} 
$$
\begin{solution}
On note que $e^{-x^2/2}<1$ pour $x\in[0,2]$
\end{solution}
\end{parts}
\question 
	\begin{parts}
	\part Soit $(\Omega,\mathcal{A}$ une espace mesurable et $x\Omega$. Montrer que l'application 
	$$
	\delta_x:\mathcal{A}\mapsto [0,+\infty),
	$$ 
	tel que 
	$$
		\delta_x(B)=
		\begin{cases} 
			1,&\text{ si }x\in B,\\
			0,&\text{ sinon.}
		\end{cases} 
	$$
	avec $B\in\mathcal{A}$, est une mesure.
	\begin{solution}
		\begin{itemize}
			\item[(i)] $x\notin \emptyset$ donc $\delta_x(\emptyset) = 0$
			\item[(ii)] Soient $(A_n)_{n\in\No^\ast}\in\mathcal{A}$ une suite d'évènements disjoints. Si $x\in\bigcup_{n\in\No^\ast} A_n$ alors $\exists n\in\No^\ast$ tel que $x\in A_n$ et $x\notin A_m$ pour $m\neq n$ (les $A_n$ sont disjoints). On en déduit que 
			$$
				\delta_x\left(\bigcup_{n\in\No^\ast} A_n\right) =1 = \sum_{n\in\No^\ast}\delta_x(A_n).
			$$
			De la même façon si $x\notin\bigcup_{n\in\No^\ast} A_n$ alors
			$$
				\delta_x\left(\bigcup_{n\in\No^\ast} A_n\right) =0 = \sum_{n\in\No^\ast}\delta_x(A_n).
			$$
		\end{itemize}
	\end{solution}
	\part[1] Soit $\mu$ une mesure sur l'espace mesurable $(\R,\mathcal{B}_\R)$ de densité 
	$$
		f_{\mu}(x) = xe^{-x}\mathbb{I}_{x>0}
	$$ 
	par rapport à la mesure de Lebesgue. Calculer $\mu([0,1])$.
	\begin{solution}
		$1-2e^{-1}$
	\end{solution}
	\part Montrer que 
	$$
		0\leq \int_{\pi/4}^{\pi/2}\sin(\ln(1+u))\text{d}u\leq \frac{\sqrt{2}}{2}. 
	$$
	\begin{solution}
		Comme $0\leq \ln(1+u)\leq u$ pour $u\geq0$ et que $u\mapsto \sin(u)$ est croissante pour $u\in(0,\pi/2)$ alors 
		$$
			\int_{\pi/4}^{\pi/2}\sin(\ln(1+u))\text{d}u\leq  \int_{\pi/4}^{\pi/2}\sin(u)\text{d}u = \frac{\sqrt 2}{2}
		$$
	\end{solution}
\end{parts}
\question
	\begin{parts}
		\part[1] Montrer que l'application  $x\mapsto \frac{1-e^{-x^2y}}{x^2}$ est intégrable sur $[0,+\infty)$
		\begin{solution}
		L'application $f:x\mapsto \frac{1-e^{-x^2y}}{x^2}$ est continue sur $(0,+\infty)$. 
		\begin{itemize}
			\item Pour $x\rightarrow 0$, on a $f(x)\rightarrow y$. $f$ est prolongeable par continuité et donc intégrable en $0$
			\item Pour $x\rightarrow +\infty$, $f(x)\sim 1/x^2$ donc intégrable. 
		\end{itemize}
		\end{solution}
		\part[2] On définit $F(y) =\int_0^{+\infty} f(x,y)\text{d}\lambda(x)$. Montrer que F est dérivable et calculer sa dérivée.
		\begin{solution}
		Pour tout $y>0$, l'application est $f:(x,y)\mapsto \frac{1-e^{-x^2y}}{x^2}$  dérivable avec 
		$$
			\frac{\partial f}{\partial y}(x,y) = e^{-x^2y}
		$$
		et 
		$$
			\left|\frac{\partial f}{\partial y}(x,y)\right|<e^{-x^2y_0}	
		$$
		où $0<y_0<y$. Comme $x\mapsto e^{-x^2y_0}$ est intégrable alors en appliquant le théorème de dérivation sous l'intégrale, on obtient
		$$
			F'(y) = \frac{\sqrt{\pi}}{2\sqrt{y}}.
		$$
		\end{solution}
	\end{parts}

	\question  
	\begin{parts}
		\part[1] Calculer la limite
		$$
			\underset{n\rightarrow+\infty}{\lim}\int_{0}^1\frac{1}{\sqrt{x}}\sin\left(\frac{1}{nx}\right) \text{d}x.
		$$
    
		\begin{solution} 
     		Soit la suite de fonction définie par 
     		$$
     			f_n(x)=\frac{1}{\sqrt{x}}\sin\left(\frac{1}{nx}\right),\text{ }n\geq0.
     		$$
     		Il s'agit d'une fonction intégrable sur $[0,1]$ qui vérifie $|f_n(x)|<x^{-1/2}$. Comme $x\mapsto x^{-1/2}$ est une fonction intégrable sur $[0,1]$ et que $\underset{n\rightarrow\infty}{\lim}f_n(x) = 0$ alors il vient par copnvergence dominée
     		$$
     			\underset{n\rightarrow+\infty}{\lim}\int_{0}^1\frac{1}{\sqrt{x}}\sin\left(\frac{1}{nx}\right) \text{d}x = 0.
     		$$
		\end{solution}
	\part[2] Soit $(\Omega, \mathcal{A}, \mu)$ un espace mesuré et $f:(\Omega,\mathcal{A})\mapsto (\R,\mathcal{B}_{\R})$. Montrer que l'application définie par 
	$$
		f_{a}(\omega) = 
		\begin{cases}
			-a & \text{ si }f(\omega)< -a\\
			f(\omega)&\text{ si }f(\omega)\in[-a,a]\\
			a&\text{ si }f(\omega)>a
		\end{cases} 
	$$
	est mesurable. 
		\begin{solution}
			$f$ est mesurable car elle peut s'écrire comme combinaison linéaire de fonctions mesurables.  		
		\end{solution}
		
	\end{parts}
	\question
	\begin{parts}
		\part[2] Soit $(\Omega, \mathcal{A}, \mu)$ un espace mesuré et $f:\Omega\mapsto \R_+$ une application mesurable positive. Montrer que l'application définie par 
		$$
			\lambda(A) = \int_{A}f\text{d}\mu,\text{ }A\in\mathcal{A}
		$$
		est une mesure sur $(\Omega,\mathcal{A})$.
		\begin{solution}
		\begin{itemize}
			\item[(i)] $\lambda(\emptyset) = \int_{\emptyset}f\text{d}\mu = 0$ car $\mu(\emptyset)=0$
			\item[(ii)] Soit $(A_n)_{n\in\No^{\ast}}$ une suites d'évènements disjoints. On a d'une part 
			$$
				\lambda\left(\bigcup_{n\in\No^{\ast}} A_n\right) = \int_{\bigcup_{n\No^{\ast}} A_n} f\text{d}\mu = \int_{\Omega} \sum_{n\in\No^{\ast}}\mathbb{I}_{A_n}f\text{d}\mu
			$$
			La suite $f_n = \sum_{k=1}^n\mathbb{I}_{A_k}f$ est une suite croissante de fonctions positives donc par croissance monotone
			$$
				\int_{\Omega} \sum_{n\in\No^{\ast}}\mathbb{I}_{A_n}f\text{d}\mu= \int_{\Omega} \lim \sum_{k=1}^n\mathbb{I}_{A_k}f\text{d}\mu=\lim\int_{\Omega}  \sum_{k=1}^n\mathbb{I}_{A_k}f\text{d}\mu=\lim \sum_{k=1}^n\int_{\Omega} \mathbb{I}_{A_k}f\text{d}\mu =\sum_{n\in\No^{\ast}}\lambda(A_n)
			$$
		\end{itemize}
		\end{solution}
		\part[1] Calculer la limite 
		$$
			\underset{n\rightarrow +\infty}{\lim}\int_{(-\infty,+\infty)}e^{1+\cos^{2n}(x)}e^{-|x|}\text{d}\lambda(x). 
		$$
		\begin{solution}
			La suite de fonction $f_n(x) = e^{1+\cos^{2n}(x)}e^{-|x|},\,n\geq0 $ vérifie $|f_n(x)|<e^{1-|x|}$, où $x\mapsto e^{2-|x|}$ est intégrable. Comme $f_n(x)\rightarrow e^{1-|x|}$ lorsque $n\rightarrow+\infty$ alors par convergence dominée
			$$
				\underset{n\rightarrow +\infty}{\lim}\int_{(-\infty,+\infty)}e^{1+\cos^{2n}(x)}e^{-|x|}\text{d}\lambda(x)= 2e^1
			$$
		\end{solution}
	\end{parts}
	\question
	\begin{parts}
		\part Soit $(\Omega, \mathcal{A}, \mu)$ un espace mesuré et $(g_n)_{n\geq1}$ une suite de fonctions mesurables à valeur dans $\mathbb{R}_+$. Montrer que 
		$$
			\int_{\Omega}\sum_{n=1}^{+\infty}g_n\text{d}\mu = \sum_{n=1}^{+\infty}\int_{\Omega}g_n\text{d}\mu 
		$$
		\begin{solution}
			Voir les notes de cours.
		\end{solution}
		\part Monter que 
		$$
			\int_{0}^{+\infty}\frac{x^{s-1}}{e^{x}-1}\text{d}x = \Gamma(s)\zeta(s), 
		$$
		où $\Gamma(s) = \int_{0}^{+\infty}e^{-x}x^{s-1}\text{d}x$ et $\zeta(s) = \sum_{n=1}^{+\infty}n^{-s}$. 
		\begin{solution}
		On a 
		\begin{eqnarray*}
			\Gamma(s)\zeta(s) &=& \int_{0}^{+\infty}\sum_{n=1}^{\infty}e^{-x}\left(\frac xn\right)^{s-1}\frac 1n \text{d}x \\
			&=&\int_{0}^{+\infty}\sum_{n=1}^{\infty}e^{-ny} y^{s-1}\text{d}y\\
			&=&\int_{0}^{+\infty}\frac{y^{s-1}}{e^x-1}\text{d}y
		\end{eqnarray*}		
		\end{solution}
	\end{parts}
	\question
	\begin{parts}
		\part Calculer l'intégrale
		$$
			\int_{(0,+\infty)^2}\frac{1}{(1+x^2y)(1+y)}\text{d}\lambda_{2}(x,y)
		$$  
		en intégrant d'abord par rapport à $x$.
		\begin{solution}
			$$
				\int_{0}^{+\infty}\int_{0}^{+\infty}\frac{1}{(1+x^2y)(1+y)}\text{d}x\text{d}y = \frac{\pi^{2}}{4}
			$$
		\end{solution}
		\part En déduire la valeur de l'intégrale 
		$$
			\int_{0}^{+\infty}\frac{\ln(x)}{x^2-1}\text{d}x 
		$$
		en intégrant par rapport $y$.
		\begin{solution}
		\begin{eqnarray*}
				\int_{0}^{+\infty}\int_{0}^{+\infty}\frac{1}{(1+x^2y)(1+y)}\text{d}y\text{d}x &=&\int_{0}^{+\infty}\int_{0}^{+\infty}\frac{x^2}{(x^2-1)(1+x^2y)}+\frac{1}{(1-x^2)(1+y)}\text{d}y\text{d}x \\
				&=& \int_{0}^{+\infty}\frac{1}{x^2-1}\left[\ln\left(\frac{1}{1+y}+x^2\frac{y}{1+y}\right)\right]_{0}^{\infty}\text{d}x = 2\int_{0}^{+\infty}\frac{\ln(x)}{x^2-1}\text{d}x
		\end{eqnarray*}
		puis $\int_{0}^{+\infty}\frac{\ln(x)}{x^2-1}\text{d}x = \frac{\pi^2}{8}$
		\end{solution}
	\end{parts}
\end{questions}

\newpage
%-------------------------------TABLE-------------------------------
\hrule
\vspace*{.15in}
\begin{center}
	\large\MakeUppercase{Fonctions Trigonométriques}
\end{center}
\vspace*{.15in}
\hrule
\vspace*{.25in}


\renewcommand\arraystretch{3.5}
\begin{table}[H]
\begin{center}
\begin{tabular}{|c|c|c|}
\hline
Fonction & Ensemble de définition& Dérivée \\
\hline\hline
$\sin x$ &$\mathbb{R}$& $\cos x$ \\
\hline
$\cos x$ &$\mathbb{R}$& $-\sin x$ \\
\hline
$\tan x$ &$\bigcup_{n\in\mathbb{Z}}\left]n\pi-\pi/2, n\pi+\pi/2\right[$ & $1+\tan^2 x$ \\
\hline
$\arccos x$ &$[-1,1]$& $-\frac{1}{\sqrt{1-x^2}}$ \\
\hline
$\arcsin x$ &$[-1,1]$& $\frac{1}{\sqrt{1-x^2}}$\\
\hline
$\arctan x$ &$\mathbb{R}$& $\frac{1}{1+x^2}$ \\
\hline
\end{tabular}
\end{center}
\end{table}%



\end{document}

%%%% Extra problems %%%%--------------------------------------

%\question A coin having probability $0.8$ of landing on heads is flipped.  $A$ observes the result -- either heads or tails -- and rushes off to tell $B$. However, with probability $0.4$, $A$ will have forgotten the result by the time he reaches $B$.  If $A$ has forgotten, then, rather than admitting this to $B$, he is equally likely to tell $B$ that the coin landed on heads or that it landed tails. (If he does remember, then he tells $B$ the correct result).
%\begin{parts}
%	\part What is the probability that $B$ is told that the coin landed on heads?
%	\part What is the probability that $B$ is told the correct result?
%	\part Given that $B$ is told that the coin landed on heads, what is the probability that it did in fact land on heads?
%\end{parts}

%\question % Variance, expected value
%Suppose that $\bP(X=0)=1-\bP(X=1)$.  If $\E[X]=3\Var(X)$, find $\bP(X=0)$.