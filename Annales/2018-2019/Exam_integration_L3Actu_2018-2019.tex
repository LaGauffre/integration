\documentclass[11pt, addpoints, answers]{exam}

\usepackage[utf8]{inputenc}
\usepackage[T1]{fontenc}
\usepackage[margin  = 1in]{geometry}
\usepackage{amsmath, amscd, amssymb, amsthm, verbatim}
\usepackage{mathabx}
\usepackage{setspace}
\usepackage{float}
\usepackage{color}
\usepackage{graphicx}
\usepackage[colorlinks=true]{hyperref}
\usepackage{tikz}
\usetikzlibrary{trees}

\shadedsolutions
\definecolor{SolutionColor}{RGB}{214,240,234}

\newcommand{\bbC}{{\mathbb C}}
\newcommand{\R}{\mathbb{R}}            % real numbers
\newcommand{\bbR}{{\mathbb R}}
\newcommand{\Z}{\mathbb{Z}}            % integers
\newcommand{\bbZ}{{\mathbb Z}}
\newcommand{\bx}{\mathbf x}            % boldface x
\newcommand{\by}{\mathbf y}            % boldface y
\newcommand{\bz}{\mathbf z}            % boldface z
\newcommand{\bn}{\mathbf n}            % boldface n
\newcommand{\br}{\mathbf r}            % boldface r
\newcommand{\bc}{\mathbf c}            % boldface c
\newcommand{\be}{\mathbf e}            % boldface e
\newcommand{\bE}{\mathbb E}            % blackboard E
\newcommand{\bP}{\mathbb P}            % blackboard P

\newcommand{\ve}{\varepsilon}          % varepsilon
\newcommand{\avg}[1]{\left< #1 \right>} % for average
%\renewcommand{\vec}[1]{\mathbf{#1}} % bold vectors
\newcommand{\grad}{\nabla }
\newcommand{\lb}{\langle }
\newcommand{\rb}{\rangle }

\def\Bin{\operatorname{Bin}}
\def\Var{\operatorname{Var}}
\def\Geom{\operatorname{Geom}}
\def\Pois{\operatorname{Pois}}
\def\Exp{\operatorname{Exp}}
\newcommand{\Ber}{\operatorname{Ber}}
\def\Unif{\operatorname{Unif}}
\def\No{\operatorname{N}}
\newcommand{\E}{\mathbb E}            % blackboard E
\def\th{\theta }            % theta shortcut
\def\V{\operatorname{Var}}
\def\Var{\operatorname{Var}}
\def\Cov{\operatorname{Cov}}
\def\Corr{\operatorname{Corr}}
\newcommand{\epsi}{\varepsilon}            % epsilon shortcut

\providecommand{\norm}[1]{\left\lVert#1\right\rVert} %norm
\providecommand{\abs}[1]{\left \lvert#1\right \rvert} %absolute value

\DeclareMathOperator{\lcm}{lcm}
\newcommand{\ds}{\displaystyle}	% displaystyle shortcut

\def\semester{2018-2019 }
\def\course{Théorie de la mesure }
\def\title{\MakeUppercase{Examen Final}}
\def\name{Pierre-O Goffard}
%\def\name{Professor Wildman}

\setlength\parindent{0pt}

\cellwidth{.35in} %sets the minimum width of the blank cells to length
\gradetablestretch{2.5}

%\bracketedpoints
%\pointsinmargin
%\pointsinrightmargin

\begin{document}


\runningheader{\course  \vspace*{.25in}}{}{\title \vspace*{.25in}}
%\runningheadrule
\runningfooter{}{Page \thepage\ of \numpages}{}

% \firstpageheader{Name:\enspace\hbox to 2.5in{\hrulefill}\\  \vspace*{2em} Section: (circle one) TR: 3-3:50 \textbar\, TR: 5-5:50 \textbar\,  TR: 6-6:50(Xu) \textbar\,  TR: 6-6:50 }{}{Perm \#: \enspace\hbox to 1.5in{\hrulefill}\\ \vspace*{2em} Score:\enspace\hbox to .6in{\hrulefill} $/$\numpoints}
\extraheadheight{.25in}

\hrulefill

\vspace*{1em}

% Heading
{\center \textsc{\Large\title}\\
	\vspace*{1em}
	\course -- \semester\\
	Pierre-O Goffard\\
}
\vspace*{1em}

\hrulefill

\vspace*{2em}

\noindent {\bf\em Instructions:} On éteint et on range son téléphone.
\begin{itemize}
	\item La calculatrice et les appareils éléctroniques ne sont pas autorisés.
	\item Vous devez justifier vos réponses de manière claire et concise.
	\item Vous devez écrire de la manière la plus lisible possible. Souligner ou encadrer votre réponse finale.

\end{itemize}

\begin{center}
	\gradetable[h]
\end{center}

\smallskip

\begin{questions}
\question Soit $(\Omega, \mathcal{A},\mu)$ un espace mesuré. Soient $A,B\in\mathcal{A}$.
	\begin{parts}
		\part[1]  Montrer que 
		$$
		\mu(A\cup B)=\mu(A)+\mu(B)-\mu(A\cap B)
		$$
		\begin{solution}
			$$
			\mu(A\cup B)=\mu[(A/A\cap B)\cup B]=\mu(A/A\cap B)+\mu(B)=\mu(A)-\mu(A\cap B)+\mu(B).
			$$
		\end{solution}
		\part[1] Montrer que l'application définie par 
		$$
		\mu_A(B)=\frac{\mu(A\cap B)}{\mu(A)}
		$$
		est une mesure de probabilité.
		\begin{solution}
		Nous avons 
			\begin{itemize}
				\item $\mu_A(\emptyset)=\frac{\mu(A\cap \emptyset)}{\mu(A)}=\frac{\mu(\emptyset)}{\mu(A)}=0$
				\item Soit $(A_n)_{n\in\mathbb{N}}$ une suite de parties disjointes de $\mathcal{A}$,
				$$
				\mu_A\left(\bigcup_{n\in\mathbb{N}}A_n\right)=\frac{\mu\left(A\cap \bigcup_{n\in\mathbb{N}}A_n\right)}{\mu(A)}=\frac{\mu\left(\bigcup_{n\in\mathbb{N}}A\cap A_n\right)}{\mu(A)}=\frac{\sum_{n\in\mathbb{N}}\mu\left(A\cap A_n\right)}{\mu(A)}=\sum_{n\in\mathbb{N}}\mu_A(A_n).
				$$   
			\end{itemize}
			$mu_A$ est une mesure. De plus, comme 
			$$
			\mu_A(\Omega)=\frac{\mu(A\cap \Omega)}{\mu(A)}=\frac{\mu(A)}{\mu(A)}=1.
			$$
			alors $mu_A$ est une mesure de probabilité.
		\end{solution}
	\end{parts}
\question Soit $\mathcal{T}=\{A\in\mathcal{P}(\R)\text{ ; }A=-A\}$, où $-A = \{-x\text{ ; }x\in A\}$.
\begin{parts}
\part[1] Montrer que $\mathcal{T}$ est une tribu sur $\R$
\begin{solution}
\begin{itemize}
	\item Pour $x\in \R$, on a $-x\in\R$ donc $\R\in \mathcal{T}$
	\item Pour $x\in A^c$, on a $-x\notin A$ donc $-x\in A^c$. On en déduit que $A^c\in \mathcal{T}$
	\item Soit $(A_n)_{n\in\mathbb{N}}\in \mathcal{T}$, si $x\in \bigcup_{n\in\mathbb{N}}A_n$ alors il existe $n\in \mathbb{N}$ tel que $x\in A_n$, donc $-x\in A_n$ puis $-x\in \bigcup_{n\in\mathbb{N}}A_n$. On en déduit que $\bigcup_{n\in\mathbb{N}}A_n\in\mathcal{T}$.    
\end{itemize}
$\mathcal{T}$ est bien une tribu.
\end{solution}
\part[1] Montrer que $\mathcal{T}'=\{A\in\mathcal{B}(\R)\text{ ; }A=-A\}$ est une tribu.
\begin{solution}
$\mathcal{T}'=\mathcal{T}\cap\mathcal{B}(\R)$ est une tribu en tant qu'intersection de deux tribus.
\end{solution} 
\end{parts}
\question La fonction gamma est définie par 
$$
\Gamma(x)=\int_0^{\infty}t^{x-1}e^{-t}\text{d}t,\text{ }x>0.
$$
et la fonction beta est définie par 
$$
B(x,y)=\int_0^{1}t^{x-1}(1-t)^{y-1}\text{d}t,\text{ }x,y>0.
$$ 
\begin{parts}
\part[1] Montrer que 
$$
\Gamma(x+1)=x\Gamma(x)
$$
\begin{solution}
Intégration par partie.
\end{solution}
\part[2] En déduire que 
$$
\Gamma(n)=(n-1)!
$$
et 
$$
\Gamma(n+1/2)=\Gamma(1/2)\frac{(2n)!}{2^{2n}n!}
$$
\begin{solution}
On a 
$$
\Gamma(n)=(n-1)\Gamma(n-1)=\ldots = n!\Gamma(0)=n!
$$
et 
\begin{eqnarray*}
\Gamma(n+1/2)&=&(n-1+1/2)\Gamma(n-1+1/2)=\ldots =\Gamma(1/2)\frac{(2n-1)(2n-3)\ldots 1}{2^n}\\
&=&\Gamma(1/2)\frac{2n!}{2^n (2n)(2n-2)\ldots 2}=\Gamma(1/2)\frac{2n!}{2^{2n} n!}
\end{eqnarray*}
\end{solution}
\part[2] A l'aide du changement de variable 
$
\begin{cases}
t = u+v\\
s= u/(u+v)
\end{cases},
$ 
établir que 
$$
\int_{\R_+\times \R_+}f(u+v)u^{x-1}v^{y-1}\text{d}u\text{d}v=B(x,y)\int_0^{+\infty}t^{x+y-1}f(t)\text{d}t,
$$
où $f:\R_+\mapsto\R_+$ est une fonction borelienne. 
\begin{solution}
L'application associée au changement de variable est donnée par 
$$
\phi:(s,t)\mapsto(st,t(1-s)),
$$ 
de Jacobien
$$
\text{Det}J_\phi = 
\left|\begin{array}{cc}
s&t\\
1-s&-t
\end{array}\right|=-t.
$$
On pose $g:(u,v)\mapsto f(u+v)u^{x-1}v^{y-1}$. Par application de la formule de changement de variable, il vient 
\begin{eqnarray*}
\int_{\R_+\times \R_+}g(u,v)\text{d}u\text{d}v&=&\int_{\phi(\R_+\times \R_+)}g(\phi(s,t))|\text{Det}J_\phi|\text{d}s\text{d}t\\
&=&\int_{\R_+\times [0,1]}f(t)s^{x-1}(1-s)^{y-1}t^{x+y-1}\text{d}s\text{d}t.
\end{eqnarray*}
\end{solution}
\part[1] En déduire la formule suivante 
$$
B(x,y)=\frac{\Gamma(x)\Gamma(y)}{\Gamma(x+y)}.
$$
\begin{solution}
On pose $f(t):=e^{-t}$.
\end{solution}
\part[2] Calculer $\Gamma(1/2)$.
\begin{solution}
On fixe $x=y=1/2$ et on calcule 
$$
B(1/2,1/2)=\int_0^{1}\frac{1}{\sqrt{t(1-t)}}\text{d}t=\ldots = \pi,
$$
en réarrangeant sous la racine carré et en appliquant le changement de variable $u = 2t-1$.
\end{solution}
\end{parts}
\question On souhaite calculer l'intégrale de Dirichlet,
$$
\int_0^{+\infty}\frac{\sin t}{t}\text{d}t,
$$
à l'aide d'une intégrale à paramètre.
\begin{parts}
\part[2] Montrer que $\int_0^{+\infty}\frac{\sin t}{t}\text{d}t$ est convergente.
\begin{solution}
La fonction $t\mapsto \frac{\sin t}{t}$ est continue sur $(0,\infty)$ et donc localement intégrable. La fonction se prolonge par continuité en $0$ avec $\underset{t\rightarrow 0}{\lim} \frac{\sin t}{t} = 1$. Au voisinage de $\infty$, une intégration par partie donne  
$$
\int_{\pi/2}^{+\infty}\frac{\sin t}{t}\text{d}t=-\frac{\cos(M)}{M}+\int_{\pi/2}^{+\infty}\frac{\cos t}{t^{2}}\text{d}t.
$$ 
Comme $\frac{\cos t}{t^{2}}\underset{t \rightarrow +\infty}{\sim} \frac{1}{t^{2}}$ alors l'intégrale converge. 
\end{solution}
\part[2] Montrer que $t\mapsto \frac{\sin t}{t}$ n'est pas intrégrable par rapport à la mesure de Lebesgue sur $\mathbb{R}_+$.\\
\underline{Indication: }\\
Définir la suite de fonctions 
$$
f_n(t)=\frac{\sin t}{t}\mathbb{I}_{[n\pi,(n+1)\pi]}(t),\text{ }n\geq0,
$$
trouver un équivalent pour la suite 
$$
u_n = \int_{\mathbb{R}_+}|f_n(t)|\text{d}\lambda(t),\text{ pour }n\rightarrow +\infty.
$$
\begin{solution}
On note que $\int_0^{n\pi}\frac{|\sin(t)|}{t}\text{d}t=\sum_{k=0}^{n-1}u_n$. On remarque ensuite que  
\begin{eqnarray*}
u_n &=&\int_{n\pi}^{(n+1)\pi}\frac{|\sin(t)|}{t}\text{d}t\\
&=&\int_{0}^{\pi}\frac{|\sin(u+n\pi)|}{u+n\pi}\text{d}u\\
&=&\int_{0}^{\pi}\frac{|\sin(u)|}{u+n\pi}\text{d}u\\
&=&\frac{1}{n\pi}\int_{0}^{\pi}|\sin(u)|\frac{n\pi}{u+n\pi}\text{d}u.
\end{eqnarray*}
Comme $|\sin(u)|\frac{n\pi}{u+n\pi}\leq 1$ alors par convergence dominée on a 
$$
\int_{0}^{\pi}|\sin(u)|\frac{n\pi}{u+n\pi}\text{d}u\rightarrow \int_{0}^{\pi}|\sin(u)| = 2.
$$
On en déduit que $u_n\underset{n\rightarrow\infty}{\sim}\frac{2}{n\pi}$ puis que l'intégrale 
$\int_{\mathbb{R}_+}\frac{\sin(t)}{t}\text{d}\lambda(t)=\sum_{n=0}^{+\infty}\int f_n(t)\text{d}\lambda(t)$ diverge.
\end{solution}
\part[2] On pose 
$$
F(x)=\int_0^{+\infty}\frac{\sin t}{t}e^{-xt}\text{d}t,\text{ }x\geq0.
$$
Montrer que $F$ est dérivable sur $\left]0,+\infty\right[$ et que 
$$
F'(x)=-\frac{1}{1+x^{2}}.
$$
\underline{Indication:} On notera que $\sin t = -\text{Im}(e^{-it})$ et on n'hésitera pas à sortir l'operateur partie imaginaire de l'intégrale si nécessaire. 
\begin{solution}
La fonction $(x,t)\mapsto f(x,t) = \frac{\sin t}{t}e^{-xt}$ admet pour dérivée partielle rapport à $x$ la fonction $t\mapsto \frac{\partial f}{\partial x}(x,t) =  -\sin(t)e^{-xt}$ qui est continue sur $(0,\infty)$ pour tout $x\geq0$. 
De plus, pour tout $x > a$, avec $a>0$, 
$$
\left|\frac{\partial f}{\partial x}(x,t)\right|=|\sin t|e^{-at}<e^{-at}.
$$
On peut toujours trouver un voisinage de $x>0$ de la forme $\left]a,\infty\right[$, par le théorème de la continuité de l'intégrale à paramètre, on conclut que $F(x)$ est dérivable. On poursuit avec le calcul de la dérivée, 
\begin{eqnarray*}
F'(x) &=& \int_0^{+\infty} \frac{\partial f}{\partial x}(x,t)\text{d}t\\
&=&\int_0^{+\infty} -\sin t e^{-xt}\text{d}t\\
&=&\text{Im}\left(\int_0^{+\infty} e^{-t(i+x)}\text{d}t\right)\\
&=&\text{Im}\left(\frac{1}{i+x}\right)\\
&=&-\frac{1}{1+x^{2}}.
\end{eqnarray*}
\end{solution}
\part[2] Calculer F(x) et en déduire la valeur de $\int_0^{+\infty}\frac{\sin t}{t}\text{d}t$.
\begin{solution}
En intégrant $F'(x)$ entre $u\leq v$, on obtient 
$$F(v)-F(u)=\arctan(u)-\arctan(v).$$
Puis en faisant tendre $u$ vers l'infini il vient
$$
F(v)=\frac{\pi}{2}-\arctan(v),
$$ 
car 
$$
F(u)<\int_0^{+\infty}e^{-ut}\text{d}t=\frac{1}{u}\underset{u\rightarrow\infty}{\rightarrow} 0.
$$
On en déduit que $\int_0^{+\infty}\frac{\sin t}{t}\text{d}t=\frac{\pi}{2}$ en posant $v=0$.
\end{solution}
\end{parts} 
\end{questions}
\newpage
%-------------------------------TABLE-------------------------------
\hrule
\vspace*{.15in}
\begin{center}
	\large\MakeUppercase{Fonctions Trigonométriques}
\end{center}
\vspace*{.15in}
\hrule
\vspace*{.25in}


\renewcommand\arraystretch{3.5}
\begin{table}[H]
\begin{center}
\begin{tabular}{|c|c|c|}
\hline
Fonction & Ensemble de définition& Dérivée \\
\hline\hline
$\sin x$ &$\mathbb{R}$& $\cos x$ \\
\hline
$\cos x$ &$\mathbb{R}$& $-\sin x$ \\
\hline
$\tan x$ &$\bigcup_{n\in\mathbb{Z}}\left]n\pi-\pi/2, n\pi+\pi/2\right[$ & $1+\tan^2 x$ \\
\hline
$\arccos x$ &$[-1,1]$& $-\frac{1}{\sqrt{1-x^2}}$ \\
\hline
$\arcsin x$ &$[-1,1]$& $\frac{1}{\sqrt{1-x^2}}$\\
\hline
$\arctan x$ &$\mathbb{R}$& $\frac{1}{1+x^2}$ \\
\hline
\end{tabular}
\end{center}
\end{table}%



\end{document}

%%%% Extra problems %%%%--------------------------------------

%\question A coin having probability $0.8$ of landing on heads is flipped.  $A$ observes the result -- either heads or tails -- and rushes off to tell $B$. However, with probability $0.4$, $A$ will have forgotten the result by the time he reaches $B$.  If $A$ has forgotten, then, rather than admitting this to $B$, he is equally likely to tell $B$ that the coin landed on heads or that it landed tails. (If he does remember, then he tells $B$ the correct result).
%\begin{parts}
%	\part What is the probability that $B$ is told that the coin landed on heads?
%	\part What is the probability that $B$ is told the correct result?
%	\part Given that $B$ is told that the coin landed on heads, what is the probability that it did in fact land on heads?
%\end{parts}

%\question % Variance, expected value
%Suppose that $\bP(X=0)=1-\bP(X=1)$.  If $\E[X]=3\Var(X)$, find $\bP(X=0)$.