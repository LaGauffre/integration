\documentclass[11pt, addpoints,answers]{exam}

\usepackage[utf8]{inputenc}
\usepackage[T1]{fontenc}
\usepackage[margin  = 1in]{geometry}
\usepackage{amsmath, amscd, amssymb, amsthm, verbatim}
%\usepackage{mathabx}
\usepackage{setspace}
\usepackage{float}
\usepackage{color}
\usepackage{graphicx}
\usepackage[colorlinks=true]{hyperref}
\usepackage{tikz}
\usetikzlibrary{trees}

\shadedsolutions
\definecolor{SolutionColor}{RGB}{214,240,234}

\newcommand{\bbC}{{\mathbb C}}
\newcommand{\R}{\mathbb{R}}            % real numbers
\newcommand{\bbR}{{\mathbb R}}
\newcommand{\Z}{\mathbb{Z}}            % integers
\newcommand{\bbZ}{{\mathbb Z}}
\newcommand{\bx}{\mathbf x}            % boldface x
\newcommand{\by}{\mathbf y}            % boldface y
\newcommand{\bz}{\mathbf z}            % boldface z
\newcommand{\bn}{\mathbf n}            % boldface n
\newcommand{\br}{\mathbf r}            % boldface r
\newcommand{\bc}{\mathbf c}            % boldface c
\newcommand{\be}{\mathbf e}            % boldface e
\newcommand{\bE}{\mathbb E}            % blackboard E
\newcommand{\bP}{\mathbb P}            % blackboard P

\newcommand{\ve}{\varepsilon}          % varepsilon
\newcommand{\avg}[1]{\left< #1 \right>} % for average
%\renewcommand{\vec}[1]{\mathbf{#1}} % bold vectors
\newcommand{\grad}{\nabla }
\newcommand{\lb}{\langle }
\newcommand{\rb}{\rangle }

\def\Bin{\operatorname{Bin}}
\def\Var{\operatorname{Var}}
\def\Geom{\operatorname{Geom}}
\def\Pois{\operatorname{Pois}}
\def\Exp{\operatorname{Exp}}
\newcommand{\Ber}{\operatorname{Ber}}
\def\Unif{\operatorname{Unif}}
\def\No{\operatorname{N}}
\newcommand{\E}{\mathbb E}            % blackboard E
\def\th{\theta }            % theta shortcut
\def\V{\operatorname{Var}}
\def\Var{\operatorname{Var}}
\def\Cov{\operatorname{Cov}}
\def\Corr{\operatorname{Corr}}
\newcommand{\epsi}{\varepsilon}            % epsilon shortcut

\providecommand{\norm}[1]{\left\lVert#1\right\rVert} %norm
\providecommand{\abs}[1]{\left \lvert#1\right \rvert} %absolute value

\DeclareMathOperator{\lcm}{lcm}
\newcommand{\ds}{\displaystyle}	% displaystyle shortcut

\def\semester{2018-2019 }
\def\course{Théorie de la mesure }
\def\title{\MakeUppercase{Deuxième session}}
\def\name{Pierre-O Goffard}
%\def\name{Professor Wildman}

\setlength\parindent{0pt}

\cellwidth{.35in} %sets the minimum width of the blank cells to length
\gradetablestretch{2.5}

%\bracketedpoints
%\pointsinmargin
%\pointsinrightmargin

\begin{document}


\runningheader{\course  \vspace*{.25in}}{}{\title \vspace*{.25in}}
%\runningheadrule
\runningfooter{}{Page \thepage\ of \numpages}{}

% \firstpageheader{Name:\enspace\hbox to 2.5in{\hrulefill}\\  \vspace*{2em} Section: (circle one) TR: 3-3:50 \textbar\, TR: 5-5:50 \textbar\,  TR: 6-6:50(Xu) \textbar\,  TR: 6-6:50 }{}{Perm \#: \enspace\hbox to 1.5in{\hrulefill}\\ \vspace*{2em} Score:\enspace\hbox to .6in{\hrulefill} $/$\numpoints}
\extraheadheight{.25in}

\hrulefill

\vspace*{1em}

% Heading
{\center \textsc{\Large\title}\\
	\vspace*{1em}
	\course -- \semester\\
	Pierre-O Goffard\\
}
\vspace*{1em}

\hrulefill

\vspace*{2em}

\noindent {\bf\em Instructions:} On éteint et on range son téléphone.
\begin{itemize}
	\item La calculatrice et les appareils éléctroniques ne sont pas autorisés.
	\item Vous devez justifier vos réponses de manière claire et concise.
	\item Vous devez écrire de la manière la plus lisible possible. Souligner ou encadrer votre réponse finale.

\end{itemize}

\begin{center}
	\gradetable[h]
\end{center}

\smallskip

\begin{questions}
\question[2]

		% \part[3] Soit $E$ un ensemble infini. Soit l'application $\mu: \mathcal{P}(E)\mapsto \overline{\mathbb{R}}_+$ tel que pour $A\in\mathcal{P}(E)$, on a 
		% $$
		% \mu(A) = 
		% 	\begin{cases}
		% 		0,&\text{ si $A$ est fini,}\\
		% 		+\infty, &\text{ si $A$ est infini.}
		% 	\end{cases}
		% $$
		% L'application $\mu$ est elle une mesure?
		% \begin{solution}
		% \begin{enumerate}
		% 	\item Comme $\emptyset$ est fini alors $\mu(\emptyset) = 0$
		% 	\item Soit $(A_n)_{n\in\mathbb{N}}$ une suite de parties, deux à deux disjointes, de $E$ alors 
		% 	\begin{itemize}
		% 		\item Si $(A_n)_{n\in\mathbb{N}}$ contient au moins une partie infinie $A_j$, $j\in\mathbb{N}$ alors $\bigcup_n A_n$ est infinie et $\mu(\bigcup_n A_n)=\infty$. Comme $A_j$ est infinie alors $A_j^{c}$ est fini. De plus, comme les parties $(A_n)_{n\in\mathbb{N}}$ sont mutuellement disjointes alors $\bigcup_{k\neq j}A_k\subset A_j^{c}$ est une partie finie. On a donc 
		% 		$$\sum_n\mu(A_n) = \mu(A_j) + \sum_{k\neq j}\mu(A_k) = \infty + 0 = +\infty.$$

		% 		\item Si $(A_n)_{n\in\mathbb{N}}$ est une suite de partie finies de $E$ alors $\mu(A_n) = 0$ pour tout $n\in\mathbb{N}$ et $\sum_n\mu(A_n) = 0$ d'une part et
		% 		$\bigcup_n A_n$ est infinie puis $\mu(\bigcup_n A_n) =\infty$. On en déduit que $\mu(\bigcup_n A_n)\neq \sum_n\mu(A_n) $ puis que $\mu$ n'est pas une mesure.   
		% 	\end{itemize}
		% \end{enumerate}
			
		% \end{solution}
Soit $(\Omega, \mathcal{A},\mu)$ un espace mesuré avec $\mu$ une mesure finie. Soit $(A_n)_{n\in\mathbb{N}}$ une suite d'éléments de $\mathcal{A}$ tels que $\mu(A_n) = \mu(\Omega),$ pour tout $n\in\mathbb{N}$.\\

		Montrer que 
		$$
		\mu\left(\bigcap_{n\in\mathbb{N}}A_n\right) = \mu(\Omega)
		$$

		\begin{solution}
		On considère $A_1$ et $A_2$ les deux premiers éléments de $(A_n)_n\in\mathbb{N}$. On a $A_1\cup A_2\in \Omega$ donc $\mu(A_1\cup A_2)\leq \mu(\Omega)$, on a aussi $\mu(A_1\cup A_2)\geq \mu(A_1) = \mu(\Omega)$ donc $\mu(A_1\cup A_2)= \mu(\Omega)$. On déduit de 
		$$
		\mu(A_1\cup A_2) = \mu(A_1) + \mu(A_2) - \mu(A_1\cap A_2)
		$$
		que $\mu(A_1\cap A_2) = \mu(\Omega)$. On montre par récurrence que $\mu\left(\bigcap_{k = 1}^{n} A_k\right) = \mu(\Omega)$, la suite définie par $\left(\bigcap_{k = 1}^{n} A_k\right)_{n\in\mathbb{N}}$ est une suite décroissante d'éléments de $\mathcal{A}$. On a donc 
		$$
		\mu\left(\bigcap_{k = 1}^{n} A_k\right) = \underset{n\rightarrow \infty}{\lim}\mu(A_n) = \mu(\Omega).
		$$
		\end{solution}

\question Soit $(f_n)_{n\in\mathbb{N}}$ une suite de fonction $f_n:\mathbb{R}\mapsto\mathbb{R}$ définies par 
$$
f_n(x) = \left(1-\frac{x}{n}\right)^{n}\cos(x)\mathbb{I}_{[0,n]}(x).
$$
\begin{parts}
	\part[1] Montrer que $(f_n)_{n\in\mathbb{N}}$ est une suite de fonction mesurable.
	\begin{solution}
	$(f_n)_{n\in\mathbb{N}}$ est une suite de fonctions réelles et continues, donc mesurables 
	\end{solution}
	\part[1] Montrer que $(f_n)_{n\in\mathbb{N}}$ converge simplement vers une fonction $f$ que l'on explicitera.
	\begin{solution}
	Pour tout $x\in \mathbb{R}$, on a l'équivalence 
	$$
	\left(1-\frac{x}{n}\right)^{n}\sim e^{-x} \text{, pour }n\rightarrow+\infty 
	$$
	On en déduit que 
	$$f_n(x)\underset{n\rightarrow +\infty}{\longrightarrow} f(x)= e^{-x}\cos(x)\mathbb{I}_{\left[0,+\infty\right[}$$
	\end{solution}
	\part[4] Pour tout $n\geq1$, on pose 
	$$
	I_n = \int_{[0,n]}\left(1-\frac{x}{n}\right)^{n}\cos(x)\text{d}\lambda(x) 
	$$
	Montrer que $(I_n)_{n\in\mathbb{N}}$ est une suite convergente et donner sa limite.\\

	\noindent \underline{Indications:} On pourra utiliser l'inégalité $\ln(1-t)\leq -t$ pour $t\in\left[0,1\right[$.
	\begin{solution}
	On a 
	$$
	I_n = \int_{\mathbb{R}}f_n(x)\lambda(x), 
	$$
	où $(f_n)_n\in\mathbb{N}$ est une suite de fonctions mesurables convergeant vers $f(x) = e^{-x}\cos(x)\mathbb{I}_{\left[0,\infty\right[}$, de plus on a pour tout $x\in\mathbb{R}$,
	$$
	|f_n(x)|\leq\left|\exp\left[n\ln\left(1-\frac{x}{n}\right)\right]\right|\mathbb{I}_{[0,n]}(x)\leq e^{-x} = g(x)
	$$
	Par application du théorème de convergence dominée, la suite $(I_n)$ converge et il vient 
	\begin{eqnarray*}
	\underset{n\rightarrow\infty}{\lim} I_n& = &\int f(x)\text{d}\lambda(x)\\
	&=& \int_0^{\infty} e^{-x}\cos(x)\text{d}x\\
	&=& \Re\left\{\int_0^{\infty} e^{-x}e^{-ix}\text{d}x\right\}\\
	&=& \Re\left\{\frac{1}{1+i}\right\} =1/2\\
	\end{eqnarray*}
	\end{solution}
\end{parts}
\question Soit l'intégrale
$$I =\int_{0}^{+\infty}\frac{\ln(x)}{x^2-1}\text{d}x.$$
\begin{parts}
\part[2] Montrer que l'intégrale $I$ est bien définie.
\begin{solution}
La fonction $f(x)  = \ln(x)/(x^2-1)$ est continue sur $(0,1)\times (1,+\infty)$ donc localement intégrable. Nous devons étudier la situation en $x = 0,1,+\infty$.
\begin{itemize}
	\item Au voisinage de $0$, $\sqrt{x}f(x)\rightarrow 0$ donc $f(x) = o(x^{-1/2})$ et $f$ est intégrable en $0$.
	\item Au voisinage de $1$, on a $f(x)\rightarrow 1/2$ donc prolongeable par continuité et partant intégrable. 
	\item Au voisinage de $\infty$, on a $x^{3/2}f(x)\rightarrow 0$ donc $f(x) = o(x^{-3/2})$ et $f$ est intégrable.
\end{itemize}
\end{solution}
\part[2] Montrer que 
$$
\int_{\left[0,\infty\right[^2}\frac{1}{(1+y)(1+x^2y)}\text{d}\lambda_2(x,y) = \frac{\pi^2}{2},
$$ 
où $\lambda_2$ est la mesure de Lebesgue sur $\mathbb{R}^2$.\\

\underline{Indication:} Intégrer d'abord par rapport à $x$.
\begin{solution}
\begin{eqnarray*}
\int_{\left[0,\infty\right[^2}\frac{1}{(1+y)(1+x^2y)}\text{d}\lambda_2(x,y)&=&\int_{0}^{+\infty}\int_{0}^{+\infty}\frac{1}{(1+y)(1+x^2y)}\text{d}x\text{d}y\\
&=&\frac{\pi}{2}\int_{0}^{+\infty}\frac{1}{(1+y)\sqrt{y}}\text{d}y\\
&=&\int_{0}^{+\infty}\frac{1}{(1+y)\sqrt{y}}\text{d}y\\
&=&\frac{\pi^2}{2}
\end{eqnarray*}
\end{solution}
\part[2] Montrer que pour tout $x>0$, $x\neq1$ 
$$
\int_{0}^{\infty}\frac{1}{(1+y)(1+x^2y)}\text{d}y = \frac{2\ln(x)}{x^2 - 1}.
$$
\underline{Indication:} On effectuera une décomposition en éléments simples.
\begin{solution}
\begin{eqnarray*}
\int_{0}^{\infty}\frac{1}{(1+y)(1+x^2y)}\text{d}y&=&\int_{0}^{\infty}\frac{-1}{(x^2-1)(1+y)}+\frac{x^2}{(x^2-1)(1+x^2y)}\text{d}y\\
&=&\frac{1}{x^2-1}\left[\ln\left(\frac{1+x^2y}{1+y}\right)\right]_0^{+\infty}\\
&=&\frac{2\ln(x)}{x^2-1}
\end{eqnarray*}
\end{solution}

\part[1] En déduire que $I =\frac{\pi^2}{4}$.
\begin{solution}
Conséquence de (b) et (c)
\end{solution}
\part[1] Montrer que 
$$
I = 2\int_0^{1}\frac{\ln(x)}{x^2-1}\text{d}x.
$$
\underline{Indication:} Relation de Chasles à partir de $I =\int_{0}^{+\infty}\frac{\ln(x)}{x^2-1}$.
\begin{solution}
$$
I = \int_0^{1}\frac{\ln(x)}{x^2-1}\text{d}x+ \int_1^{+\infty}\frac{\ln(x)}{x^2-1}\text{d}x
$$
puis changement de variable $u = 1/x$ dans la deuxième intégrale
$$
I = 2\int_0^{1}\frac{\ln(x)}{x^2-1}\text{d}x
$$
\end{solution}
\part[1] Déduire de la question précédente que  
$$
\sum_{k = 0}^{+\infty}\frac{1}{(2n+1)^2} = \frac{\pi^2}{8}.
$$
\underline{Indication:} Effectuer un développement en série entière de $x\mapsto\frac{1}{1-x^{2}}$ valable pour $x\in\left[0,1\right[$.
\begin{solution}
\begin{eqnarray*}
I &=& -2\sum_{k=0}^{\infty}\int_0^{1}x^{2k}\ln(x)\text{d}x\\
&=&2\sum_{k=0}^{\infty}\frac{1}{(2k+1)^2}.
\end{eqnarray*}
puis $\sum_{k=0}^{\infty}\frac{1}{(2k+1)^2}=\pi^2/8$

\end{solution}
\end{parts}
\end{questions}

\newpage
%-------------------------------TABLE-------------------------------
\hrule
\vspace*{.15in}
\begin{center}
	\large\MakeUppercase{Fonctions Trigonométriques}
\end{center}
\vspace*{.15in}
\hrule
\vspace*{.25in}


\renewcommand\arraystretch{3.5}
\begin{table}[H]
\begin{center}
\begin{tabular}{|c|c|c|}
\hline
Fonction & Ensemble de définition& Dérivée \\
\hline\hline
$\sin x$ &$\mathbb{R}$& $\cos x$ \\
\hline
$\cos x$ &$\mathbb{R}$& $-\sin x$ \\
\hline
$\tan x$ &$\bigcup_{n\in\mathbb{Z}}\left]n\pi-\pi/2, n\pi+\pi/2\right[$ & $1+\tan^2 x$ \\
\hline
$\arccos x$ &$[-1,1]$& $-\frac{1}{\sqrt{1-x^2}}$ \\
\hline
$\arcsin x$ &$[-1,1]$& $\frac{1}{\sqrt{1-x^2}}$\\
\hline
$\arctan x$ &$\mathbb{R}$& $\frac{1}{1+x^2}$ \\
\hline
\end{tabular}
\end{center}
\end{table}%



\end{document}

%%%% Extra problems %%%%--------------------------------------

%\question A coin having probability $0.8$ of landing on heads is flipped.  $A$ observes the result -- either heads or tails -- and rushes off to tell $B$. However, with probability $0.4$, $A$ will have forgotten the result by the time he reaches $B$.  If $A$ has forgotten, then, rather than admitting this to $B$, he is equally likely to tell $B$ that the coin landed on heads or that it landed tails. (If he does remember, then he tells $B$ the correct result).
%\begin{parts}
%	\part What is the probability that $B$ is told that the coin landed on heads?
%	\part What is the probability that $B$ is told the correct result?
%	\part Given that $B$ is told that the coin landed on heads, what is the probability that it did in fact land on heads?
%\end{parts}

%\question % Variance, expected value
%Suppose that $\bP(X=0)=1-\bP(X=1)$.  If $\E[X]=3\Var(X)$, find $\bP(X=0)$.