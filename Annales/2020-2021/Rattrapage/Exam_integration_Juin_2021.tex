\documentclass[11pt, addpoints, answers]{exam}

\usepackage[utf8]{inputenc}
\usepackage[T1]{fontenc}
\usepackage[margin  = 1in]{geometry}
\usepackage{amsmath, amscd, amssymb, amsthm, verbatim}
\usepackage{mathabx}
\usepackage{setspace}
\usepackage{float}
\usepackage{color}
\usepackage{graphicx}
\usepackage[colorlinks=true]{hyperref}
\usepackage{tikz}
\usetikzlibrary{trees}

\shadedsolutions
\definecolor{SolutionColor}{RGB}{214,240,234}

\newcommand{\bbC}{{\mathbb C}}
\newcommand{\R}{\mathbb{R}}            % real numbers
\newcommand{\bbR}{{\mathbb R}}
\newcommand{\Z}{\mathbb{Z}}            % integers
\newcommand{\bbZ}{{\mathbb Z}}
\newcommand{\bx}{\mathbf x}            % boldface x
\newcommand{\by}{\mathbf y}            % boldface y
\newcommand{\bz}{\mathbf z}            % boldface z
\newcommand{\bn}{\mathbf n}            % boldface n
\newcommand{\br}{\mathbf r}            % boldface r
\newcommand{\bc}{\mathbf c}            % boldface c
\newcommand{\be}{\mathbf e}            % boldface e
\newcommand{\bE}{\mathbb E}            % blackboard E
\newcommand{\bP}{\mathbb P}            % blackboard P

\newcommand{\ve}{\varepsilon}          % varepsilon
\newcommand{\avg}[1]{\left< #1 \right>} % for average
%\renewcommand{\vec}[1]{\mathbf{#1}} % bold vectors
\newcommand{\grad}{\nabla }
\newcommand{\lb}{\langle }
\newcommand{\rb}{\rangle }

\def\Bin{\operatorname{Bin}}
\def\Var{\operatorname{Var}}
\def\Geom{\operatorname{Geom}}
\def\Pois{\operatorname{Pois}}
\def\Exp{\operatorname{Exp}}
\newcommand{\Ber}{\operatorname{Ber}}
\def\Unif{\operatorname{Unif}}
\def\No{\operatorname{N}}
\newcommand{\E}{\mathbb E}            % blackboard E
\def\th{\theta }            % theta shortcut
\def\V{\operatorname{Var}}
\def\Var{\operatorname{Var}}
\def\Cov{\operatorname{Cov}}
\def\Corr{\operatorname{Corr}}
\newcommand{\epsi}{\varepsilon}            % epsilon shortcut

\providecommand{\norm}[1]{\left\lVert#1\right\rVert} %norm
\providecommand{\abs}[1]{\left \lvert#1\right \rvert} %absolute value

\DeclareMathOperator{\lcm}{lcm}
\newcommand{\ds}{\displaystyle}	% displaystyle shortcut

\def\semester{2020-2021 }
\def\course{Théorie de la mesure et intégration}
\def\title{\MakeUppercase{Examen Deuxième Session }}
\def\name{Pierre-O. Goffard}
%\def\name{Professor Wildman}

\setlength\parindent{0pt}

\cellwidth{.35in} %sets the minimum width of the blank cells to length
\gradetablestretch{2.5}

%\bracketedpoints
%\pointsinmargin
%\pointsinrightmargin

\begin{document}


\runningheader{\course  \vspace*{.25in}}{}{\title \vspace*{.25in}}
%\runningheadrule
\runningfooter{}{Page \thepage\ of \numpages}{}

% \firstpageheader{Name:\enspace\hbox to 2.5in{\hrulefill}\\  \vspace*{2em} Section: (circle one) TR: 3-3:50 \textbar\, TR: 5-5:50 \textbar\,  TR: 6-6:50(Xu) \textbar\,  TR: 6-6:50 }{}{Perm \#: \enspace\hbox to 1.5in{\hrulefill}\\ \vspace*{2em} Score:\enspace\hbox to .6in{\hrulefill} $/$\numpoints}
\extraheadheight{.25in}

\hrulefill

\vspace*{1em}

% Heading
{\center \textsc{\Large\title}\\
	\vspace*{1em}
	\course -- \semester\\
	Pierre-O Goffard\\
}
\vspace*{1em}

\hrulefill

\vspace*{2em}

\noindent {\bf\em Instructions:} On éteint et on range son téléphone.
\begin{itemize}
	\item La calculatrice et les appareils éléctroniques ne sont pas autorisés.
	\item Vous devez justifier vos réponses de manière claire et concise.
	\item Vous devez écrire de la manière la plus lisible possible. Souligner ou encadrer votre réponse finale.
\end{itemize}

\begin{center}
	\gradetable[h]
\end{center}

\smallskip

\begin{questions}
\question[1] Soient $(\Omega_i,\mathcal{A}_i),\text{ }i = 1,2,3, $ trois espaces mesurables. Soient $f:\Omega_1\mapsto\Omega_2$ et $g:\Omega_2\mapsto\Omega_3$ deux applications mesurables. Monter que l'application $g\circ f$ est une application mesurable de $(\Omega_1,\mathcal{A}_1)$ vers $(\Omega_3,\mathcal{A}_3)$
\begin{solution}
Voir le cours
\end{solution}
\question
\begin{parts}
\part[1] Soit $(\Omega,\mathcal{A}, \mu)$ un espace mesurable et $B\in \mathcal{A}$ tel que $0<\mu(B)<\infty$. Montrer que l'application 
$$
\mu_B(A) = \frac{\mu(A\cap B)}{\mu(B)},\text{ }A\in \mathcal{A},
$$ 
est une mesure de probabilité.
\begin{solution}
Voir le cours
\end{solution}
\part[2] Supposons que $\Omega = \mathbb{N}$, $\mathcal{A} = \mathcal{P}(\mathbb{N})$ et  
$$
\mu = \sum_{k = 1}^\infty\frac{1}{2^k}\delta_k 
$$
où $\delta_k$ est la mesure de Dirac en $k\in\mathbb{N}$, définie par
$$
\delta_k(A) = \begin{cases}1&\text{ si }k\in A\\
0&\text{ sinon}
\end{cases}
$$
pour $A\in\mathcal{P}(\mathbb{N})$. Soit $B = 2\mathbb{N}$ (les entiers pairs). Calculer $\mu(B)$ et montrer que 
$$
\mu_B = \sum_{k = 1}^\infty\frac{3}{4^k}\delta_{2k}. 
$$
\begin{solution}
On a 
$$
\mu(B) =  \mu\left(\bigcup_{n\in\mathbb{N}^\ast}\{2k\}\right) =\sum_{n\in\mathbb{N}^\ast}\frac{1}{4^k} = \frac{1}{4}\frac{1}{1-1/4} = 1/3 
$$
Soit $A\in\mathcal{A}$ alors 
$$
\mu_B(A) = 3\sum_{n\in\mathbb{N}^\ast}\frac{1}{2^{2k}}\delta_{2k}(A) = \sum_{n\in\mathbb{N}^\ast}\frac{3}{4^{k}}\delta_{2k}(A)
$$
\end{solution}
\end{parts}

\question L'objectif est de montrer que
$$
\int_{0}^1\frac{\text{d}x}{x^x} = \sum_{n=1}^\infty\frac{1}{n^n}
$$ 
\begin{parts}
\part[1] Montrer que $x\mapsto 1/x^x\mathbb{I}_{\left]0, 1\right[}$ est intégrable.
\begin{solution}
Il s'agit d'une fonction continue sur un intervalle borné.
\end{solution}
\part[2] Montrer que (en justifiant les étapes)
$$
\int_{0}^1\frac{\text{d}x}{x^x} = \sum_{n=0}^\infty\int_{0}^1\frac{(-x\ln x)^n }{n!}\text{d}x
$$ 
\begin{solution}
On a 
$$
\int_{0}^1\frac{\text{d}x}{x^x} = \int_{0}^1\sum_{n=0}^\infty\frac{(-x\ln x)^n }{n!}\text{d}x
$$
On inverse ensuite les signes somme et intégrale via le théorème de Beppo-Lévi (convergence monotone)
\end{solution}
\part[2] Pour tout $(m,n)\in \mathbb{N}\times \mathbb{N}^\ast$, on pose
$$
I_{m,n} = \int_{0}^1 x^n(\ln x)^m\text{d}x
$$
Montrer par récurrence sur $m\in\mathbb{N}$ que 
$$
I_{m,n} = (-1)^m\frac{m!}{(n+1)^{m+1}}.
$$ 
\begin{solution}
On a 
$$
I_{0,n}=\frac{1}{n+1}
$$
donc la propriété est vérifiée au rang $0$. Supposons la propriété vérifiée au rang $m$, qu'en est il au rang $n+1$? On a 
$$
I_{m+1,n} = \int_{0}^1 x^{n}(\ln x)^{m+1}\text{d}x\overset{IPP}{=}-\frac{m+1}{n+1}\int_{0}^1 x^{n}(\ln x)^{m}\text{d}x=(-1)^{m+1}\frac{(m+1)!}{(n+1)^{m+1}}
$$
et la propriété est vérifiée au rang $m+1$.

\end{solution}
\part[1] Conclure.
\begin{solution}
On remarque que 
$$
\sum_{n=0}^\infty\int_{0}^1\frac{(-x\ln x)^n }{n!}\text{d}x = \sum_{n=0}^\infty I_{n,n} =  \sum_{n=1}^\infty\frac{1}{n^n}.
$$ 
\end{solution}
\end{parts}
\question[2] Calculer, en justifiant, 
$$
\underset{n\rightarrow\infty}{\lim}\int_{\left[0,\infty\right[}\frac{e^{-n x ^2}}{n}\text{d}\lambda(x).
$$
\begin{solution}
Les fonctions
$$
f_n(x) = \frac{e^{-n x ^2}}{n}, \text{ }n\geq0,
$$
sont intégrables et converge simplement vers $0$. De plus, on a $|f_n(x)|<e^{-x}$. On en déduit par convergence dominée que 
$$
\underset{n\rightarrow\infty}{\lim}\int_{\left[0,\infty\right[}\frac{e^{-n x ^2}}{n}\text{d}\lambda(x)=0.
$$

\end{solution}
\question Pour $n\geq1$ et $x>0$, on pose 
$$
I_n(x) = \int_{0}^\infty\frac{\text{dt}}{(x^2+t^2)^n}
$$
\begin{parts}
\part[1] Calculer $I_1(x)$
\begin{solution}
$$
I_1(x) = \frac{\pi}{2x}
$$
\end{solution}
\part[1] Montrer que $x\mapsto I_n(x)$ est continue 
\begin{solution}
On note $f(t,x) = (x^2+t^2)^{-n}$. La fonction $t\mapsto f(t,x)$ est intégrable sur $(0,+\infty)$. De plus $f(t,x)< t^{-2}$ pour tout $x>0$ donc $I_n(x)$ est continue.
\end{solution}
\part[2] Montrer que $x\mapsto I_n(x)$ est dérivable et trouver une relation entre $I_n'(x)$ et $I_{n+1}(x)$.
\begin{solution}
La fonction $t\mapsto \frac{\partial}{\partial x}f(t,x) = -2nx^2(t^2+x^2)^{-n-1}$ est intégrable sur $(0,\infty)$ et pour $x\in[a,b]$ avec $0<a<b<\infty$ on a 
$$
\left\rvert\frac{\partial}{\partial x}f(t,x)\right\rvert \leq \frac{2bn}{(a^2+t^2)^{n+1}}
$$
On peut choisir $a$ et $b$ arbitrairement proche de $0$ et $\infty$ respectivement donc $I_n(x)$ est bien dérivable. On a de plus 

$$
I_n'(x) = -nx I_n(x)\text{ (IPP)}
$$
\end{solution}
\part[2] Montrer par récurrence l'existence d'une suite $(\lambda_n)_{n\geq1}$ telle que 
$$
I_n(x) = \frac{\lambda_n}{x^{2n-1}},
$$
Donner la valeur de $\lambda_n$ en fonction de $n$ (avec des factoriels).
\begin{solution}
Au rang $1$, la propriété est vérifié avec $\lambda_1 = \pi/2$. Supposons la propriété vérifiée au rang $n$, qu'en est il au rang $n+1$? On a 
$$
I_{n+1}(x) = -\frac{I_n'(x)}{nx} = \lambda_n\frac{2n-1}{n}\frac{1}{x^{2n+1}}.
$$
La propritété est vérifiée au rang $n+1$. On a 
$$
\lambda_n = \frac{2n-3}{n-1}\lambda_{n-1}\lambda_{n-1} = \frac{(2n-2)!}{2^{n-1}(n-1)!}\frac{\pi}{2}
$$
\end{solution}
\end{parts}
\question[2] Calculer l'intégrale
$$
\int\int_{\mathcal{D}}\frac{3y}{\sqrt{1+(x+y)^3}}\text{d}\lambda(x,y),
$$
où $\mathcal{D} = \{x,y>0\text{ }x+y < a\}$, avec $a\in\mathbb{R}$. On pourra utiliser le changement de variable
$(u,v):=(x,y)\mapsto \phi(x,y) = (x+y, x-y)$.
\begin{solution}
On applique la formule de changement de variable
$$
\int\int_{\mathcal{D}}\frac{3y}{\sqrt{1+(x+y)^3}}\text{d}\lambda(x,y) = \frac{3}{4}\int_{0}^a\int_{0}^u\frac{u-v}{\sqrt{1+u^3}}\text{d}\lambda(u,v) = \frac{\sqrt{1+a^3}-1}{4}
$$
\end{solution}



% \begin{parts}
% \part df
% \end{parts}
\end{questions}

\newpage
%-------------------------------TABLE-------------------------------
\hrule
\vspace*{.15in}
\begin{center}
	\large\MakeUppercase{Fonctions Trigonométriques}
\end{center}
\vspace*{.15in}
\hrule
\vspace*{.25in}


\renewcommand\arraystretch{3.5}
\begin{table}[H]
\begin{center}
\begin{tabular}{|c|c|c|}
\hline
Fonction & Ensemble de définition& Dérivée \\
\hline\hline
$\sin x$ &$\mathbb{R}$& $\cos x$ \\
\hline
$\cos x$ &$\mathbb{R}$& $-\sin x$ \\
\hline
$\tan x$ &$\bigcup_{n\in\mathbb{Z}}\left]n\pi-\pi/2, n\pi+\pi/2\right[$ & $1+\tan^2 x$ \\
\hline
$\arccos x$ &$[-1,1]$& $-\frac{1}{\sqrt{1-x^2}}$ \\
\hline
$\arcsin x$ &$[-1,1]$& $\frac{1}{\sqrt{1-x^2}}$\\
\hline
$\arctan x$ &$\mathbb{R}$& $\frac{1}{1+x^2}$ \\
\hline
\end{tabular}
\end{center}
\end{table}%
\underline{Quelques identités}: Pour $a,b\in\mathbb{R}$,\\
$$
\cos(a+b) = \cos(a)\cos(b)-\sin(a)\sin(b)
$$
$$
\sin(a+b) = \sin(a)\cos(b)+\cos(a)\sin(b)
$$
$$
\cos^2(a)+\sin^2(a) = 1
$$


\end{document}

