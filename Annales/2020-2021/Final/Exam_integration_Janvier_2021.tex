\documentclass[11pt, addpoints, answers]{exam}

\usepackage[utf8]{inputenc}
\usepackage[T1]{fontenc}
\usepackage[margin  = 1in]{geometry}
\usepackage{amsmath, amscd, amssymb, amsthm, verbatim}
\usepackage{mathabx}
\usepackage{setspace}
\usepackage{float}
\usepackage{color}
\usepackage{graphicx}
\usepackage[colorlinks=true]{hyperref}
\usepackage{tikz}
\usetikzlibrary{trees}

\shadedsolutions
\definecolor{SolutionColor}{RGB}{214,240,234}

\newcommand{\bbC}{{\mathbb C}}
\newcommand{\R}{\mathbb{R}}            % real numbers
\newcommand{\bbR}{{\mathbb R}}
\newcommand{\Z}{\mathbb{Z}}            % integers
\newcommand{\bbZ}{{\mathbb Z}}
\newcommand{\bx}{\mathbf x}            % boldface x
\newcommand{\by}{\mathbf y}            % boldface y
\newcommand{\bz}{\mathbf z}            % boldface z
\newcommand{\bn}{\mathbf n}            % boldface n
\newcommand{\br}{\mathbf r}            % boldface r
\newcommand{\bc}{\mathbf c}            % boldface c
\newcommand{\be}{\mathbf e}            % boldface e
\newcommand{\bE}{\mathbb E}            % blackboard E
\newcommand{\bP}{\mathbb P}            % blackboard P

\newcommand{\ve}{\varepsilon}          % varepsilon
\newcommand{\avg}[1]{\left< #1 \right>} % for average
%\renewcommand{\vec}[1]{\mathbf{#1}} % bold vectors
\newcommand{\grad}{\nabla }
\newcommand{\lb}{\langle }
\newcommand{\rb}{\rangle }

\def\Bin{\operatorname{Bin}}
\def\Var{\operatorname{Var}}
\def\Geom{\operatorname{Geom}}
\def\Pois{\operatorname{Pois}}
\def\Exp{\operatorname{Exp}}
\newcommand{\Ber}{\operatorname{Ber}}
\def\Unif{\operatorname{Unif}}
\def\No{\operatorname{N}}
\newcommand{\E}{\mathbb E}            % blackboard E
\def\th{\theta }            % theta shortcut
\def\V{\operatorname{Var}}
\def\Var{\operatorname{Var}}
\def\Cov{\operatorname{Cov}}
\def\Corr{\operatorname{Corr}}
\newcommand{\epsi}{\varepsilon}            % epsilon shortcut

\providecommand{\norm}[1]{\left\lVert#1\right\rVert} %norm
\providecommand{\abs}[1]{\left \lvert#1\right \rvert} %absolute value

\DeclareMathOperator{\lcm}{lcm}
\newcommand{\ds}{\displaystyle}	% displaystyle shortcut

\def\semester{2020-2021 }
\def\course{Théorie de la mesure et intégration}
\def\title{\MakeUppercase{Examen Final}}
\def\name{Pierre-O. Goffard}
%\def\name{Professor Wildman}

\setlength\parindent{0pt}

\cellwidth{.35in} %sets the minimum width of the blank cells to length
\gradetablestretch{2.5}

%\bracketedpoints
%\pointsinmargin
%\pointsinrightmargin

\begin{document}


\runningheader{\course  \vspace*{.25in}}{}{\title \vspace*{.25in}}
%\runningheadrule
\runningfooter{}{Page \thepage\ of \numpages}{}

% \firstpageheader{Name:\enspace\hbox to 2.5in{\hrulefill}\\  \vspace*{2em} Section: (circle one) TR: 3-3:50 \textbar\, TR: 5-5:50 \textbar\,  TR: 6-6:50(Xu) \textbar\,  TR: 6-6:50 }{}{Perm \#: \enspace\hbox to 1.5in{\hrulefill}\\ \vspace*{2em} Score:\enspace\hbox to .6in{\hrulefill} $/$\numpoints}
\extraheadheight{.25in}

\hrulefill

\vspace*{1em}

% Heading
{\center \textsc{\Large\title}\\
	\vspace*{1em}
	\course -- \semester\\
	Pierre-O Goffard\\
}
\vspace*{1em}

\hrulefill

\vspace*{2em}

\noindent {\bf\em Instructions:} On éteint et on range son téléphone.
\begin{itemize}
	\item La calculatrice et les appareils éléctroniques ne sont pas autorisés.
	\item Vous devez justifier vos réponses de manière claire et concise.
	\item Vous devez écrire de la manière la plus lisible possible. Souligner ou encadrer votre réponse finale.
\end{itemize}

\begin{center}
	\gradetable[h]
\end{center}

\smallskip

\begin{questions}
\question Soit $(\Omega, \mathcal{A},\mu)$ un espace mesurable.
\begin{parts}
\part[1] Soit $(A_n)_{n\geq1}$ une suite croissante d'évènements de $\mathcal{A}$. Montrer que $(\mu(A_n))_{n\geq1}$ définit une suite croissante qui converge vers $\mu(A)$, où $A = \bigcup_{n\geq1}A_n$.
\begin{solution}
Voir le cours
\end{solution}
\part[2] Soit $(B_n)_{n\geq1}$ une suite décroissante d'évènements de $\mathcal{A}$ telle que $\mu(B_1)<\infty$. Montrer que $(\mu(B_n))_{n\geq1}$ définit une suite décroissante qui converge vers $\mu(B)$, où $B = \bigcap_{n\geq1}B_n$. 
\begin{solution}
Comme la suite $(B_n)_{n\geq1}$ est décroissante alors $B_{n+1}\subset B_n$, ce qui implique $\mu(B_n)\geq\mu(B_{n+1})$. La suite $E_n = B_1/ B_n, n \geq 1$ est une suite croissante d'évènements qui converge vers $B_1/B$. D'après la question précédente, nous avons $\lim \mu(E_n) = \mu(B_1)-\mu(B)$
On note que 
$$
\mu(E_n) = \mu(B_1)-\mu(B_n),\text{ pour tout }n\geq1.
$$
on obtient $\lim \mu(B_n) =  \mu(B)$ en passant à la limite l'équation précédente. 
\end{solution}
\part[1] Soit $(F_n)_{n\geq 1}$ une suite d'évènements de $\mathcal{A}$ tels que $\mu\left(\bigcup_{n\geq1}F_n\right)<\infty$, montrer que 
$$
\mu\left(\bigcap_{n\geq1}\bigcup_{k\geq n} F_k\right) = \underset{n\rightarrow \infty}{\lim}\mu\left(\bigcup_{k\geq n} F_k\right).
$$
\underline{Indication:} Utiliser le résultat des questions précédentes.
\begin{solution}
La suite d'évènements $\left(\bigcup_{k\geq n} F_k\right)_{n\geq1}$ est décroissante, il ne reste qu'à appliquer le résultat précédent.
\end{solution}
\part[1] Montrer que 
$$
\sum_{n\geq1}\mu(F_n) < \infty \Rightarrow  \mu\left(\bigcap_{n\geq1}\bigcup_{k\geq n} F_k\right) = 0.
$$
\underline{Indication:} Utiliser le résultat des questions précédentes.
\begin{solution}
Il s'agit du lemme de Borel Cantelli partie I
\end{solution}
\end{parts}
\question[3] Soit $f$ une application mesurable de $(\Omega, \mathcal{A},\mu)$ à valeurs dans $\mathbb{R}_+$ telle que 
$$
\int_\Omega f\text{d}\mu<\infty.
$$
Montrer que 
$$
\mu(\{\omega\in\Omega\text{ ; }f(\omega) = \infty\})=0.
$$ 
\underline{Indication:} On pourra introduire par exemple la suite de fonction définie par $f_n = n\mathbb{I}_{f>n},\text{ }n\geq1$.
\begin{solution}
Les évènements 
$$
A_n = \{f>n\},\text{ }n\geq1
$$
forme une suite décroissante d'évènement de limite $\{f = \infty\}$, on en déduit que $\mu(A_n)  \rightarrow \mu(\{f=\infty\})$. Comme $f_n<f$ pour tout $n\geq1$ alors 
$$
\mu(A_n) = \frac{1}{n}\int f_n\text{d}\mu <\frac{1}{n}\int f\text{d}\mu.
$$
Par passage à la limite, il vient $\mu(\{f=\infty\}) = 0$.
\end{solution}
\question Calculer, en justifiant, les limites ($n\rightarrow+\infty$) des suites 
\begin{parts}
\part[2] 
$$
u_n = \int_{0}^{1}\frac{n}{1+x^2}\sin\left(\frac{x}{n}\right)\text{d}\lambda(x).
$$ 
\begin{solution}
On définit la suite de fonctions $f_n(x) = \frac{n}{1+x^2}\sin\left(\frac{x}{n}\right), n\geq0$ continue sur $[0,1]$ et donc intégrable. On note que 
$$
f_n(x)\rightarrow \frac{x}{1+x^2},\forall x\in [0, 1].
$$
Comme 
$$
|f_n(x)|<\frac{x}{1+x^2},\forall n\geq 1,
$$
alors d'après le théorème de convergence dominée
$$
u_n\rightarrow \frac{\ln(2)}{2}.
$$
\end{solution}
\part[2] $$
u_n = \int_0^n \left(1-\frac{x}{n}\right)^n x^m\text{d}\lambda(x)\text{, pour }m\in\mathbb{N}.$$

\begin{solution}
On définit la suite de fonctions $f_n(x) = \left(1-\frac{x}{n}\right)^n x^m\mathbb{I}_{x\in[0,n]}(x), n\geq0$ continue sur $[0,n]$ et donc intégrable. On note que 
$$
f_n(x)\rightarrow e^{-x}x^m\mathbb{I}_{x\in[0,\infty[}(x),\forall x\in [0, +\infty[.
$$
Comme 
$$
|f_n(x)|<e^{-x}x^m,\forall n\geq 1,
$$
alors d'après le théorème de convergence dominée
$$
u_n\rightarrow m!\text{ (IPP répétée)}
$$

\end{solution}
\end{parts}
\question Le but de l'exercice est de montrer que 
$$
\sum_{n\geq1}\frac{1}{n^2} = \frac{\pi^2}{6}
$$
\begin{parts}
\part[2] Montrer que (on énoncera clairement le(s) théorème(s) utilisés)
$$
\int_{[0,1]^2}\frac{1}{1-xy}\text{d}\lambda(x,y) = \sum_{n\geq1}\frac{1}{n^2}.
$$
\underline{Indication:} On pourra considérer le développement en série entière de $s\mapsto\frac{1}{1-s}$.
\begin{solution}
Le développement en série entière donne 
\begin{eqnarray}
\int_{[0,1]^2}\sum_{n = 0}^\infty x^ny^n\text{d}\lambda(x,y)&\overset{\text{Beppo-Levi}}{=}&\sum_{n = 0}^\infty\int_{[0,1]^2}\frac{1}{1-xy}\text{d}\lambda(x,y)\\
&\overset{\text{Fubini-Tonelli}}{=}&\sum_{n\geq0}\frac{1}{n^2}.
\end{eqnarray}
\end{solution}
\part[2] On note $I = \int_{[0,1]^2}\frac{1}{1-xy}\text{d}\lambda(x,y)$, montrer que 
$$
I = 4\left(\int_{0}^{1/\sqrt{2}}\int_{0}^{u}\frac{\text{d}v}{2-u^2+v^2}\text{d}u+\int_{1/\sqrt{2}}^{\sqrt{2}}\int_{0}^{\sqrt{2} - u}\frac{\text{d}v}{2-u^2+v^2}\text{d}u\right)
$$
\underline{Indication:} On pourra considérer le changement de variable $(x,y) = \phi(u,v) = \left(\frac{u-v}{\sqrt{2}},\frac{u+v}{\sqrt{2}}\right)$, il est aussi vivement conseillé de faire un dessin pour représenter le nouveau domaine d'intégration $\phi^{-1}([0,1]^2)$ dans le repère $(u,v)$.
\begin{solution}
La fonction réciproque de $\phi$ est donnée par
$$(u,v) = \phi^{-1}(x,y) = \left(\frac{\sqrt{2}}{2}(x+y), \frac{\sqrt{2}}{2}(y-x)\right)$$
Le Jacobien (en valeur absolue) est donnée par $|\text{Jac}_{\phi^{-1}}| = 1$. Le changement de variable consiste à effectuer une rotation des axes d'angle $\pi/4$ dans le sens horaire, le domaine d'intégration demeure un carré sauf que ces sommets se situent aux points $(0,0)$, $(1/\sqrt{2},1/\sqrt{2})$, $(1/\sqrt{2},-1/\sqrt{2})$ et $(0,\sqrt{2})$. Par symétrie du domaine d'intégration par rapport à l'axe $u$ et parité de la fonction par rapport à sa deuxième variable on obtient
$$
I = 4\left(\int_{0}^{1/\sqrt{2}}\int_{0}^{u}\frac{\text{d}v}{2-u^2+v^2}\text{d}u+\int_{1/\sqrt{2}}^{\sqrt{2}}\int_{0}^{\sqrt{2} - u}\frac{\text{d}v}{2-u^2+v^2}\text{d}u\right)
$$
\end{solution}
\part[2] Evaluer $$I_1 = \int_{0}^{1/\sqrt{2}}\int_{0}^{u}\frac{\text{d}v}{2-u^2+v^2}\text{d}u.$$ 
\underline{Indication:} On rappelle que 
$$
\int_{0}^x \frac{\text{d}t}{a^2+t^2} = \frac{1}{a}\arctan\left(\frac{x}{a}\right). 
$$
On pourra considérer le changement de variable $u = \sqrt{2}\sin\theta$.
\begin{solution}
$$
I_1 = \frac{\pi^2}{72}
$$
\end{solution}
\part[2] Evaluer 
$$
I_2 = \int_{1/\sqrt{2}}^{\sqrt{2}}\int_{0}^{\sqrt{2} - u}\frac{\text{d}v}{2-u^2+v^2}\text{d}u.
$$
et conclure.\\
\underline{Indication:} On pourra considérer le changement de variable $u = \sqrt{2}\cos(2\theta)$
\begin{solution}
$$
I_2 = \frac{\pi^2}{36}
$$
\end{solution}

\end{parts}
\end{questions}

\newpage
%-------------------------------TABLE-------------------------------
\hrule
\vspace*{.15in}
\begin{center}
	\large\MakeUppercase{Fonctions Trigonométriques}
\end{center}
\vspace*{.15in}
\hrule
\vspace*{.25in}


\renewcommand\arraystretch{3.5}
\begin{table}[H]
\begin{center}
\begin{tabular}{|c|c|c|}
\hline
Fonction & Ensemble de définition& Dérivée \\
\hline\hline
$\sin x$ &$\mathbb{R}$& $\cos x$ \\
\hline
$\cos x$ &$\mathbb{R}$& $-\sin x$ \\
\hline
$\tan x$ &$\bigcup_{n\in\mathbb{Z}}\left]n\pi-\pi/2, n\pi+\pi/2\right[$ & $1+\tan^2 x$ \\
\hline
$\arccos x$ &$[-1,1]$& $-\frac{1}{\sqrt{1-x^2}}$ \\
\hline
$\arcsin x$ &$[-1,1]$& $\frac{1}{\sqrt{1-x^2}}$\\
\hline
$\arctan x$ &$\mathbb{R}$& $\frac{1}{1+x^2}$ \\
\hline
\end{tabular}
\end{center}
\end{table}%
\underline{Quelques identités}: Pour $a,b\in\mathbb{R}$,\\
$$
\cos(a+b) = \cos(a)\cos(b)-\sin(a)\sin(b)
$$
$$
\sin(a+b) = \sin(a)\cos(b)+\cos(a)\sin(b)
$$
$$
\cos^2(a)+\sin^2(a) = 1
$$


\end{document}

