\documentclass[11pt, addpoints, answers]{exam}

\usepackage[utf8]{inputenc}
\usepackage[T1]{fontenc}
\usepackage[margin  = 1in]{geometry}
\usepackage{amsmath, amscd, amssymb, amsthm, verbatim}
\usepackage{mathabx}
\usepackage{setspace}
\usepackage{float}
\usepackage{color}
\usepackage{graphicx}
\usepackage[colorlinks=true]{hyperref}
\usepackage{tikz}
\usetikzlibrary{trees}

\shadedsolutions
\definecolor{SolutionColor}{RGB}{214,240,234}

\newcommand{\bbC}{{\mathbb C}}
\newcommand{\R}{\mathbb{R}}            % real numbers
\newcommand{\bbR}{{\mathbb R}}
\newcommand{\Z}{\mathbb{Z}}            % integers
\newcommand{\bbZ}{{\mathbb Z}}
\newcommand{\bx}{\mathbf x}            % boldface x
\newcommand{\by}{\mathbf y}            % boldface y
\newcommand{\bz}{\mathbf z}            % boldface z
\newcommand{\bn}{\mathbf n}            % boldface n
\newcommand{\br}{\mathbf r}            % boldface r
\newcommand{\bc}{\mathbf c}            % boldface c
\newcommand{\be}{\mathbf e}            % boldface e
\newcommand{\bE}{\mathbb E}            % blackboard E
\newcommand{\bP}{\mathbb P}            % blackboard P

\newcommand{\ve}{\varepsilon}          % varepsilon
\newcommand{\avg}[1]{\left< #1 \right>} % for average
%\renewcommand{\vec}[1]{\mathbf{#1}} % bold vectors
\newcommand{\grad}{\nabla }
\newcommand{\lb}{\langle }
\newcommand{\rb}{\rangle }

\def\Bin{\operatorname{Bin}}
\def\Var{\operatorname{Var}}
\def\Geom{\operatorname{Geom}}
\def\Pois{\operatorname{Pois}}
\def\Exp{\operatorname{Exp}}
\newcommand{\Ber}{\operatorname{Ber}}
\def\Unif{\operatorname{Unif}}
\def\No{\operatorname{N}}
\newcommand{\E}{\mathbb E}            % blackboard E
\def\th{\theta }            % theta shortcut
\def\V{\operatorname{Var}}
\def\Var{\operatorname{Var}}
\def\Cov{\operatorname{Cov}}
\def\Corr{\operatorname{Corr}}
\newcommand{\epsi}{\varepsilon}            % epsilon shortcut

\providecommand{\norm}[1]{\left\lVert#1\right\rVert} %norm
\providecommand{\abs}[1]{\left \lvert#1\right \rvert} %absolute value

\DeclareMathOperator{\lcm}{lcm}
\newcommand{\ds}{\displaystyle}	% displaystyle shortcut

\def\semester{2021-2022 }
\def\course{Théorie de la mesure et intégration}
\def\title{\MakeUppercase{Examen de deuxième session}}
\def\name{Pierre-O. Goffard}
%\def\name{Professor Wildman}

\setlength\parindent{0pt}

\cellwidth{.35in} %sets the minimum width of the blank cells to length
\gradetablestretch{2.5}

%\bracketedpoints
%\pointsinmargin
%\pointsinrightmargin

\begin{document}


\runningheader{\course  \vspace*{.25in}}{}{\title \vspace*{.25in}}
%\runningheadrule
\runningfooter{}{Page \thepage\ of \numpages}{}

% \firstpageheader{Name:\enspace\hbox to 2.5in{\hrulefill}\\  \vspace*{2em} Section: (circle one) TR: 3-3:50 \textbar\, TR: 5-5:50 \textbar\,  TR: 6-6:50(Xu) \textbar\,  TR: 6-6:50 }{}{Perm \#: \enspace\hbox to 1.5in{\hrulefill}\\ \vspace*{2em} Score:\enspace\hbox to .6in{\hrulefill} $/$\numpoints}
\extraheadheight{.25in}

\hrulefill

\vspace*{1em}

% Heading
{\center \textsc{\Large\title}\\
	\vspace*{1em}
	\course -- \semester\\
	Pierre-O Goffard\\
}
\vspace*{1em}

\hrulefill

\vspace*{2em}

\noindent {\bf\em Instructions:} On éteint et on range son téléphone.
\begin{itemize}
	\item La calculatrice et les appareils éléctroniques ne sont pas autorisés.
	\item Vous devez justifier vos réponses de manière claire et concise.
	\item Vous devez écrire de la manière la plus lisible possible. Souligner ou encadrer votre réponse finale.
	\item \underline{Document autorisé:} Une feuille manuscrite recto-verso
\end{itemize}

\begin{center}
	\gradetable[h]
\end{center}

\smallskip

\begin{questions}
\question Question de cours indépendantes
\begin{parts}
\part[2] Soit $f:\Omega\mapsto \overline{\mathbb{R}}$ une application mesurable et intégrable par rapport à une mesure $\mu$. Montrer que l'ensemble $\{f = \infty\}$ est de mesure nulle, autrement dit que $f$ est finie $\mu-pp$. 
\begin{solution}
Il s'agit d'un résultat du cours.
\end{solution}
\part[2] Soient $\alpha, \beta>0$ et $\Delta = \{(x,y)\in \mathbb{R}^2\text{ ; }0<x<y\}$. Calculer
\[
\int_{\mathbb{R}^2}e^{-\alpha x}e^{-\beta x}\mathbb{I}_\Delta(x,y)\text{d}\lambda(x,y).
\]
\begin{solution}
$$
\frac{1}{\beta(\alpha + \beta)}
$$
\end{solution}
\part[2]Calculer 
\[
\underset{n\rightarrow+\infty}{\lim}\int_{0}^{+\infty}\frac{4t^3+12}{12t^6+3nt + 2}\text{d}t.
\]
\begin{solution}
On définit $f_n(t) = \frac{4t^3+12}{12t^6+3nt + 2}$, on a 
$$
|f_n(t)|\leq \frac{4t^3+12}{12t^6+ 2}\text{, pour tout }n\geq0
$$
Par application du théorème de convergence dominée, il vient 
\[
\int f_n(t)\text{d}t\rightarrow 0\text{ pour }n\rightarrow +\infty.
\]
\end{solution}
\end{parts}
\question Soit $f:\mathbb{R}^d\mapsto \mathbb{R}_+$ une fonction mesurable tel que $0<\int_{\mathbb{R}^d} f\text{d}\lambda<\infty$ où $\lambda$ désigne la mesure de Lebesgue sur $\mathbb{R}^d$. Soit $\alpha>0$ un paramètre, on définit la suite 
$$
a_n = \int_{\mathbb{R^d}} n\log\left[1+\left(\frac{f(x)}{n}\right)^\alpha\right]\text{d}\lambda(x),\text{ }n\geq1.
$$
\begin{parts}
\part[1] Montrer que 
$$
\lambda\left(\{x\in\mathbb{R}^d\text{ : }f(x)\neq0\}\right)>0.
$$
\begin{solution}
Par le cours nous avons l'équivalence
$$
\int f\text{d}\lambda = 0\Leftrightarrow f = 0 \text{ }\lambda-pp
$$
Ici comme $\int f\text{d}\lambda >0$ alors par contraposée $f$ n'est pas nulle presque partout. 
\end{solution}
\part[1] Montrer que 
$$
\underset{n\rightarrow\infty}{\lim}a_n = \infty,
$$
pour $\alpha<1$.\\
\underline{Hint:} Penser au Lemme de Fatou
\begin{solution}
La suite de fonction 
$$
f_n(x) = n\log\left[1+\left(\frac{f(x)}{n}\right)^\alpha\right]
$$
est une suite de fonction mesurable positve. Le lemme de Fatou s'applique, nous avons donc
$$
\int \liminf f_n\text{d}\lambda < \liminf \int  f_n\text{d}\lambda
$$ 
Or $f_n(x)\sim n^{1-\alpha}f(x)\rightarrow +\infty $ pour $n\rightarrow \infty$ et $x\in\mathbb{R}^d$. On en déduit que 
$ \liminf \int  f_n\text{d}\lambda = \infty$ et donc $\underset{n\rightarrow\infty}{\lim}a_n = \infty$. 
\end{solution}

\part[1] Montrer que 
$$
\underset{n\rightarrow\infty}{\lim}a_n = \int_{\mathbb{R}^d} f\text{d}\lambda,
$$
pour $\alpha=1$.
\begin{solution}
Pour $x\geq 0$, nous avons l'inégalité $\log(1+x)\leq x$. Nous en déduisons que 
$$
|f_n(x)|\leq f(x).
$$ 
De plus comme $f_n(x)\rightarrow f(x)$ alors il vient 
$$
\underset{n\rightarrow\infty}{\lim}a_n = \int_{\mathbb{R}^d} f\text{d}\lambda,
$$
par convergence dominée.

\end{solution}
\part[2] Montrer que 
$$
\underset{n\rightarrow\infty}{\lim}a_n = 0,
$$
pour $\alpha>1$.\\

\underline{Hint:} On peut commencer par démontrer que la fonction $y\mapsto \frac{\log(1+y^\alpha)}{y}$ est bornée sur $\mathbb{R}_+$.
\end{parts}

\begin{solution}
Pour $x\in (0,\infty)$ la fonction $y\mapsto \frac{\log(1+y^\alpha)}{y}$ est continue. De plus on a 
$$
\frac{\log(1+y^\alpha)}{y}\sim y^{\alpha-1}\rightarrow 0\text{ pour }y\rightarrow 0
$$
et
$$
\frac{\log(1+y^\alpha)}{y}\rightarrow 0\text{ pour }y\rightarrow \infty
$$
On en déduit (en vertu de la continuité de la fonction et le fait que en $0$ et $\infty$ la limite soit $0$) l'existence d'une constante $C_\alpha$ tel que 
$$
\frac{\log(1+y^\alpha)}{y} <C_\alpha.
$$
En prenant $y = f(x)/n$, on obtient que 
$$
|f_n(x)|<f(x)
$$
puis 
$$
\underset{n\rightarrow\infty}{\lim}a_n = 0
$$
par convergence dominée.
\end{solution}
\question Soit $f:\mathbb{R}_+\mapsto \overline{\mathbb{R}}$ une fonction mesurable et intégrable par rapport à la mesure de Lebesgue $\lambda$. La transformée de Laplace de la fonction $f$ est définie par 
$$
L_f(t) = \int_{\mathbb{R}_+}e^{-tx}f(x)\text{d}\lambda(x),\text{ pour }t\geq0.
$$
\begin{parts}
\part[1] Montrer que $x\mapsto e^{-tx}f(x)$ est mesurable pour tout $t\in\mathbb{R}_+$.
\begin{solution}
Il s'agit du produit de deux fonctions mesurables.
\end{solution}
\part[1] Montrer que $t\mapsto L_f(t)$ est une fonction continue.
\begin{solution}
Soit $g(x,t) = e^{-tx}f(x)$, l'application $t\mapsto e^{-tx}f(x)$ est continue pour tout $x\geq0$, de plus on a 
$$
|g(x,t)|\leq|f(x)|.
$$
On en déduit la continuité de $L_f(t)$ en vertu du théorème de continuité de l'intégrale dépendant d'un paramètre.
\end{solution}
\part[2] Montrer que $\underset{t\rightarrow+\infty}{\lim}L_f(t) = 0$.
\begin{solution}
Grâce à la continuité de $t\mapsto L_f(t)$, nous pouvons montrer que $\underset{n\rightarrow\infty}{\lim} L_f(t_n)= 0$ pour une suite $(t_n)_{n\geq 0}$ telle que $t_n\rightarrow\infty$. Considérons une telle suite et définissons la suite de fonctions
$$
f_n(x) = e^{-t_n x }f(x)
$$
Comme $f(x)$ est fini $\lambda-pp$ (voir a)) alors $f_n(x)\rightarrow 0$ pour $n\rightarrow \infty$ $\lambda-pp$. De plus, on a 
$$
|f_n(x)|\leq |f(x)|.
$$
Par application du théorème de convergence dominé, il vient 
$$
\underset{t\rightarrow+\infty}{\lim}L_f(t) = 0.
$$
\end{solution}
\part[1] Donner l'expression de $L_f(t)$ pour $f(x) = e^{-\theta x}$ avec $\theta >0$.
\begin{solution}
$$
L_f(t) = \frac{1}{t+\theta}.
$$
\end{solution}
\part[1] De même pour $f(x) = \sin(x)\mathbb{I}_{[0, 1]}(x)$.
\begin{solution}
$$
L_f(t)  = \Im\int_0^1 e^{-x(t-i)}\text{d}x  =\frac{1-e^{-t}(\cos1 + t\sin1)}{1+t^2} .
$$
\end{solution}
\end{parts}
\end{questions}

\newpage
%-------------------------------TABLE-------------------------------
\hrule
\vspace*{.15in}
\begin{center}
	\large\MakeUppercase{Fonctions Trigonométriques}
\end{center}
\vspace*{.15in}
\hrule
\vspace*{.25in}


\renewcommand\arraystretch{3.5}
\begin{table}[H]
\begin{center}
\begin{tabular}{|c|c|c|}
\hline
Fonction & Ensemble de définition& Dérivée \\
\hline\hline
$\sin x$ &$\mathbb{R}$& $\cos x$ \\
\hline
$\cos x$ &$\mathbb{R}$& $-\sin x$ \\
\hline
$\tan x$ &$\bigcup_{n\in\mathbb{Z}}\left]n\pi-\pi/2, n\pi+\pi/2\right[$ & $1+\tan^2 x$ \\
\hline
$\arccos x$ &$[-1,1]$& $-\frac{1}{\sqrt{1-x^2}}$ \\
\hline
$\arcsin x$ &$[-1,1]$& $\frac{1}{\sqrt{1-x^2}}$\\
\hline
$\arctan x$ &$\mathbb{R}$& $\frac{1}{1+x^2}$ \\
\hline
\end{tabular}
\end{center}
\end{table}%
\underline{Quelques identités}: Pour $a,b\in\mathbb{R}$,\\
$$
\cos(a+b) = \cos(a)\cos(b)-\sin(a)\sin(b)
$$
$$
\sin(a+b) = \sin(a)\cos(b)+\cos(a)\sin(b)
$$
$$
\cos^2(a)+\sin^2(a) = 1
$$
\underline{Quelques développement en série entière}: Pour tout $x\in\mathbb{R}$,\\
$$
\cos(x) = \sum_{n = 0}^\infty\frac{(-1)^nx^{2n}}{(2n)!},
$$
et pour $p\in [0,1)$
$$
\frac{1}{1-p}=\sum_{n=0}^\infty p^n
$$



\end{document}

