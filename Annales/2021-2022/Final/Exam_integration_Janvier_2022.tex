\documentclass[11pt, addpoints, answers]{exam}

\usepackage[utf8]{inputenc}
\usepackage[T1]{fontenc}
\usepackage[margin  = 1in]{geometry}
\usepackage{amsmath, amscd, amssymb, amsthm, verbatim}
\usepackage{mathabx}
\usepackage{setspace}
\usepackage{float}
\usepackage{color}
\usepackage{graphicx}
\usepackage[colorlinks=true]{hyperref}
\usepackage{tikz}
\usetikzlibrary{trees}

\shadedsolutions
\definecolor{SolutionColor}{RGB}{214,240,234}

\newcommand{\bbC}{{\mathbb C}}
\newcommand{\R}{\mathbb{R}}            % real numbers
\newcommand{\bbR}{{\mathbb R}}
\newcommand{\Z}{\mathbb{Z}}            % integers
\newcommand{\bbZ}{{\mathbb Z}}
\newcommand{\bx}{\mathbf x}            % boldface x
\newcommand{\by}{\mathbf y}            % boldface y
\newcommand{\bz}{\mathbf z}            % boldface z
\newcommand{\bn}{\mathbf n}            % boldface n
\newcommand{\br}{\mathbf r}            % boldface r
\newcommand{\bc}{\mathbf c}            % boldface c
\newcommand{\be}{\mathbf e}            % boldface e
\newcommand{\bE}{\mathbb E}            % blackboard E
\newcommand{\bP}{\mathbb P}            % blackboard P

\newcommand{\ve}{\varepsilon}          % varepsilon
\newcommand{\avg}[1]{\left< #1 \right>} % for average
%\renewcommand{\vec}[1]{\mathbf{#1}} % bold vectors
\newcommand{\grad}{\nabla }
\newcommand{\lb}{\langle }
\newcommand{\rb}{\rangle }

\def\Bin{\operatorname{Bin}}
\def\Var{\operatorname{Var}}
\def\Geom{\operatorname{Geom}}
\def\Pois{\operatorname{Pois}}
\def\Exp{\operatorname{Exp}}
\newcommand{\Ber}{\operatorname{Ber}}
\def\Unif{\operatorname{Unif}}
\def\No{\operatorname{N}}
\newcommand{\E}{\mathbb E}            % blackboard E
\def\th{\theta }            % theta shortcut
\def\V{\operatorname{Var}}
\def\Var{\operatorname{Var}}
\def\Cov{\operatorname{Cov}}
\def\Corr{\operatorname{Corr}}
\newcommand{\epsi}{\varepsilon}            % epsilon shortcut

\providecommand{\norm}[1]{\left\lVert#1\right\rVert} %norm
\providecommand{\abs}[1]{\left \lvert#1\right \rvert} %absolute value

\DeclareMathOperator{\lcm}{lcm}
\newcommand{\ds}{\displaystyle}	% displaystyle shortcut

\def\semester{2021-2022 }
\def\course{Théorie de la mesure et intégration}
\def\title{\MakeUppercase{Examen Final}}
\def\name{Pierre-O. Goffard}
%\def\name{Professor Wildman}

\setlength\parindent{0pt}

\cellwidth{.35in} %sets the minimum width of the blank cells to length
\gradetablestretch{2.5}

%\bracketedpoints
%\pointsinmargin
%\pointsinrightmargin

\begin{document}


\runningheader{\course  \vspace*{.25in}}{}{\title \vspace*{.25in}}
%\runningheadrule
\runningfooter{}{Page \thepage\ of \numpages}{}

% \firstpageheader{Name:\enspace\hbox to 2.5in{\hrulefill}\\  \vspace*{2em} Section: (circle one) TR: 3-3:50 \textbar\, TR: 5-5:50 \textbar\,  TR: 6-6:50(Xu) \textbar\,  TR: 6-6:50 }{}{Perm \#: \enspace\hbox to 1.5in{\hrulefill}\\ \vspace*{2em} Score:\enspace\hbox to .6in{\hrulefill} $/$\numpoints}
\extraheadheight{.25in}

\hrulefill

\vspace*{1em}

% Heading
{\center \textsc{\Large\title}\\
	\vspace*{1em}
	\course -- \semester\\
	Pierre-O Goffard\\
}
\vspace*{1em}

\hrulefill

\vspace*{2em}

\noindent {\bf\em Instructions:} On éteint et on range son téléphone.
\begin{itemize}
	\item La calculatrice et les appareils éléctroniques ne sont pas autorisés.
	\item Vous devez justifier vos réponses de manière claire et concise.
	\item Vous devez écrire de la manière la plus lisible possible. Souligner ou encadrer votre réponse finale.
	\item \underline{Document autorisé:} Une feuille manuscrite recto-verso
\end{itemize}

\begin{center}
	\gradetable[h]
\end{center}

\smallskip

\begin{questions}
\question Question de cours indépendantes
\begin{parts}
\part[2] Soient $(\Omega,\mathcal{A},\mu)$ un espace mesuré et $f:\Omega\mapsto \mathbb{R}_+$ une fonction mesurable. Soit l'application 
$$
\nu(A) = \int_A f\text{d}\mu,\text{ }A\in\mathcal{A},
$$
Montrer que $\mu(A)=0$ alors $\nu(A) = 0$, puis que $\nu$ est une mesure positive sur $(\Omega,\mathcal{A})$. 
\begin{solution}
Supposons que $\mu(A) = 0$, et posons 
$$
h = \sum_{n\in\mathbb{N}^\ast}\alpha_n\mathbb{I}_{A_n},
$$
avec $\alpha_n = \sup\{f(\omega)\text{ ; }\omega\in A_n \}$. On a $f\geq h$ et donc 
$$
\nu(A) = \int_A f\text{d}\mu = \sum\alpha_n\mu(A_n\cap A) = 0.
$$
On a 
\begin{itemize}
	\item[(i)] $\nu(\emptyset) = \int_{\emptyset}f\text{d}\mu =0$
	\item[(ii)] Soit $(A_n)_{n\in\mathbb{N}^\ast}\in\mathcal{A}$ disjoints. On a 
	$$
	\nu(\bigcup_{n\in\mathbb{N}^\ast}A_n) = \int_{\bigcup_{n\in\mathbb{N}^\ast}A_n}f\text{d}\mu = \int\sum_{n\in\mathbb{N}^\ast}\mathbb{I}_{A_n}f\text{d}\mu = \sum_{n\in\mathbb{N}^\ast}\int\mathbb{I}_{A_n}f\text{d}\mu 
	$$
	où la dernière étape est justifiée par l'emploi du théorème de convergence monotone.
\end{itemize}

\end{solution}
\part[2] Soit $(\Omega,\mathcal{A},\mu)$ un espace probabilisé. Montrer que
$$
\mathcal{F} = \{A\in\mathcal{A}\text{ ; }\mu(A)\in\{0,1\}\}
$$ 
est une tribu.
\begin{solution}
\begin{itemize}
	\item Comme $\mu(\emptyset)= 0$ alors $\emptyset\in\mathcal{F}$.
	\item Soit $A\in\mathcal{F}$, on a 
	$$
	\mu(A^c) = 1-\mu(A) = \begin{cases}0&\text{ si }\mu(A) = 1\\
	1&\text{ si }\mu(A) = 0
	\end{cases}
	$$
	Donc $A^c\in\mathcal{F}$.
	\item Soit $(A_n)_{n\in\mathbb{N}^\ast}\in\mathcal{F}$ disjoint, alors soit
	\begin{itemize}
		\item $\exists n\in \in\mathbb{N}^\ast$ tel que $\mu(A_n)=1$ et 
		$$
		1=\mu(A_n)\leq \mu\left(\bigcup_n A_n\right)\leq \mu(\Omega) = 1
		$$
		puis $\mu\left(\bigcup_n A_n\right) = 1$
		\item $\mu(A_n)=0$ pour tout $n\in\mathbb{N}^\ast$ et 
		$$
		\mu\left(\bigcup_{n\in\mathbb{N}^\ast} A_n\right)\leq \sum_{n\in\mathbb{N}^\ast}\mu(A_n) = 0
		$$	
	\end{itemize}
\end{itemize}
\end{solution}
\part[1] Soit $f:\R^2\mapsto\R_+$ une application mesurable. On considère 
$$
I = \int_{\R^2}f(u,v)\text{d}\lambda(u,v)\text{ et }J =\int_{\R^2}f(3x,x-2y)\text{d}\lambda(x,y).
$$ 
Exprimer $I$ en fonction de $J$ ou inversement.
\begin{solution}
Application de la formule de changement de variable. On trouve que
$$
I = 6\cdot J
$$
\end{solution}
\end{parts}
\question On considère l'espace mesurable $(\R,\mathcal{B}(\mathbb{R}), \lambda)$, où $\lambda$ désigne la mesure de Lebesgue. Soient 
$$
A_n = [n-2^{n+1}, n-3^{-n-1}],\text{ }n\geq0\text{ et }A = \bigcap_{n = 0}^\infty A_n.
$$
\begin{parts}
\part[2]  Montrer qu'une intersection dénombrable d'évènement de $\mathcal{B}(\R)$ appartient à $\mathcal{B}(\R)$. En déduire que $A\in\mathcal{B}(\R)$.
\begin{solution}
Voir le cours
\end{solution}
\part[1] Calculer $\lambda(A)$ 
\begin{solution}
$(A_n)_{n\geq0}$ est une suite croissante, on a donc $A = A_0$ puis $\lambda(A) = \lambda(A_0) =-1/3-(-2) = 5/3$
\end{solution}
\end{parts}
\question Soit la fonction $t\mapsto F(t)$ définie par 
$$
F(t) = \int_{[0, \infty)}e^{-x^2}\cos(2tx)\text{d}\lambda(x),\text{ pour }t\in\R.
$$
\begin{parts}
\part[1] Montrer que $t\mapsto F(t)$ est continue.
\begin{solution}
On pose 
$$
(x,t)\mapsto f(x,t) = e^{-x^2}\cos(2tx). 
$$
On note que $t\mapsto f(x,t)$ est continue pour tout $x\in [0,+\infty)$, et de plus 
$$
|f(x,t)|<e^{-x^2},\text{ où }x\mapsto e^{-x^2}\text{ est intégrable.} 
$$ 
On en déduit par théorème que $t\mapsto F(t)$ est continue
\end{solution}
\part[1] Montrer que $t\mapsto F(t)$ est de classe $C^1$ (dérivable et de dérivée continue).
\begin{solution}
On note que $t\mapsto f(x,t)$ est dérivable pour tout $x\in [0,+\infty)$, et de plus 
$$
|\frac{\partial}{\partial t}f(x,t)| = 2x e^{-x^2}\sin(2tx)  <2x e^{-x^2},\text{ où }x\mapsto 2x e^{-x^2}e^{-x^2}\text{ est intégrable.} 
$$
On en déduit par théorème que $t\mapsto F(t)$ est de classe $C^1$
\end{solution}
\part[2] Montrer que $t\mapsto F(t)$ vérifie
$$
F'(t)+2tF(t) =0.
$$
\begin{solution}
D'après la question précédente, on sait que 
$$
F'(t) = - \int_0^\infty 2x e^{-x^2}\sin(2tx)\text{d}x.
$$
Via une intégration par partie, il vient 
$$
F'(t) =-2t F(t)
$$
\end{solution}
\part[2] Donner une expression simple pour $F(t)$.\\
\underline{Indication:}\\
Que vaut $F(0)$? On pourra calculer 
$$
\int_{\R_{+}^2} e^{-(x^2+y^2)}\text{d}\lambda(x,y)
$$
via un changement de variable classique pour s'en sortir.
\begin{solution}
D'après la question précédente 
$$
F(t) = F(0)e^{-t^2}
$$
puis $F(0) = \int_0^\infty e^{-x^2}\text{d}x =\sqrt{\pi}/2 $ et donc 
$$
F(t) = \frac{\sqrt{\pi}}{2}e^{-t^2}.
$$
\end{solution}
\end{parts}
\question Inversion series-intégrales. Les réponses sont à justifier soigneusement en utilisant les théorèmes du cours!
\begin{parts}
\part[1] Montrer que pour $a,b>0$, on a 
$$
\int_0^{+\infty}\frac{te^{-at}}{1-e^{-bt}}\text{d}\lambda(t) = \sum_{n=0}^{+\infty}\frac{1}{(a+bn)^2}.
$$
\begin{solution}
On a
$$
\int_0^{+\infty}\frac{te^{-at}}{1-e^{-bt}}\text{d}\lambda(t) = \int_0^{+\infty}\sum_{n=0}^{+\infty}te^{-(a+bn)t}\text{d}\lambda(t).
$$ 
On pose
$$
f_n(t) = te^{-(a+bn)t},\text{ pour }n\geq0.
$$
Il s'agit d'une suite de fonctions positives, on peut donc intervertir série et intégrale par convergence monotone. Les intégrales de Lebesgue et de Riemann coincident car les fonctions $t\mapsto \frac{te^{-at}}{1-e^{-bt}}$ et $t\mapsto f_n(t)$ sont intégrables. Lorsque $t\rightarrow0$, on a $1-e^{-bt}\sim bt$ et lorsque $t\rightarrow+\infty$ la fonction est équivalente à $te^{-at}$ qui est intégrable. Il vient alors après une integration par partie
$$
\int_0^{+\infty}te^{-(a+bn)t}\text{d}t = \frac{1}{(a+bn)^2}.
$$
\end{solution}
\part[2] Montrer que 
$$
\int_{0}^{+\infty}\cos(\sqrt{x})e^{-x}\text{d}\lambda(x)=\sum_{n=0}^{+\infty}(-1)^n\frac{n!}{(2n)!}
$$
\begin{solution}
On utilise le développement en série entière suivant 
$$
\cos(\sqrt{x}) = \sum_{n=0}^{+\infty}(-1)^n\frac{x^n}{(2n)!}
$$
pour obtenir
$$
\int_{0}^{+\infty}\sum_{n=0}^{+\infty}(-1)^n\frac{x^n}{(2n)!}e^{-x}\text{d}\lambda(x).
$$
L'interversion série intégrale est justifié soit par l'emploi du critère sur les séries de fonctions alternées ou en observant que 
$$
\sum_{n=0}^{+\infty}\int_{0}^{+\infty}|f_n(x)|\text{d}\lambda(x)<\infty
$$
via le critère de D'alembert avec $f_n(x) = (-1)^n\frac{x^n}{(2n)!}e^{-x}$. On note que $\int_{0}^{+\infty}t^ne^{-x}\text{d}x=n!$ (facile a montrer par récurrence) pour conclure. 
\end{solution}
\end{parts}
\question Soit $(\Omega, \mathcal{A}, \mathbb{P})$ un espace probabilisé, et $f:\Omega\mapsto \mathbb{R}$ une application intégrable. 
\begin{parts}
\part[1] Montrer que 
\begin{equation}\label{eq:Jensen}
\left(\int_\Omega f\text{d}\mu\right)^2\leq \int_{\Omega}f^2\text{d}\mu
\end{equation}
\begin{solution}
On applique l'inégalité de Jensen avec la fonction $\varphi:x\mapsto x^2$
\end{solution}
\part[2] Montrer que l'inégalité \eqref{eq:Jensen} est une égalité si et seulement si $f$ est une fonction constante. 
\begin{solution}
$\Leftarrow$ Supposons que $f=a$ $\mu-$pp, alors on a immédiatement 
$$
\left(\int_\Omega f\text{d}\mu\right)^2 = \int f^2\text{d}\mu =a^2
$$ 
$\Rightarrow$ Supposons que $\left(\int_\Omega f\text{d}\mu\right)^2 = \int f^2\text{d}\mu$
On pose $a=\int f\text{d}\mu$ et on remarque que 
$$
\int(f-a)^2\text{d}\mu = \int f^2\text{d}\mu - a^2 = 0
$$
Comme $(f-a)^2$ est une application positive alors $(f-a)^2 = 0$ $\mu$-pp puis $f=a$ $\mu$-pp. 
\end{solution}
\end{parts}
\end{questions}

\newpage
%-------------------------------TABLE-------------------------------
\hrule
\vspace*{.15in}
\begin{center}
	\large\MakeUppercase{Fonctions Trigonométriques}
\end{center}
\vspace*{.15in}
\hrule
\vspace*{.25in}


\renewcommand\arraystretch{3.5}
\begin{table}[H]
\begin{center}
\begin{tabular}{|c|c|c|}
\hline
Fonction & Ensemble de définition& Dérivée \\
\hline\hline
$\sin x$ &$\mathbb{R}$& $\cos x$ \\
\hline
$\cos x$ &$\mathbb{R}$& $-\sin x$ \\
\hline
$\tan x$ &$\bigcup_{n\in\mathbb{Z}}\left]n\pi-\pi/2, n\pi+\pi/2\right[$ & $1+\tan^2 x$ \\
\hline
$\arccos x$ &$[-1,1]$& $-\frac{1}{\sqrt{1-x^2}}$ \\
\hline
$\arcsin x$ &$[-1,1]$& $\frac{1}{\sqrt{1-x^2}}$\\
\hline
$\arctan x$ &$\mathbb{R}$& $\frac{1}{1+x^2}$ \\
\hline
\end{tabular}
\end{center}
\end{table}%
\underline{Quelques identités}: Pour $a,b\in\mathbb{R}$,\\
$$
\cos(a+b) = \cos(a)\cos(b)-\sin(a)\sin(b)
$$
$$
\sin(a+b) = \sin(a)\cos(b)+\cos(a)\sin(b)
$$
$$
\cos^2(a)+\sin^2(a) = 1
$$
\underline{Quelques développement en série entière}: Pour tout $x\in\mathbb{R}$,\\
$$
\cos(x) = \sum_{n = 0}^\infty\frac{(-1)^nx^{2n}}{(2n)!},
$$
et pour $p\in [0,1)$
$$
\frac{1}{1-p}=\sum_{n=0}^\infty p^n
$$



\end{document}

