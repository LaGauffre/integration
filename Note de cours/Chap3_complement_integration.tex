\documentclass[8pt,notheorems]{beamer}
\usetheme{Copenhagen}
\usepackage[utf8]{inputenc}
\usepackage[T1]{fontenc}
\usepackage{beamerthemesplit}
\usepackage{graphicx}
% \usepackage{tkz-graph}
\usepackage{color}
\usepackage{listings}
\usepackage{relsize}
\usepackage{amsmath,amsfonts,amsthm,t1enc}
\usepackage{fourier}
\usepackage{listings}
\usepackage{amsmath}
\usepackage{tikz}
\usetikzlibrary{automata,arrows,positioning,calc}
\setbeamertemplate{footline}{\hfill \insertframenumber/\inserttotalframenumber}
% \setbeamertemplate{headline}{}
\def \si {\sigma}
\def \la {\lambda}
\def \al {\alpha}
% \def\e*{\end{eqnarray*}}
\def \di{\displaystyle}

\def \E{\mathbb E}
\def \N{\mathbb N}
\def \R{\mathbb{R}}
\def \A{\mathcal{A}}
\def \Om{\Omega}
\def \om{\omega}
\def \Bor{\mathcal{B}(\mathbb{R})}
\def \limsup{\underset{n\rightarrow+\infty}{\overline{\lim}}}
\def \liminf{\underset{n\rightarrow+\infty}{\underline{\lim}}}

\def \NZ{\mathbb{N}_0}
\def \I{\mathbb I}
\def \w{\widehat}
\def \P {\mathbb P}
\def \V{\mathbb V}



\newcommand{\CL}{\mathbb{C}}
\newcommand{\RL}{\mathbb{R}}
\newcommand{\nat}{{\mathbb N}}
\newcommand{\Laplace}{\mathscr{L}}
\newcommand{\e}{\mathrm{e}}
\newcommand{\ve}{\bm{\mathrm{e}}} % vector e

\renewcommand{\L}{\mathcal{L}} % e.g. L^2 loss.

\newcommand{\ih}{\mathrm{i}}
\newcommand{\oh}{{\mathrm{o}}}
\newcommand{\Oh}{{\mathcal{O}}}
\newcommand{\Exp}{\mathbb{E}}

\newcommand{\Norm}{\mathcal{N}}
\newcommand{\LN}{\mathcal{LN}}
\newcommand{\SLN}{\mathcal{SLN}}

\renewcommand{\Pr}{\mathbb{P}}
\newcommand{\Ind}{\mathbb I}
\newcommand\bfsigma{\bm{\sigma}}
\newcommand\bfSigma{\bm{\Sigma}}
\newcommand\bfLambda{\bm{\Lambda}}
\newcommand{\stimes}{{\times}}

\setbeamertemplate{theorem}[ams style]
\setbeamertemplate{theorems}[numbered]
%\makeatletter
%\def\th@mystyle{%
%    \normalfont % body font
%    \setbeamercolor{block title example}{bg=orange,fg=white}
%    \setbeamercolor{block body example}{bg=blue!20,fg=black}
%    \def\inserttheoremblockenv{block}
%  }
%\makeatother
%\theoremstyle{mystyle}

\makeatletter
    \ifbeamer@countsect
      \newtheorem{theorem}{\translate{Theorem}}[section]
    \else
      \newtheorem{theorem}{\translate{Theoreme}}
    \fi
    \newtheorem{corollary}{\translate{Corollaire}}
    \newtheorem{prop}{\translate{Proposition}}
    \newtheorem{lemma}{\translate{Lemme}}
    \newtheorem{problem}{\translate{Probleme}}
    \newtheorem{solution}{\translate{Solution}}

    \theoremstyle{definition}
    \newtheorem{definition}{\translate{Definition}}
    \newtheorem{definitions}{\translate{Definitions}}

    \theoremstyle{example}
    \newtheorem{example}{\translate{Exemple}}
    \newtheorem{remark}{\translate{Remarque}}
    \newtheorem{examples}{\translate{Examples}}

\makeatletter
\def\th@mystyle{%
    \normalfont % body font
    \setbeamercolor{block title example}{bg=orange,fg=white}
    \setbeamercolor{block body example}{bg=orange!20,fg=black}
    \def\inserttheoremblockenv{exampleblock}
  }
\makeatother
\theoremstyle{mystyle}
\newtheorem{fact}{Fact}





    % Compatibility
    \newtheorem{Beispiel}{Beispiel}
    \newtheorem{Beispiele}{Beispiele}
    \theoremstyle{plain}
    \newtheorem{Loesung}{L\"osung}
    \newtheorem{Satz}{Satz}
    \newtheorem{Folgerung}{Folgerung}
    \newtheorem{Fakt}{Fakt}
    \newenvironment{Beweis}{\begin{proof}[Beweis.]}{\end{proof}}
    \newenvironment{Lemma}{\begin{lemma}}{\end{lemma}}
    \newenvironment{Proof}{\begin{proof}}{\end{proof}}
    \newenvironment{Theorem}{\begin{theorem}}{\end{theorem}}
    \newenvironment{Problem}{\begin{problem}}{\end{problem}}
    \newenvironment{Corollary}{\begin{corollary}}{\end{corollary}}
    \newenvironment{Example}{\begin{example}}{\end{example}}
    \newenvironment{Examples}{\begin{examples}}{\end{examples}}
    \newenvironment{Definition}{\begin{definition}}{\end{definition}}
\makeatother







% ============================================================
% Title
% ============================================================

\title[]{Intégration L3 Actuariat}
\subtitle{Chapitre III: Complément d'Intégration}
\author{Pierre-Olivier Goffard}
\institute{
	   Université de Lyon 1\\
	ISFA\\
	   \texttt{pierre-olivier.goffard@univ-lyon1.fr}
	  }
\date{
ISFA\\
\today}
\lstset{language=SAS}
\begin{document}

\frame{\titlepage}


% ============================================================
\section{Mesures définie par des densités}
\begin{frame}[allowframebreaks]
Nous allons étudier dans ce chapitre diverses propriétés qui permettent d'effectuer des calculs en théorie de l'intégration. Toutes ces notions admettent des applications en calcul des probabilité. \\
\underline{I. Mesures définie par des densités}\\
\begin{prop}[Mesure à densité]
Soit $(\Om,\mathcal{A},\mu)$ un espace mesuré, et $f:\Om\mapsto\R_{+}$ une application mesurable positive. L'application $\nu:\mathcal{A}\mapsto\overline{\R}_{+}$ définie par
$$
\nu(A)=\int_A f\text{d}\mu,\text{ }A\in \mathcal{A},
$$
est une mesure sur $(\Omega,\mathcal{A})$ appelé mesure de densité $f$ par rapport à $\mu$
\end{prop}
\underline{preuve:}\\
\begin{enumerate}
    \item $\nu(\emptyset) = \int_\emptyset f\text{d}\mu$
    \item Soit $(A_n)_{n\in \N^{\ast}}$, une suite d'éléments disjoints de $\mathcal{A}$. On note que 
    \begin{eqnarray*}
    \nu\left(\bigcup_{n\in \N^{\ast}} A_n\right) &=& \int_{\bigcup_{n\in \N^{\ast}} A_n}f\text{d}\mu\\
    &=&\int\mathbb{I}_{\bigcup_{n\in \N^{\ast}} A_n}f\text{d}\mu\\
    &=&\int\sum_{n\in \N^{\ast}}\mathbb{I}_{ A_n}f\text{d}\mu\\
    &=&\int\underset{n\rightarrow\infty}{\lim} \sum_{k=1}^{n}\mathbb{I}_{ A_k}f\text{d}\mu\\
    &=&\underset{n\rightarrow\infty}{\lim}\int \sum_{k=1}^{n}\mathbb{I}_{ A_k}f\text{d}\mu\\
    &=&\underset{n\rightarrow\infty}{\lim}\sum_{k=1}^{n}\int \mathbb{I}_{ A_k}f\text{d}\mu\\
    &=&\sum_{k\in\N^{\ast}}\int \mathbb{I}_{ A_k}f\text{d}\mu\\
    &=&\sum_{k\in\N^{\ast}}\nu(A_k).\\
    \end{eqnarray*}
\end{enumerate}
\end{frame}
\begin{frame}[allowframebreaks]
\begin{remark}
\begin{itemize}
    \item Si $f$ est intégrable alors $\nu$ est finie, en effet:
    $$
    \nu(\Omega) = \int_\Omega f\text{d}\nu \leq \int_\Omega |f|\text{d}\nu <\infty
    $$
    \item Si $f$ est telle que $\int_\Omega f\text{d}\nu = 1$ alors $\nu$ est une mesure de probabilité de densité $f$.  
\end{itemize}
\end{remark}
De nombreuses mesure de probabilité $(\R,\mathcal{B}_\R)$ sont à densité par rapport à la mesure de Lebesgue et sont associées à la loi d'une variable aléatoire $X$ avec 
$$
\mathbb{P}_X(A) =\mathbb{P}(X\in A)= \int_{A} f_{X}(x)\text{d}\lambda(x)
$$
Une densité de probabilité $f_X:\R\mapsto \R^+$ est essentiellement une fonction positive, continue par morceaux d'intégrale 1!
\begin{remark}
Une variable aléatoire $X$ dont la loi de probabilité $\P_X$ est à densité par rapport à la mesure de Lebesgue est dite continue. On remarque que 
$$
\P_X(\{x\}) =\P(X = x)=\int_{\{x\}}f_X(x)\text{d}\lambda(x) = 0 
$$
Cependant

$$
\P_X([x, x+\text{d}x]) =\P(X \in [x, x+\text{d}x])=\int_{[x, x+\text{d}x]}f_X(x)\text{d}\lambda(x) \approx f(x)\text{d}x 
$$

\end{remark}
\begin{example}[Lois de probabilités classiques]
\begin{enumerate}
    \item $X\sim\text{unif}([a,b])$
    $$
    f(x) = \frac{1}{b-a}\mathbb{I}_{[a,b]}(x),\text{ }a<b.
    $$
    \item $X\sim\mathcal{N}(\mu, \sigma^2)$
    $$
    f(x) = \frac{1}{\sigma\sqrt{2\pi}}\exp\left[-\frac{1}{2\sigma^2}(x-\mu)\right],\text{ }
    $$
    \item $X\sim \text{gamma}(\alpha, \beta)$
    $$
    f(x) = \frac{e^{-x/\beta}x^{\alpha -1}}{\beta^{\alpha}\Gamma(\alpha)}\mathbb{I}_{\R_{+}^\ast}(x),\text{ }\alpha,\beta >0,
    $$ 
    où $\Gamma(\alpha) = \int_0^{+\infty}e^{-x}x^{\alpha-1}\text{d}x$ désigne la fonction gamma.
\end{enumerate}
\end{example}
\end{frame}
\begin{frame}[allowframebreaks]
\begin{theorem}
Soit $(\Omega,\mathcal{A},\mu)$ un espace mesuré, $\nu$ une mesure de densité $f$ par rapport à $\mu$, et $g$ une application mesurable définie sur $(\Omega,\mathcal{A})$. Alors $g$ est $\nu$ intégrable si et seulement si $f.g$ est $\mu$-intégrable, et 
$$
\int g\text{d}\nu = \int g.f\text{d}\mu.
$$
\end{theorem}
\underline{preuve:}\\
Supposons que $g$ soit étagée positive avec 
$$
g = \sum_{i =1}^k\alpha_i\mathbb{I}_{A_i}.
$$
On a 
$$
\int g\text{d}\nu =\sum_{i = 1}^{k}\alpha_i\nu(A_i)=\sum_{i = 1}^{k}\alpha_i\int_{A_i}f\text{d}\mu =  \sum_{i = 1}^{k}\alpha_i \int\mathbb{I}_{A_i}.f\text{d}\mu = \int \sum_{i = 1}^{k}\mathbb{I}_{A_i}.f\text{d}\mu = \int g.f\text{d}\mu.
$$
On passe ensuite aux fonction mesurable positive via Beppo-Lévi puis aux fonctions mesurables. 
\end{frame}
\begin{frame}[allowframebreaks]
\begin{definition}[Mesure absolument continue/étrangère]
Soit $\mu,\nu$ deux mesures sur $(\Om,\mathcal{A})$.
\begin{enumerate}
\item $\nu$ est absolument continue par rapport à $\mu$, $\nu<<\mu$, si et seulement si
$$
\forall A\in\mathcal{A},\text{ }\mu(A)=0\text{ }\Rightarrow\text{ }\nu(A)=0
$$
\item $\nu$ et $\mu$ ont étrangères, $\nu\bot\mu$,
$$
\exists A\in \mathcal{A}\text{ tel que }\mu(A)=0\text{ et }\nu(A^{c})=0.
$$
\end{enumerate}
\end{definition}
\begin{example}
Si $\nu$ est à densité $f$ par rapport à $\mu$, alors 
$$
\mu(A) = 0\Rightarrow \nu(A) = \int_A f\text{d}\nu =0, 
$$ 
donc $\nu <<\mu$.
\end{example}
\begin{theorem}[Radon-Nikodym]
Soient $\mu$ une mesure $\sigma$-finie sur un espace mesurable $(\Omega,\mathcal{A})$ et $\nu$ une mesure $\sigma$-finie, absolument continue par rapport à $\mu$ alors il existe une fonction, mesurable, positive, $f$ telle que
$$
\nu(A)=\int_A f\text{d}\mu,\text{ }A\in \mathcal{A}. 
$$
De plus, $f$ est unique à une $\mu$-équivalence près, c'est à dire que si $f$ et $g$ sont toutes deux densités de $\nu$ par rapport à $\mu$ alors 
$$
f=g\text{ }\mu\text{-pp}.
$$
On dira que $f$ est la dérivée de Radon Nikodym de $\nu$ par rapport à $\mu$ et on note 
$$
f = \frac{\text{d}\nu}{\text{d}\mu},\text{ ou }\text{d}\nu =f\text{d}\mu   
$$
de sorte que 
$$
\nu(A) = \int_{A}\frac{\text{d}\nu}{\text{d}\mu}\text{d}\mu\text{ }A\in \mathcal{A}.
$$
\end{theorem}
\underline{preuve:}\\
On va se contenter de montrer l'unicité, qui se limite à montrer que si $f$ et $g$ sont toutes deux des densités de $\nu$ par rapport à $\mu$ alors $f=g$ $\mu$-presque partout.\\
Posons $A=\{\om\in\Om\text{ ; }f(\om)\geq g(\om)\}$, on a
\begin{eqnarray*}
\int |f-g|\text{d}\mu&=&\int_A |f-g|\text{d}\mu+\int_{A^{c}} |f-g|\text{d}\mu=\int_A (f-g)\text{d}\mu-\int_{A^{c}} f-g\text{d}\mu\\
&=&\nu(A)-\nu(A)-\nu(A^{c})+\nu(A^{c})=0.
\end{eqnarray*}
On en déduit que $|f-g|=0$ $\mu$-p.p. puis $f=g$.\\
$\square$\\

\begin{example}
Une variable aléatoire de comptage $N$ est une variable aléatoire réelle qui ne prends que des valeurs entières. Sa loi de probabilité est donnée par 
$$
\P_N(A) = \P(N\in A) = \sum_{n\in \N}\P(N=n)\delta_n(A), \text{ }A\in \mathcal{B}_{\R},
$$
absolument continue par rapport à la mesure de comptage, définie par 
$$
\nu(A) = \sum_{n\in \N}\delta_n(A),\text{ }A\in \mathcal{B}_{\R},
$$
qui compte le nombre d'entier dans $A$. En effet, on observe que pour $A\in \mathcal{B}_{\R}$,
$$
\nu(A) = 0\Rightarrow \P_N(A) = 0.
$$
Donc $\P_X<<\nu$ donc par application du théorème de R-N, il existe une application $p_N$ telle que 
$$
\P_N(A) = \int_{A} p_N(x)\text{d}\nu(x) = \sum_{n\in \N}p_N(n)\delta_n(A),\text{ }A\in \mathcal{B}_{\R}.
$$
On identifie alors la loi de probabilité de $N$ avec la dérivée de R-N de $\P_N$ par rapport à $\nu$. Cela donne 
$$
p_N(n) = \P(N=n) = \P_N(n),\text{ }n\in \N.
$$ 
\end{example}
\end{frame}
\begin{frame}[allowframebreaks]
\begin{prop}[Décomposition de Lebesgue]
Soient $\mu$ et $\nu$ deux mesures $\sigma$-finies sur $(\Omega, \mathcal{A})$. Il existe alors une application mesurable positive $f$, unique à une $\mu$ équivalence près, et une mesure $\gamma$ sur $(\Omega, \mathcal{A})$, unique, étrangère à $\mu$ telles que 
$$
\nu = f.\mu + \gamma,
$$
c'est à dire 
$$
\nu(A) = \int_A f\text{d}\mu + \gamma(A), \text{ }A\in \mathcal{A}.
$$
\end{prop}
\begin{example}
Soit une variable aléatoire défini par 
$$
X = \mathbb{I}_A\times U
$$
où
\begin{itemize}
    \item $A$ désigne l'évènement l'assuré reporte au moins un sinistre dans l'année
    \item $X$ est une variable aléatoire réelle positive égale au montant des indemnisations versées à l'assuré. La loi de cette variable aléatoire est à densité $f_U$ par rapport à la mesure de Lebesgue. 
\end{itemize}
On suppose de plus que les variables aléatoire $\mathbb{I}_A$ et $U$ sont indépendantes. La loi de $X$ est une loi de probabilité mixte au sens ou 
$$
\P_X(B) = (1-\mathbb{P}(A))\delta_0(B) + \mathbb{P}(A)\int_B f_U(u)\text{d}\lambda(u),\text{ }B\in\mathcal{B}_\R.
$$
On remarque que $\delta_0$ et $\lambda$ sont étrangères avec 
$$
\lambda(\{0\}) = 0\text{ et }\delta_0\left[\{0\}^c\right]=0.
$$
\end{example}

\end{frame}
\begin{frame}[allowframebreaks]
\section{Intégration par rapport à une mesure image}
\underline{II. Intégration par rapport à une mesure image}\\
On rappelle que si $f$ est une application mesurable de $(\Om,\mathcal{A},\mu)$ dans $(E,\mathcal{B})$, on note $\mu^{f}$ la mesure sur $\mathcal{B}$ définie par $\mu^{f}(B)=\mu[f^{-1}(B)]$. Le théorème suivant permet l'intégration par rapport à une mesure image.
\begin{theorem}[Théorème de transfert]
Soit $h:E\mapsto\R $ une application mesurable réelle, $h$ est $\mu^f$-intégrable si et seulement si $h\circ f$ est $\mu$-intégrable et 
$$
\int_E h\text{d}\mu^{f}=\int_\Om h\circ f\text{d}\mu.
$$
\end{theorem}
\underline{preuve:}\\

\begin{tikzpicture}[->, -stealth', auto, semithick, node distance=3cm]
\tikzstyle{every state}=[text=black,scale=0.8]
\node[]    (1)               {$(\Omega,\mathcal{A},\mu)$};
\node[]    (2)[right of=1]   {$(E,\mathcal{B},\mu^f)$};
\node[]    (3)[below of=2]   {$(\R,\mathcal{B}_\R)$};

\path
(1) edge[above]     node{$f$}         (2)
(2) edge[right]     node{$h$}           (3)
(1) edge[below left]     node{$h\circ f$}       (3);

\end{tikzpicture}

Si $h = \sum_{i=1}^{k}\alpha_i\mathbb{I}_{A_i}$ est une application mesurable positive, alors $h\circ f = \sum_{i=1}^{k}\alpha_i\mathbb{I}_{A_i}\circ f$. Remarquons que 
$$
(\mathbb{I}_{A_i}\circ f)(\omega) =\mathbb{I}_{A_i}\left[f(\omega)\right] = \begin{cases}
1&\text{ si }f(\omega)\in A_i \Leftrightarrow \omega\in f^{-1}(A_i)\\
0&   \text{ si }f(\omega)\notin A_i \Leftrightarrow \omega\notin f^{-1}(A_i)
\end{cases} 
$$
et donc $\mathbb{I}_{A_i}\circ f = \mathbb{I}_{f^{-1}(A_i)}$. On en déduit que 
$$
\int_{\Omega}h\circ f\text{d}\mu = \sum_{i = 1}^k\alpha_i\int \mathbb{I}_{A_i} \circ f\text{d}\mu = \sum_{i = 1}^k\alpha_i\int \mathbb{I}_{f^{-1}(A_i)} \text{d}\mu = \sum_{i = 1}^k\alpha_i\mu\left[f^{-1}(A_i)\right] = \sum_{i = 1}^k\alpha_i\mu^f(A_i)=\int h\text{d}\mu^f . 
$$

 Pour $h$ mesurable positive, on définit une suite croissante de fonctions $(h_n)_{n\in\N}$ étagées positives convergeant vers $h$. La suite $(h_n\circ f)_{n\in\N}$ une suite croissante, de fonction étagées positives qui convergent vers $h\circ f$. Par application du théorème de Beppo Lévy, il vient
$$
\int h\text{d}\mu^{f}=\underset{n\rightarrow +\infty}{\lim}\int h_{n}\text{d}\mu^{f}=\underset{n\rightarrow +\infty}{\lim}\int h_{n}\circ f\text{d}\mu=\int h\circ f\text{d}\mu.
$$
Pour le cas $h$ mesurable, on observe que
$$
\int |h|\text{d}\mu^{f}=\int |h|\circ f\text{d}\mu=\int |h\circ f|\text{d}\mu,
$$
donc $h$ est $\mu^{f}$-intégrable si et seulement si $h\circ f$ est $\mu$-intégrable et dans ce cas
\begin{eqnarray*}
\int h\text{d}\mu^{f}&=&\int h^{+}-h^{-}\text{d}\mu^{f}=\int h^{+}\text{d}\mu^{f}-\int h^{-}\text{d}\mu^{f}\\
&=&\int h^{+}\circ f\text{d}\mu-\int h^{-}\circ f\text{d}\mu=\int h\circ f\text{d}\mu.
\end{eqnarray*}
$\square$\\
\end{frame}
\begin{frame}[allowframebreaks]
\begin{corollary}
Soit $X$ une variable aléatoire sur un espace probabilisé  $(\Omega,\mathcal{A},\P)$ à valeur dans un espace mesurable $(E, \mathcal{B})$, et de loi $\P_X$. Soit $g$ une application mesurable de $(E,\mathcal{B})$ vers $(\R,\mathcal{B}_\R)$. Alors l'espérance de $g\circ X$ existe si et seulement si $g$ est $\P_X$-intégrable et 
$$
\E\left(g\circ X\right) = \E\left[g(X )\right]= \int_\Omega g\circ X\text{d}\,\P = \int_E g\text{d}\,\P_X.
$$
\end{corollary}
\begin{definition}[Espérance mathématique]
Si $X$ est une variable aléatoire réelle $\P$-intégrable, l'espérance mathématique de $X$ est définie par 
$$
\E(X) = \int_{\Omega} X(\omega)\text{d}\,\P(\omega) = \int X\text{d}\,\P.
$$
\end{definition}
\begin{remark}
L'application du corollaire précédent en prenant $g = \text{Id}$ conduit à écrire 
$$
\E(X) = \int_\R x \text{d}\P_X(x)
$$
\end{remark}
\begin{example}
    \begin{enumerate}
        \item Si $(E,\mathcal{B}) = (\R,\mathcal{B}_{\R})$ et si $P_X = \sum_{n\in \N} p_n\delta_{x_n} $ (X est une v.a. discrète) alors $\E(g\circ X)$ existe si et seulement si $\sum_{n\in \N}p_n|g(x_n)|<\infty$ et 
        $$
        \E[g(X)] = \sum_{n\in \N}p_ng(x_n).
        $$
        \item Si $(E,\mathcal{B}) = (\R,\mathcal{B}_{\R})$ et si $P_X$ est a densité par rapport à la mesure de Lebesgue, $\E(g\circ X)$ existe si et seulement si $g.f$ est Lebesgue-intégrable et 
        $$
        \E[g(X)] = \int g(x)f_X(x)\text{d}\lambda(x).
        $$ 
    \end{enumerate}
\end{example}
\end{frame}
\section{Intégrale par rapport à une mesure produit}
\subsection{Mesure produit}
\begin{frame}[allowframebreaks]

\underline{III. Intégrale par rapport à une mesure produit }\\
\underline{1. Mesure produit}\\
L'objet de cette partie est de répondre à deux questions
\begin{itemize}
    \item Soient $(\Om_1,\mathcal{A}_1, \mu_1)$ et $(\Om_2,\mathcal{A}_2, \mu_2)$ deux espaces mesurés, existe-t-il une mesure $\mu$ sur $(\Omega_1\times \Omega_2,\mathcal{A}_1\otimes \mathcal{A}_2)$ telle que 
    $$
    \mu(A_1\times A_2)=\mu_1(A_1)\times \mu_2(A_2)
    $$
    \item Si une telle mesure existe, et si $f$ est $\mu$-intégrable, peut-on calculer $\int f\text{d}\mu$ en utilisant des intégration par rapport à $\mu_1$ et à $\mu_2$?
\end{itemize}
Soit la tribu produit $\mathcal{A}=\mathcal{A}_1\otimes\mathcal{A}_2$. Pour $A\in\mathcal{A}$, les sections
$$
A_{\om_1}=\{\om_2\in\Om_2\text{ ; }(\om_1,\om_2)\in A\}\text{ et }A_{\om_2}=\{\om_1\in\Om_1\text{ ; }(\om_1,\om_2)\in A\}
$$
sont mesurables (i.e. $A_{\om_1}\in\mathcal{A}_2$ et $A_{\om_2}\in\mathcal{A}_1$).
\begin{theorem}[Mesure produit]
Soient $\mu_1$ et $\mu_2$ deux mesures $\sigma$-finie définies respectivement sur $(\Om_1,\mathcal{A}_1)$ et $(\Om_2,\mathcal{A}_2)$.
\begin{enumerate}
    \item Pour tout $A$ de $\mathcal{A}_1\otimes \mathcal{A}_2$, les applications
    \begin{eqnarray*}
    (\Omega_2,\mathcal{A}_2)\mapsto \left(\overline{\R}^{+},\mathcal{B}_{\overline{\R}^{+}}\right)&\text{ et }&(\Omega_1,\mathcal{A}_1)\mapsto \left(\overline{\R}^{+},\mathcal{B}_{\overline{\R}^{+}}\right) \\
    \omega_2\mapsto \mu_{1}(A_{\omega_2}) && \omega_1\mapsto \mu_{2}(A_{\omega_1}).
    \end{eqnarray*}

 sont mesurables. 
    \item L'unique mesure $\mu$ sur $(\Om_1\times\Om_2,\mathcal{A}_1\otimes\mathcal{A}_2)$ telle que
$$
\mu(A_1\times A_2)=\mu_1(A_1)\mu_2(A_2),\text{ pour tout }A_1\in\mathcal{A}_1\text{ et }A_2\in\mathcal{A}_2,
$$
est l'application définie par 
$$
\mu(A) = \int_{\Omega_2}\mu_1(A_{\omega_2})\text{d}\mu_2(\omega_2) =\int_{\Omega_1}\mu_2(A_{\omega_1})\text{d}\mu_1(\omega_1) 
$$
et notée $\mu = \mu_1\otimes\mu_2$.
\end{enumerate}
 
\end{theorem}
\underline{preuve:} Admis\\

Si $A=A_1\times A_2$ alors
$$
A_{\om_2}=
\begin{cases}
A_1&\text{ si }\om_2\in A_2\\
\emptyset&\text{ si }\om_2\notin A_2\\
\end{cases}
$$
et par suite
$$
\mu_1(A_{\om_2})=\mu_1(A_1)\mathbb{I}_{A_2}(\om_2)=
\begin{cases}
\mu(A_1)&\text{ si }\om_2\in A_2,\\
0&\text{ si }\om_2\notin A_2.\\
\end{cases}
$$
On peut faire les mêmes remarques pour $A_{\om_1}$ et on en déduit que
\begin{eqnarray*}
\mu(A)&=&\int_{\Om_2}\mu_1(A_{\om_2})\text{d}\mu_2=\int_{\Om_1}\mu_2(A_{\om_1})\text{d}\mu_1\\
&=&\int_{\Om_1}\int_{\Om_2}\mathbb{I}_{A}(\om_1,\om_2)\text{d}\mu_2\text{d}\mu_1=\int_{\Om_2}\int_{\Om_1}\mathbb{I}_{A}(\om_1,\om_2)\text{d}\mu_1\text{d}\mu_2
\end{eqnarray*}
On peut donc inter-changer l'ordre d'intégration pour les fonctions indicatrices, l'objet des théorèmes suivant est de changer l'ordre d'intégration pour des fonctions mesurables.\\
\underline{2. Théorème de Fubini et Tonelli}
\end{frame}
\subsection{Théorème de Fubini et Tonelli}
\begin{frame}[allowframebreaks]
\begin{theorem}[Tonelli]
Soit $f:\Om_1\times\Om_2\mapsto\overline{\R}^{+}$ mesurable.
\begin{itemize}
\item $\om_2\mapsto f(\om_1,\om_2)$ est mesurable pour tout $\om_1\in\Om_1$ et la fonction
$$
\om_1\mapsto \int_{\Om_2}f(\om_1,\om_2)\text{d}\mu_2(\om_2)
$$
est mesurable et positive.
\item $\om_1\mapsto f(\om_1,\om_2)$ est mesurable pour tout $\om_2\in\Om_2$ et la fonction
$$
\om_2\mapsto \int_{\Om_1}f(\om_1,\om_2)\text{d}\mu_1(\om_1)
$$
est mesurable et positive.
\end{itemize}
Enfin, on a les égalités
\begin{eqnarray*}
\int f\text{d}(\mu_1\otimes\mu_2)&=&\int_{\Om_2}\left(\int_{\Om_1}f(\om_1,\om_2)\text{d}\mu_1(\om_1)\right)\text{d}\mu_2(\om_2)\\
&=&\int_{\Om_1}\left(\int_{\Om_2}f(\om_1,\om_2)\text{d}\mu_2(\om_2)\right)\text{d}\mu_1(\om_1).
\end{eqnarray*}
\end{theorem}
\begin{theorem}[Fubini]
Soit $f:\Om_1\times\Om_2\mapsto \R$ mesurable. Si
$$
\int |f|\text{d}(\mu_1\otimes\mu_2)<\infty
,$$
alors
\begin{eqnarray*}
\int f\text{d}(\mu_1\otimes\mu_2)&=&\int_{\Om_2}\left(\int_{\Om_1}f(\om_1,\om_2)\text{d}\mu_1(\om_1)\right)\text{d}\mu_2(\om_2)\\
&=&\int_{\Om_1}\left(\int_{\Om_2}f(\om_1,\om_2)\text{d}\mu_2(\om_2)\right)\text{d}\mu_1(\om_1).
\end{eqnarray*}
\end{theorem}
\end{frame}
\begin{frame}[allowframebreaks]
\begin{example}
Nous cherchons à évaluer l'intégrale 
$$
\int_{0}^{+\infty}\frac{e^{-y}\sin^2(y)}{y}\text{d}y
$$
\end{example}
Notons que $g:y\mapsto \frac{e^{-y}\sin^2(y)}{y}$ est continue sur $]0,\infty[$ et donc localement intégrable. 
\begin{itemize}
    \item Lorsque $y\rightarrow 0$ alors
    $$
    \frac{e^{-y}\sin^2(y)}{y}=e^{-y}y\frac{\sin^2(y)}{y^2}\rightarrow 0
    $$
    et donc $g$ est prolongeable par continuité en $0$.
    \item Lorsque $y\rightarrow\infty$, on a 
    $$
    g(y) = o(e^{-y})
    $$
    donc $g$ est intégrable au voisinage de $\infty$.
\end{itemize}
La fonction $g$ est Riemann intégrable sur $]0,\infty[$, elle donc
Lebesgue intégrable et les deux intégrables coincident. 
On définit $f:(x,y)\mapsto e^{-y}\sin(2xy)$ et on note que 
$$
|f(x,y)|<e^{-y},\text{ pour tout }(x,y)\in [0,1]\times \R^{+}
$$
et donc que $f$ est $\lambda_2$ intégrable sur $[0,1]\times \R^{+}$ On remarque que $x\mapsto e^{-y}\sin(2xy)$ est continue sur $[0,1]$ donc Riemann intégrable puis Lebesgue intégrable et 
$$
\int_{[0,1]}e^{-y}\sin(2xy)\text{d}\lambda(x) =  e^{-y}\frac{sin^2 y}{y}. 
$$
Par Fubini, on a 
\begin{eqnarray*}
\int_{[0,1]\times \R^+} f(x,y)\text{d}\lambda_2(x,y) &=&\int_{[0,1]}\int_{\R^+} f(x,y)\text{d}\lambda(y)\text{d}\lambda(x)\\
&=&\int_{\R^+} \int_{[0,1]}f(x,y)\text{d}\lambda(x) \text{d}\lambda(y) 
\end{eqnarray*}
De plus 
$$
\int_{\left]0,+\infty\right[}e^{-y}\sin(2xy)\text{d}\lambda(y) = \int_{0}^{+\infty}e^{-y}\sin(2xy)\text{d}\lambda(y) = \frac{2x}{1+4x^2}
$$
après deux intégrations par parties. On a également 
$$
\int_{[0,1]}\frac{2x}{4x^2+1}\text{d}\lambda(x) =\int_{0}^1\frac{2x}{4x^2+1}\text{d}x = \ln(5)/ 4 
$$ 
On en déduit que 
$$
\int_{0}^{+\infty}\frac{e^{-y}\sin^2(y)}{y}\text{d}y = \ln(5)/ 4 .
$$
\end{frame}
\section{Changement de variables}
\subsection{Fonction de répartition}
\begin{frame}[allowframebreaks]
\underline{IV. Changement de variables}\\
\underline{1. Fonction de répartition}\\
\begin{definition}[Fonction de répartition]
La fonction de répartition $F$ d'une mesure $\nu$ définie sur $(\R,\mathcal{B}_\R)$ est définie par 
$$
F(x) = \nu(\left]-\infty,x\right]),\text{ }x\in \R. 
$$ 
\end{definition}
Le résultat suivant établit un lien entre l'absolue continuité d'une mesure $\nu$ sur $(\R,\mathcal{B}_\R)$ par rapport à la mesure de Lebesgue et l'absolue continuité de la fonction de répartition $F$ sur $\left]a,b\right[$ définie par 
$$
\forall \epsilon>0,\text{ }\exists \eta >0\text{ tel que }\sum_{n\in\N^\ast }(b_n-a_n)<\eta\Rightarrow \sum_{n\in\N^\ast }[F(b_n)-F(a_n)]<\epsilon,
$$
avec $a\leq a_n<b_n\leq b,\text{ }\forall n\in\N^\ast$.
\begin{theorem}
Si $\nu$ est une mesure finie $(\R,\mathcal{B}_\R)$, alors sa fonction de répartition $F$ est absolument continue si et seulement si $\nu$ est absolument continue par rapport à la mesure de Lebesgue sur $R$. 
\end{theorem}
\begin{theorem}
Soit $F$ une application absolument continue. Sa dérivé $F'$ est $\lambda$-presque partout définie et intégrable avec
$$
F(x) = \int_{\left]-\infty,x\right]}F'(t)\text{d}\lambda(t).
$$
\end{theorem}
$F$ est donc la fonction de répartition d'une mesure $\nu$ absolument continue par rapport à $\lambda$ dont la densité de Radon-Nykodim est $F'$.
\end{frame}
\subsection{Théorème de changement de variables}
\begin{frame}[allowframebreaks]
\underline{2. Théorème de changement de variables}\\
Soit $\Phi:\left]a,b\right[\mapsto\left]\phi(a),\phi(b)\right[$ strictement croissante et de dérivée $\Phi'$ continue. Soit $\mu$ la mesure sur $(\left]a,b\right[,\mathcal{B}_{\left]a,b\right[})$, définie comme la mesure image de la mesure de Lebesgue sur $(\left]\phi(a),\phi(b)\right[,\mathcal{B}_{\left]\phi(a),\phi(b)\right[})$ par la fonction $\phi^{-1}$. Soit $f:\R\mapsto\R$ mesurable et Lebesgue intégrable sur $\left]\phi(a),\phi(b)\right[$. En résumé, 


\begin{tikzpicture}[->, -stealth', auto, semithick, node distance=3cm]
\tikzstyle{every state}=[text=black,scale=0.8]
\node[]    (1)               {$(\left]a,b\right[,\mathcal{B}_{\left]a,b\right[},\mu)$};
\node[]    (2)[right of=1]   {$(\left]\phi(a),\phi(b)\right[,\mathcal{B}_{\left]\phi(a),\phi(b)\right[},\lambda)$};
\node[]    (3)[below of=2]   {$(\R,\mathcal{B}_\R)$};

\path
(1) edge[above]     node{$\phi$}         (2)
(2) edge[right]     node{$f$}           (3)
(1) edge[below left]     node{$f\circ \phi$}       (3);

\end{tikzpicture}

On note que $\mu(\left]a,x\right[) = \lambda(\left]\Phi(a),\Phi(x)\right[) = \Phi(x)-\Phi(a)$. Cela implique que la fonction de répartition de $\mu$ est absolument continue et donc que $\mu$ est absolument continue par rapport à Lebesgue, sa densité est la dérivée de $\Phi(x)-\Phi(a)$ soit $\Phi'(x)$. On en déduit par le théorème de transfert que 
$$
\int_{\left]\phi(a),\Phi(b)\right[}f\text{d}\lambda=\int_{\left]a,b\right[}f\circ\Phi\,\text{d}\mu =\int_{\left]a,b\right[}f(\Phi(x))\Phi'(x)\text{d}\lambda(x).
$$

Soient $U$ et $V$ deux ouverts de $\R^{n}$ et $\phi:U\mapsto V$ une bijection, dont les dérivées partielle sont continues. On note
$$
\phi(x)=(\phi_1(x),\ldots, \phi_n(x)) := (y_1,\ldots, y_n) =y , \text{ pour }x=(x_1,\ldots, x_n).
$$
La matrice jacobienne de $\phi$ est définie par
$$
\frac{D\phi}{Dx}(x)=\left(
\begin{array}{ccc}
\frac{\partial \phi_1}{\partial x_1}(x)&\ldots&\frac{\partial \phi_1}{\partial x_d}(x)\\
\vdots&\ddots&\vdots\\
\frac{\partial \phi_d}{\partial x_1}(x)&\ldots&\frac{\partial \phi_d}{\partial x_d}(x)
\end{array}
\right),\text{ pour }x\in U,
$$
son déterminant $\text{det}\,\left(\frac{D\phi}{Dx}(x)\right)$ est appelé jacobien.
\begin{theorem}[Formule de changement de variable]
Soit $f:\R^n\mapsto\R^n$ une application Lebesgue intégrable sur $V$, alors 
$$
\int_{V}f(y)\text{d}\lambda_n(y)=\int_{U}f[\Phi(x)]\left|\text{det}\,\left(\frac{D\phi}{Dx}(x)\right)\right|\,\text{d}\lambda_n(x).
$$
A noter que la formule fait intervenir la valeur absolue du Jacobien. 
\end{theorem}

\begin{example}[Intégrale de Gauss]
On souhaite calculer 
$$
I = \int_{0}^{+\infty}e^{-ax^2}\text{d}x.
$$

\end{example}
Comme $f:(x,y)\mapsto \exp\left(-a(x^{2}+y^{2})\right)$ est Riemann intégrable sur $\R_+^2$ alors elle est Lebesgue intégrable, par Fubini on peut écrire 
$$
I^2 = \int_{\R_+}\int_{\R_+} \exp\left(-a(x^{2}+y^{2})\right)\text{d}x\text{d}y =\int_{\R_+^2} \exp\left(-a(x^{2}+y^{2})\right)\text{d}\lambda^2(x,y).
$$
L'application 
$$
\phi:(r,\theta)\mapsto( r\cos\theta,r\sin\theta) = (x,y).
$$
est une bijection de $\R_+\times \left]0,\pi/2\right[$ dans $\R_+\times \R_+$  qui possède des dérivée partielle. Le Jacobien de $\Phi$ est donné par 
$$
\left|\frac{D\phi}{D(r,\theta)}\right|=\left|\begin{array}{cc}
\cos \theta & -r\sin\theta\\
\sin\theta &r\cos\theta\
\end{array}\right|=r.
$$
Par application de la formule de changement de variable, il vient
$$
I^2 = \int_{\R_+^2} \exp\left(-a(x^{2}+y^{2})\right)\text{d}\lambda^2(x,y) =\int_{\R_+\times \left]0,\pi/2\right[} \exp\left(-a r^2\right)r\text{d}\lambda^2(r,\theta)\Leftrightarrow I =\frac{\sqrt{\pi}}{2\sqrt{a}}.
$$

\end{frame}
\section{Intégrale dépendant d'un paramètre}
\begin{frame}[allowframebreaks]
\underline{V. Intégrale dépendant d'un paramètre}\\

Soit $(\Om,\mathcal{A},\mu)$ un espace mesuré et $f:\Om\times T\mapsto \mathbb{R}$ une fonction de deux variables, où $T$ est un intervalle de $\R$. On suppose que, $\forall t\in T$ $\om\mapsto f(\om,t)$ est mesurable par rapport à $\mathcal{A}$ et intégrable par rapport à $\mu$.
\begin{prop}[Continuité de l'intégrale]
Si $t\mapsto f(\om,t)$ est continue $\mu$-presque partout et qu'il existe une fonction $g$ intégrable par rapport à $\mu$ telle que
$$
|f(\om,t)|\leq g(\om),\text{ }\forall t\in T.
$$
Alors
$$
F(t)=\int_{\Om} f(\om,t)\text{d}\mu(\om)
$$
est continue sur $T$.
\end{prop}
\underline{preuve:} Comme $T$ est un espace métrique, la continuité est caractérisée par le comportement des suites. $F(t)$ est continue si pour toute suite $(t_n)_{n\in\N}$ convergeant vers $t$, $F(t_n)$ converge vers $F(t)$. La suite de fonction $(f(\om,t_n))_{n\in\N}$ converge vers $f(\om, t)$ par continuité de $t\mapsto f(\om,t)$ puis comme $|f(\om,t_n)|\leq g(\om)$ alors $(F(t_n))_{n\in\N}$ converge vers $F(t)$ en vertu du théorème de convergence dominée.
\begin{prop}[Dérivabilité de l'intégrale]
Si $t\mapsto f(\om,t)$ est dérivable par rapport à $t$ $\mu$-presque partout et qu'il existe une fonction $g$ intégrable par rapport à $\mu$ telle que
$$
\left|\frac{\partial f}{\partial t}(\om,t)\right|\leq g(\om),\text{ }\forall t\in T.
$$
Alors
$$
F(t)=\int_{\Om} f(\om,t)\text{d}\mu(\om)
$$
définit une fonction dérivable sur $T$, avec
$$
F'(t)=\int_{\Om} \frac{\partial f}{\partial t}(\om,t)\text{d}\mu(\om).
$$
\end{prop}
\underline{preuve:}\\
Il s'agit de montrer que pour toute suite $(t_n)_{n\in\N}$ convergeant vers $t$, on a
$$
\underset{n\rightarrow+\infty}{\lim}\frac{F(t_n)-F(t)}{t_n-t}=\int_\Om \frac{\partial f}{\partial t}(\om,t)\text{d}\mu(\om).
$$
On pose $f_n(\om)=\frac{f(\om,t_n)-f(\om,t)}{t_n-t}$, qui est une suite de fonctions mesurables convergeant vers $\frac{\partial f}{\partial t}(\om,t)$ qui est donc mesurable. De plus, le théorème des accroissements finis entraine l'inégalité
$$
|f_n(\om)|\leq g(\om).
$$
L'application du théorème de convergence dominée sur la suite $(f_n)_{n\in\N}$ donne
$$
\underset{n\rightarrow+\infty}{\lim}\frac{F(t_n)-F(t)}{t_n-t}=\int_{\Om} f_n(\om)\text{d}\mu
(\om)\rightarrow \int_{\Om} \frac{\partial f}{\partial t}(\om,t)\text{d}\mu(\om)
$$
\begin{example}[La fonction Gamma]
On note $\Gamma$ la fonction définie par
$$
\Gamma(x)=\int_{0}^{+\infty}e^{-t}t^{x-1}\text{d}t,\text{ }x\in\left]0,+\infty\right[
$$
On pose
$$
H_n=\sum_{k=1}^{n}\frac{1}{k},\text{ }n\geq1.
$$
\end{example}
\begin{enumerate}
\item Vérifier que $\Gamma$ est bien définie.
\item Démontrer que la fonction $\Gamma$ est dérivable sur $\left]0,+\infty\right[$, avec
$$
\Gamma'(x)=\int_{0}^{+\infty}e^{-t}t^{x-1}\log(t)\text{d}t.
$$
\item Montrer que la suite
$$
u_n=H_n-\log(n),\text{ }n\geq 1
$$
admet une limite lorsque $n$ tend vers l'infini. On notera $\gamma$ cette limite, auusi appelé constante d'Euler.
\item montrer que
$$
H_n=\int_{0}^{1}\frac{1-(1-v)^{n}}{v}\text{d}v,\text{ }n\geq1.
$$
\item En déduire que, pour tout $n\geq 1$,
$$
\frac{H_{n+1}}{n+1}=-\int_{0}^{1}(1-v)^{n}\log(v)\text{d}v.
$$
\item Etablir que pour tout $t\geq 0$, $1-t\leq e^{-t}$.
\item On pose $I_n=\int_{0}^{n}\left(1-\frac{t}{n}\right)^{n}\log(t)\text{d}t$. Montrer que
$$
\lim I_n =\int_{0}^{+\infty}e^{-t}\log(t)\text{d}t.
$$
\item Montrer que $\gamma=-\Gamma'(1)$.\\
\underline{Hint:} On pourra montrer que $I_n=\frac{n}{n+1}(\log(n)-H_{n+1})$.
\end{enumerate}
\end{frame}
\section{Les espaces Lp}
\begin{frame}[allowframebreaks]
\underline{VI. Espaces $\mathcal{L}^p$}\\
Soit $(\Omega,\mathcal{A},\mu)$ un espace mesuré. On considère des applications mesurables  
\begin{itemize}
\item $f:\Omega\mapsto \mathbb{R}$, auquel cas $|f|=f^+ + f^-$
\item $f:\Omega\mapsto \mathbb{C}$, auquel cas $|f| = \left[\Re(f)^2+\Im(f)^2\right]^{1/2}$
\end{itemize}
Pour tout réel $p\in[1,+\infty[$, on définit
$$
||f||_p = \left[\int|f|^p\text{d}\mu\right]^{1/p}.
$$
\begin{definition}[Espace "L-p"]
$$
\mathcal{L}^p:=\mathcal{L}^p(\Omega,\mathcal{A},\mu) = \left\{f \in\mathcal{M}\text{ ; }||f||_p<\infty\right\}
$$
\end{definition}
La relation d'égalité $\mu$ presque partout définit une relation d'équivalence dans l'espace vectoriel des applications mesurables, avec 
$$
f\sim g\Leftrightarrow f=g\text{ }\mu-pp.
$$
On désigne par $L^p(\Omega,\mathcal{A},\mu) = \mathcal{L}^p(\Omega,\mathcal{A},\mu)/ \sim$ l'ensemble quotient associé. Un élément de $L^p$ est une classe de fonction egales $\mu-$pp qu'on assimilera à un de ses représentants. 
\begin{theorem}
Pour $1\leq p<\infty$, $L^p$ est un espace vectoriel normé par $||.||_p$.
\end{theorem} 
\underline{preuve:}

Montrons que $L^p$ est un espace vectoriel
    \begin{itemize}
        \item[(i)] Soient $f\in L^p$ et $a$ un scalaire, on a 
        $$
        ||a\cdot f||_p\leq |a|\cdot ||f||_p<\infty
        $$
        et donc $a\cdot f\in L^p$
        \item[(ii)] La fonction $x\mapsto |x|^p$ est convexe $\forall x \in\R$ alors 
        $$
        \Big\lvert\frac{x+y}{2}\Big\rvert^p\leq\frac{|x|^p+|y|^p}{2}\Leftrightarrow |x+y|^p\leq 2^{p-1}|x|^p+2^{p-1}|y|^p
        $$
        avec $x,y\in \R$. Pour $x,y\in \mathbb{C}$, on a
        $$
        |x+y|^p\leq (|x|+|y|)^p\leq 2^{p-1}|x|^p+2^{p-1}|y|^p
        $$
        On en déduit que pour $f,g\in L^p$,
        $$
        \int|f+g|^p\text{d}\mu\leq 2^{p-1}\int|f|^p\text{d}\mu^p+2^{p-1}\int|g|^p\text{d}\mu
        $$
        et donc que $f+g\in L^p$. 
    \end{itemize} 
Montrons que $||\cdot||_p$ est une norme, notons que 
$$||f||_p = 0 \Rightarrow f = 0 \text{ }\mu-pp$$
et 
$$
||a\cdot f||_p = |a|\cdot||f||_p,
$$
pour tout scalaire $a$. L'inégalité triangulaire nécessite deux lemmes importants.
\begin{lemma}[Inégalité de Hölder]
Soient $p,q>1$ deux nombres conjugués au sens où $1/p+1/q=1$. Si $f\in L^p$ et $g\in L^q$, alors $f\cdot g \in L^1,$ et 
$$
\int |f\cdot g|\text{d}\mu\leq ||f||_p\cdot ||g||_q.
$$
\end{lemma}
Si $||f||_p = 0$ ou $||f||_q = 0$ alors $f\cdot g = 0$ $\mu$-pp et on a bien 
$$
\int |f\cdot g|\text{d}\mu\leq ||f||_p\cdot ||g||_q.
$$
Sinon, on pose 
$$
a = \frac{|f|}{||f||_p}\text{, et }b = \frac{|g|}{||g||_q}.
$$
La fonction $x\mapsto \log x$ est concave, on a 
$$
\log\left(\frac{|a|^p}{p}+\frac{|b|^q}{q}\right)\geq \frac{\log(|a|)^p}{p}+\frac{\log(|b|)^q}{q} = log(|ab|)
$$
puis finalement $|ab|\leq \frac{|a|^p}{p}+\frac{|b|^q}{q}$ ce qui est équivalent à  
$$
\frac{|f\cdot g|}{||f||_p\cdot ||g||_q}\leq \frac{|f|^p}{p||f||_p^p}+\frac{|g|^q}{q||g||_q^q}.
$$
En intégrant, il vient 
$$
\int |f\cdot g|\text{d}\mu\leq ||f||_p\cdot ||g||_q<\infty.
$$
$\square$
\begin{remark}
Pour $X,Y$ deux variables aléatoires réelles, le cas $p = q=2$ correspond à l'inégalité de Cauchy-Schwarz
$$
\E(|X\cdot Y|)\leq \sqrt{\E(X^2)\cdot\E(Y^2)}.
$$
\end{remark}
\begin{lemma}[Inégalité de Minkowski]
Soient $p\in [1,\infty[$ et $f,g\in L^p$ alors 
$$
||f+g||_p\leq ||f||_p + ||g||_p
$$
\end{lemma}
\underline{preuve:}
Si $p = 1$ alors l'inégalité résulte de $|f+g|\leq |f|+|g|$. Sinon, on pose $q = \frac{p}{p-1}$ de sorte que $1/p+1/q = 1$. Comme $f+g\in L^p$ alors $|f+g|^{p-1}\in L^q$. On utilise l'inégalité de Hölder, 
\begin{eqnarray*}
\int|f+g|^p\text{d}\mu &=& \int|f+g|\cdot |f+g|^{p-1}\text{d}\mu\\  
&\leq&\int(|f|+|g|)\cdot |f+g|^{p-1}\text{d}\mu\\
&\leq&\left(\int|f|^p\text{d}\mu\right)^{1/p} \cdot \left(\int|f+g|^{q(p-1)}\text{d}\mu\right)^{1/q}\\
&+&\left(\int |g|^p\text{d}\mu\right)^{1/p}\cdot\left( \int |f+g|^{q(p-1)}\text{d}\mu\right)^{1/q}.
\end{eqnarray*}
Il vient finalement 
$$
||f+g||_p\leq ||f||_p+||g||_p
$$
$\square$

\end{frame}
\begin{frame}[allowframebreaks]
\begin{definition}
Pour tout fonction $f\in\mathcal{M}$ sur $(\Omega,\mathcal{A}, \mu)$, on définit
$$
||f||_\infty = \inf\{M \in \overline{\R}_+\text{ ; }|f|\leq M\}.
$$ 
L'ensemble $\mathcal{L}^\infty = \{f\in\mathcal{M}\text{ ; }||f||_\infty<\infty\}$ correspond à l'ensemble des fonctions mesurables et bornées $\mu$-pp. 
\end{definition}
\begin{remark}
\begin{enumerate}
    \item On sous-entend que $L^\infty$ est l'espace quotient de $\mathcal{L}^\infty$ pour la relation d'égalité $\mu$-pp.
    \item $p=1$ et $q = \infty$ sont conjugués, l'inégalité de Hölder est valide car si $f\in L^1$et  $g\in L^\infty$ alors 
    $$
    |f\cdot g|\leq |f|\cdot|g||_\infty.
    $$
    \item On $L^r\subset L^p$ avec $p\leq r$ si $\mu$ est fini. (on remarque que $|f|^p\leq |f|r + 1$)  
\end{enumerate}
\end{remark}
\begin{theorem}
$L^\infty$ est un espace vectoriel normé pour la norme $||\cdot||_\infty$
\end{theorem}
\underline{preuve:}\\
$L^\infty$ est un espace vectoriel car 
\begin{enumerate}
    \item[(i)] Pour $a$ un scalaire et $f\in L^\infty$, on a 
    $$
    ||a\cdot f||_\infty = |a|\cdot||f||_\infty
    $$
    et donc $a\cdot f\in L^\infty$.
    \item[(ii)] Comme $|f+g|\leq |f|+|g|$ alors on a directement 
    $||f+g||_\infty\leq ||f||+||g||_\infty$
\end{enumerate}
Pour montrer que $||\cdot||_\infty$ est bien une norme, on observe que 
$$
||f||_\infty = 0 \Rightarrow f =0\text{ }\mu-pp
$$
$\square$
\begin{theorem}[Théorème de convergence dominée dans $L^p$]
Soit $(f_n)_{n\geq 0}\in\mathcal{M}$ qui converge simplement vers $f$. Supposons qu'il existe $g\in L^p$ pour $p\in[1,\infty[$ telle que 
$$
|f_n|\leq g,\text{ }\mu-pp.
$$
alors $f \in L^p$ et $\underset{n\rightarrow\infty}{\lim}||f_n-f||_p  =0$
\end{theorem}
\underline{Preuve:}\\
Comme $|f|\leq g$ alors $f\in L^p$. La suite $(|f_n-f|^p)_{n\geq 0}$ converge vers $0$ et vérifie 
$$
|f_n-f|^p\leq 2^{p}g.
$$
Par application du théorème de convergence dominée, il vient 
$$
\underset{n\rightarrow \infty}{\lim}||f_n - f||_p = 0.
$$
\begin{remark}
\begin{enumerate}
    \item La convergence $\underset{n\rightarrow \infty}{\lim}||f_n||_p=||f||_p $ est la convergence en moyenne d'ordre $p$.
    \item L'espace $L^2$ est un espace de Hilbert muni du produit scalaire 
    $$
    <f,g> = \int f\cdot g\text{d}\mu,\text{ pour }f,g\in L^2.
    $$
    ou
    $$
    <f,g> = \int f\cdot \overline{g}\text{d}\mu,
    $$
    si $f$ et $g$ sont à valeurs complexe.
\end{enumerate}
\end{remark}
\end{frame}
\begin{frame}[allowframebreaks]

\begin{theorem}[Inégalité de Jensen]
Supposons que $\mu$ est une mesure de probabilité et soit $\varphi:\R\mapsto \R_+$ une fonction convexe. Alors pour $f\in\mathcal{L}^1(\Omega,\mathcal{A},\mu)$,
$$
\int\varphi\circ f\text{d}\mu \geq \varphi\left(\int f\text{d}\mu\right).
$$
\end{theorem}
\underline{preuve:}\\
Soit 
$\mathcal{E}_{\varphi} = \{(a,b)\in\mathbb{R}^2\text{ : }\forall x\in\R,\varphi(x)\geq ax+b\}$. Comme $\varphi$ est convexe alors 

$$
\varphi(x) = \underset{(a,b)\in \mathcal{E}_{\varphi}}{\sup}(a x+b),\text{ pour }x\in \R.
$$
NB:Le sup est pris sur un ensemble de réel $ax+b$ pour $x$ fixé.\\
En effet, $\varphi$ est au dessus de toutes ces tangentes mais égale en un point à une d'entre elle. On en déduit que 
\begin{eqnarray*}
\int\varphi\circ f\text{d}\mu&\geq&\underset{(a,b)\in \mathcal{E}_{\varphi}}{\sup} \int(af+b)\text{d}\mu\\
&=&\underset{(a,b)\in \mathcal{E}_{\varphi}}{\sup} a\int f\text{d}\mu + b\\
&=&\varphi\left( \int f\text{d}\mu\right)
\end{eqnarray*}
\begin{theorem}
Pour $1\leq p\leq \infty$, $\mathcal{L}^p(\Omega, \mathcal{A},\mu)$  est espace de Banach (espace vectoriel normé et complet).
\end{theorem}
\underline{preuve:}\\
Il faut montrer que les suites de Cauchy convergent en norme $||.||_p$. Nous avons besoin de deux lemmes 
\begin{lemma}
Soit $(f_n)_{n\geq1}\in L^p$ telle que 
$$
\sum_{n=1}^\infty ||f_n||_p<\infty,
$$
alors la suite $(\sum_{k=1}^n f_n)$ converge en moyenne d'ordre $p$.
\end{lemma}
\underline{preuve:}\\
Supposons que les $f_n$ sont positives, alors la suite $S_n = \sum_{k=1}^n f_k,\text{ } n\geq1$ converge simplement vers une application mesurable $S$. L'inégalité triangulaire implique que 
$$
\int S_n^p\text{d}\mu\leq \left(\sum_{k=1}^n ||f_k||_p\right)^p
$$
puis en vertu du théorème de convergence monotone, il vient 
$$
\int S^p \text{d}\mu\leq \left(\sum_{k=1}^\infty ||f_k||_p\right)^p<\infty
$$
et $S\in L^p$. Considérons $n,n'\in\N$ tels que $n'>n$, alors 
$$
(S_{n'}-S_n)^p=\left(\sum_{k=n+1}^{n'}f_k\right)^p.
$$
On fait tendre $n'$ vers $\infty$, on obtient (par convergence monotone)
$$
\int(S-S_n)^p\text{d}\mu=\underset{n'\rightarrow\infty}{\lim}||S_{n'}-S_n||_p
$$
L'inégalité triangulaire implique alors
$$
\int(S-S_n)^p\text{d}\mu\leq \sum_{k = n+1}^\infty||f_k||_p\rightarrow 0 \text{ (Reste de série convergente).}
$$
Finalement $||S-S_n||_p\rightarrow 0$. Pour le cas général des $f_n$ mesurable, on éffectue le même travail pour les suites $f_n^+, f_n^-, S_n^+,$ et $S_n^-$ et l'inégalité triangulaire permet de conclure avec 
$$
||S-S_n||_p\leq||S^+-S_n^+||_p +||S^- -S_n^-||_p. 
$$
On a montré qu'une série absolument convergente converge dans $L^p$. \\
$\square$
\begin{lemma}
Un espace vectoriel normé où toute série absolument convergente converge est complet. 
\end{lemma}
\underline{Preuve:}\\
Une suite de Cauchy est convergente si elle admet une sous-suite convergente. En effet soit $(u_n)$ une suite de Cauchy telle que $u_{n_k}$ converge vers une limite $l$. Pour tout $\epsilon >0$, il existe $k_0\in\N$ tel que $||u_{n_k} -l||\leq \epsilon / 2$. De plus, il existe $b_0$ tel que si $\min(k,k')> b_0$ alors $||u_k - u_k'||\leq \epsilon/2$. En choisissant $n\geq\max(n_{k_0}, n_{b_0})$, on a $||u_n - l||\leq \epsilon$.\\

Soit $(u_n)$ une suite de Cauchy. On pose $n_0 = 1$, puis 
$$
n_k = \inf\{n>n_{k-1}\text{ ; }i,i'\geq n\rightarrow ||u_i - u_{i'}||\leq 2^{-k}\}
$$
forme une suite d'indice croissantes. Patr construction, on a 
$$
||u_{n_k} - u_{n_{k+1}}||\leq 2^{-k}
$$
pour $k\geq 1$. La série de terme générale $(u_{n_k} - u_{n_{k+1}})$ est absolument convergente donc convergente par hypothèse. la suite $(u_{n_k})$ converge, puis $(u_n)$ converge en tant que suite de Cauchy dont une sous-suite converge. 
$\square$.
\end{frame}
\begin{frame}[allowframebreaks]
\begin{theorem}
L'ensemble $\mathcal{E}$ des fonctions étagées $\mu$-intégrable est dense dans $L^p$
\end{theorem}
\underline{preuve}:\\
Soit $f\in L^p$, alors il existe une suite de fonctions étagées $(f_n)_{n\geq1}$ telle que  converge vers $f$. Pour tout $n\geq1$, 
$$
g_n(\omega) = \begin{cases}
0&\text{ si }|f_n(\omega)|>2|f(\omega)|\\
f_n(\omega)\text{ sinon }.
\end{cases}
$$
La suite $(g_n)_{n\geq1}$ est une suite de fonctions étagées de $L^p$ qui converge vers $f$ et qui vérifie $|g_n|\leq 2\cdot |f|$.  Par convergence dominée, il vient
$$
||f-g_n||_p\rightarrow 0.
$$
\end{frame}

\begin{frame}[allowframebreaks]{Bibliographie}
Mes notes se basent sur les ouvrages suivants \cite{BaLe07,Ga97,GaHe13,GaKu11,Ca09,le2006integration}
\bibliographystyle{plain}
\bibliography{Integration_notes}
 \end{frame}

\end{document}
