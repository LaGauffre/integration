\documentclass[8pt,notheorems]{beamer}
\usetheme{Copenhagen}
\usepackage[utf8]{inputenc}
\usepackage[T1]{fontenc}
\usepackage{beamerthemesplit}
\usepackage{graphicx}
% \usepackage{tkz-graph}
\usepackage{color}
\usepackage{listings}
\usepackage{relsize}
\usepackage{amsmath,amsfonts,amsthm,t1enc}
\usepackage{fourier}
\usepackage{listings}
\usepackage{amsmath}
\usepackage{tikz}
\usepackage{pgfplots}
\usetikzlibrary{automata,arrows,positioning,calc}
\setbeamertemplate{footline}{\hfill \insertframenumber/\inserttotalframenumber}
% \setbeamertemplate{headline}{}
\def \si {\sigma}
\def \la {\lambda}
\def \al {\alpha}
% \def\e*{\end{eqnarray*}}
\def \di{\displaystyle}

\def \E{\mathbb E}
\def \N{\mathbb N}
\def \R{\mathbb{R}}
\def \A{\mathcal{A}}
\def \Om{\Omega}
\def \om{\omega}
\def \Bor{\mathcal{B}(\mathbb{R})}
\def \limsup{\underset{n\rightarrow+\infty}{\overline{\lim}}}
\def \liminf{\underset{n\rightarrow+\infty}{\underline{\lim}}}

\def \NZ{\mathbb{N}_0}
\def \I{\mathbb I}
\def \w{\widehat}
\def \P {\mathbb P}
\def \V{\mathbb V}



\newcommand{\CL}{\mathbb{C}}
\newcommand{\RL}{\mathbb{R}}
\newcommand{\nat}{{\mathbb N}}
\newcommand{\Laplace}{\mathscr{L}}
\newcommand{\e}{\mathrm{e}}
\newcommand{\ve}{\bm{\mathrm{e}}} % vector e

\renewcommand{\L}{\mathcal{L}} % e.g. L^2 loss.

\newcommand{\ih}{\mathrm{i}}
\newcommand{\oh}{{\mathrm{o}}}
\newcommand{\Oh}{{\mathcal{O}}}
\newcommand{\Exp}{\mathbb{E}}

\newcommand{\Norm}{\mathcal{N}}
\newcommand{\LN}{\mathcal{LN}}
\newcommand{\SLN}{\mathcal{SLN}}

\renewcommand{\Pr}{\mathbb{P}}
\newcommand{\Ind}{\mathbb I}
\newcommand\bfsigma{\bm{\sigma}}
\newcommand\bfSigma{\bm{\Sigma}}
\newcommand\bfLambda{\bm{\Lambda}}
\newcommand{\stimes}{{\times}}

\setbeamertemplate{theorem}[ams style]
\setbeamertemplate{theorems}[numbered]
%\makeatletter
%\def\th@mystyle{%
%    \normalfont % body font
%    \setbeamercolor{block title example}{bg=orange,fg=white}
%    \setbeamercolor{block body example}{bg=blue!20,fg=black}
%    \def\inserttheoremblockenv{block}
%  }
%\makeatother
%\theoremstyle{mystyle}

\makeatletter
    \ifbeamer@countsect
      \newtheorem{theorem}{\translate{Theorem}}[section]
    \else
      \newtheorem{theorem}{\translate{Theoreme}}
    \fi
    \newtheorem{corollary}{\translate{Corollaire}}
    \newtheorem{prop}{\translate{Proposition}}
    \newtheorem{lemma}{\translate{Lemme}}
    \newtheorem{problem}{\translate{Probleme}}
    \newtheorem{solution}{\translate{Solution}}

    \theoremstyle{definition}
    \newtheorem{definition}{\translate{Definition}}
    \newtheorem{definitions}{\translate{Definitions}}

    \theoremstyle{example}
    \newtheorem{example}{\translate{Exemple}}
    \newtheorem{remark}{\translate{Remarque}}
    \newtheorem{examples}{\translate{Examples}}

\makeatletter
\def\th@mystyle{%
    \normalfont % body font
    \setbeamercolor{block title example}{bg=orange,fg=white}
    \setbeamercolor{block body example}{bg=orange!20,fg=black}
    \def\inserttheoremblockenv{exampleblock}
  }
\makeatother
\theoremstyle{mystyle}
\newtheorem{fact}{Fact}





    % Compatibility
    \newtheorem{Beispiel}{Beispiel}
    \newtheorem{Beispiele}{Beispiele}
    \theoremstyle{plain}
    \newtheorem{Loesung}{L\"osung}
    \newtheorem{Satz}{Satz}
    \newtheorem{Folgerung}{Folgerung}
    \newtheorem{Fakt}{Fakt}
    \newenvironment{Beweis}{\begin{proof}[Beweis.]}{\end{proof}}
    \newenvironment{Lemma}{\begin{lemma}}{\end{lemma}}
    \newenvironment{Proof}{\begin{proof}}{\end{proof}}
    \newenvironment{Theorem}{\begin{theorem}}{\end{theorem}}
    \newenvironment{Problem}{\begin{problem}}{\end{problem}}
    \newenvironment{Corollary}{\begin{corollary}}{\end{corollary}}
    \newenvironment{Example}{\begin{example}}{\end{example}}
    \newenvironment{Examples}{\begin{examples}}{\end{examples}}
    \newenvironment{Definition}{\begin{definition}}{\end{definition}}
\makeatother







% ============================================================
% Title
% ============================================================

\title[]{Intégration L3 Actuariat}
\subtitle{Chapitre II: Intégration}
\author{Pierre-Olivier Goffard}
\institute{
	   Université de Lyon 1\\
	ISFA\\
	   \texttt{pierre-olivier.goffard@univ-lyon1.fr}
	  }
\date{
ISFA\\
\today}
\lstset{language=SAS}
\begin{document}

\frame{\titlepage}


% ============================================================
\section{Intégrale par rapport à une mesure}
\subsection{Intégrale des fonctions étagées positives}
\begin{frame}[allowframebreaks]
Soit $(\Om,\mathcal{A}, \mu)$ un espace mesuré. Nous allons définir l'intégrale d'une fonction $f:\Om\mapsto\overline{\mathbb{R}}$ mesurable. On note $\mathcal{M}$ l'ensemble des fonctions mesurables de $(\Om,\mathcal{A})$ dans $(\overline{\mathbb{R}},\mathcal{B}_{\overline{\mathbb{R}}})$.\\
\underline{I. Intégrale par rapport à une mesure}\\
% On note $\mathcal{M}^{+}$ l'ensemble des fonctions mesurables positives. Soit $f\in\mathcal{M}^+$. On définit l'intégrale de $f$, notée $\int f\text{d}\mu$ ou $\int f(\omega)\text{d}\mu$ par
% $$
% \int f\text{d}\mu =\sup\sum_{i\in I}\inf\{f(\omega)\text{ ; }\omega\in\Omega_i\}\mu(\Omega_i),
% $$
% où le supremum est pris sur l'ensemble des partitions $(\Omega_i)_{i\in I}$ de $\Omega$ et $I$ est un ensemble fini d'indice. On note
% $$
% I\left((\Om_i)_{i\in I}, f\right)=\sum_{i\in I}\inf\{f(\omega)\text{ ; }\omega\in\Omega_i\}\mu(\Omega_i)
% $$
% \begin{remark}
% Soit $(\Om_i)_{i\in I}$ et $(\Om_j')_{j\in J}$ deux partitions de $\Om$. $(\Om_i\cap\Om_j')_{(i,j)\in I\times J}$ est encore une partition de $\Om$ telle que
% $I\left((\Om_i\cap\Om_j')_{(i,j)\in I\times J}, f\right)\geq I\left((\Om_i)_{i\in I}, f\right)$
% \end{remark}
% \end{frame}

% \begin{frame}[allowframebreaks]
\underline{1. Intégrale des fonctions étagées positives}\\
Le passage de la mesure d'un ensemble à la mesure d'une fonction (ou intégrale d'une fonction) procède d'une idée simple. Pour $A\subset\Om$, on attribue la mesure $\mu(A)$ à la fonction indicatrice
$$
\mathbb{I}_{A}(\om)=\begin{cases}1,&\text{ si }\om\in A\\
0,&\text{ sinon}.
\end{cases}
$$
\begin{definition}[Intégrale de la fonction indicatrice]
L'intégrale de la fonction $\mathbb{I}_{A}$ par rapport à $\mu$ est définie par
$$
\int_{\Om} \mathbb{I}_A\text{d}\mu=\int_{\Om} \mathbb{I}_A(\om)\text{d}\mu(\om)=\mu(A)
$$
Plus généralement, si $B\in\mathcal{A}$, l'intégrale de $f=\mathbb{I}_A$ sur $B$ par rapport à $\mu$ est définie par
$$
\int_{B}\mathbb{I}_A\text{d}\mu=\int_{\Om}\mathbb{I}_B\mathbb{I}_A\text{d}\mu=\int_{\Om}\mathbb{I}_B(\om)\mathbb{I}_A(\om)\text{d}\mu(\om)=\mu(A\cap B).
$$
\end{definition}
\end{frame}
\begin{frame}[allowframebreaks]
\begin{definition}[Fonction étagées positives]
  On appelle fonction étagée positive une fonction $f:\Om\mapsto \overline{\R}$, définie par
  $$
    f(\om)=\sum_{i=1}^{n}a_i\mathbb{I}_{A_i}(\om),
  $$
  où $A_1,A_2,\ldots,A_n$ est une partition de $\Omega$ de $\mathcal{A}$, et $a_1,\ldots,a_n\geq 0$ des coefficients réels et positifs. On note $\mathcal{E}^+$ l'ensemble des applications étagées positives. 
\end{definition}

\begin{definition}[Intégrale d'une fonction étagée positive]
Soit $f\in\mathcal{E}_+$, l'intégrale de $f$ par rapport à $\mu$ est donnée par
$$
\int f\text{d}\mu=\sum_{i=1}^{n}a_i\int \mathbb{I}_{A_i}\text{d}\mu=\sum_{i=1}^{n}a_i\mu(A_i),
$$
L'intégrale de $f$ sur $B\in\mathcal{A}$ par rapport à $\mu$ est donnée par
$$
\int_{B}f\text{d}\mu=\sum_{i=1}^{n}a_i\int_B\mathbb{I}_{A_i}\text{d}\mu=\sum_{i=1}^{n}a_i\mu(A_i\cap B),
$$
\end{definition}
% Supposons l'existence d'une autre écriture de $f$ sous la forme 
% $$
% \sum_{j= 1}^m b_j\mathbb{I}_{B_j}, 
% $$
% où $B_1,\ldots, B_m$ forme une partition de $\Omega$. Les $b_j$ ne sont pas forcément distincts. En supposant que les $a_i$ sont disjoints alors chaque $A_i$ est la réunion disjointe $\cup_{j\in J}B_j$ telle que $b_j = a_i$ pour $j\in J$. On a alors 
% $$
% \int f\text{d}\mu = \sum_{i = 1}^{n}a_i \mu(A_i) =\sum_{i = 1}^{n}a_i \sum_{\{j: b_j = a_i\}}\mu(B_j) = \sum_{j = 1}^{m}b_i \mu(B_j).  
% $$ 
% La représentation sous la forme d'une fonction étagée positive n'est pas unique mais chaque représentation mène à la même valeur de l'intégrale. 
\begin{example}[Variable aléatoire discrète]
Soit $(\Omega, \mathcal{A}, \mathbb{P})$ un espace probabilisé et $X:(\Omega, \mathcal{A})\mapsto (E, \mathcal{P}(E))$ une variable aléatoire discrète, avec $E = \{x_1,\ldots, x_n\}$. $X$ peut s'écrire 
$$
X = \sum_{i = 1}^n x_i\mathbb{I}_{A_i}, \text{ avec }A_i = \{X = x_i\}.
$$ 
On a 
$$
\int X\text{d}\mathbb{P} = \sum x_i\mathbb{P}(A_i) = \sum x_i\mathbb{P}(X = x_i) = \E(X).
$$
\end{example}
\end{frame}
\begin{frame}[allowframebreaks]
\begin{prop}
Soit $f,g\in \mathcal{E}^+$ et $\alpha>0$. 
\begin{enumerate}
    \item $f+g\in \mathcal{E}^+$ et 
    $$
    \int(f+g)\text{d}\mu = \int f \text{d}\mu + \int g\text{d}\mu. 
    $$
    \item 

    $$
    \int(\alpha f)\text{d}\mu = \alpha \int f\text{d}\mu.
    $$ 
    \item 
    $$
    f\leq g \Rightarrow \int f\text{d}\mu\leq \int g\text{d}\mu.  
    $$
\end{enumerate}
\end{prop} 
\underline{preuve:}\\
\begin{enumerate}
    \item Soient 
    $$
    f = \sum_{i = 1}^{n}a_i\mathbb{I}_{A_i}\text{ et }g = \sum_{j = 1}^{m}b_j\mathbb{I}_{B_i} 
    $$
    L'ensemble $\{A_i\cap B_j\text{ ; }i = 1,\ldots, n, \text{ et }j = 1,\ldots,m \}$ forme une partition. On a donc 
    $$
    f + g = \sum_{i = 1}^{n}\sum_{j = 1}^{m}(a_i+b_j)\mathbb{I}_{A_i\cap B_j}.
    $$
    puis
    \begin{eqnarray*}
    \int f + g \text{d} \mu &=& \sum_{i = 1}^{n}\sum_{j = 1}^{m}(a_i+b_j)\mu(A_i\cap B_j) \\
    &=& \sum_{i = 1}^{n}\sum_{j = 1}^{m}a_i\mu(A_i\cap B_j) + \sum_{i = 1}^{n}\sum_{j = 1}^{m}b_j\mu(A_i\cap B_j)\\
    &=& \sum_{i = 1}^{n}a_i\mu(A_i) + \sum_{j = 1}^{m}b_j\mu( B_j)\\
    &=& \int f\text{d}\mu + \int g\text{d}\mu.
    \end{eqnarray*} 
    \item Immédiat
    \item On note que $g - f \in \mathcal{E}^+$ puis on conclut en intégrant $g = f + (g-f)$.
\end{enumerate}
$\square$
\begin{prop}[Beppo-Lévi $1^{ère}$ partie]\label{prop:BL1}
Soit $(f_n)_{n\in \N^{\ast}}$ une suite croissante de $\mathcal{E}^+$ qui converge simplement vers $f\in \mathcal{E}^{+}$, alors 
$$
\underset{n\rightarrow \infty}{\lim}\int f_n\text{d}\mu =\int f \text{d}\mu.
$$
\end{prop}
\underline{preuve:}\\
On a 
$$
f_n\leq f\Rightarrow \int f_n\text{d}\mu \leq \int f\text{d}\mu,\text{ pour tout }n\in \N^{\ast}. 
$$
donc $\lim \int f_n\text{d}\mu \leq \int f\text{d}\mu$. D'autre part, on pose $A_n = \{f_n \geq c\times f\}$, où $c\in \left]0,1\right[$. Comme la suite des $f_n$ est croissante alors la suite des $A_n$ est croissante pour l'inclusion et de réunion $\Omega$. Comme $f\in \mathcal{E}^+$ alors $f = \sum_{i = 1}^{k}\beta_i\mathbb{I}_{B_i}$ et 
$$
\mathbb{I}_{A_n}.f = \sum_{i = 1}^{k}b_i\mathbb{I}_{B_i\cap A_n}.
$$
On a 
$$
\underset{n\rightarrow \infty}{\lim} \int \mathbb{I}_{A_n}.f = \underset{n\rightarrow \infty}{\lim} \sum_{i = 1}^kb_i\mu(A_n\cap B_i) = \sum_{i = 1}^kb_i\mu( B_i) = \int f\text{d}\mu
$$
Comme $f_n \geq c.f.\mathbb{I}_{A_n}$ pour tout $n\in \N^{\ast}$, on a 
$$
\underset{n\rightarrow \infty}{\lim} \int f_n\text{d}\mu\geq c. \underset{n\rightarrow \infty}{\lim} \int f.\mathbb{I}_{A_n}\text{d}\mu = c. \int f\text{d}\mu.
$$
En choisissant $c$ arbitrairement proche de $1$, il vient 
$$
\underset{n\rightarrow \infty}{\lim} \int f_n\text{d}\mu\geq \int f\text{d}\mu.
$$
$\square$\\
Le lien entre fonctions mesurables positives et étagées se concrétisent avec les résultat suivants. On note $\mathcal{M}_+$ l'ensemble des fonction mesurables de $\Om$ vers $\overline{\R}_{+}$.
\begin{theorem}
Toute fonction $f\in\mathcal{M}_{+}$ est limite simple d'une suite croissante de fonction de $\mathcal{E}_+$.
\end{theorem}
\flushleft
\underline{preuve:}\\
Posons
$$
f_n=n\mathbb{I}_{\{f\geq n\}}+\sum_{k=0}^{n2^{n}-1}\frac{k}{2^{n}}\mathbb{I}_{\left\{\frac{k}{2^{n}}\leq f<\frac{k+1}{2^{n}}\right\}},\text{ }n\geq1.
$$
Par exemple,
$$
f_1=\mathbb{I}_{\{f\geq 1\}}+\frac{1}{2}\mathbb{I}_{\left\{\frac{1}{2}\leq f<1\right\}},\text{ }n\geq1.
$$
et
$$
f_2=2\mathbb{I}_{\{f\geq 2\}}+\frac{1}{4}\mathbb{I}_{\left\{\frac{1}{4}\leq f<\frac{1}{2}\right\}}+\frac{1}{2}\mathbb{I}_{\left\{\frac{1}{2}\leq f<\frac{3}{4}\right\}}+\ldots,\text{ }n\geq1.
$$
Voici une visualisation, pour un $\omega\in \Omega$, on a $f(\omega)\in \R_+$ et
\begin{figure}
\begin{center}
\begin{tikzpicture}[scale = 0.75]
\begin{axis}[
axis x line=bottom,
axis y line = left, xlabel={$f(\omega)$}]
\addplot coordinates {
(0,0) (0.5,0) (0.5,0.5)
(1,0.5) (1,1) (3,1)};
\addplot coordinates {
(0,0) (0.25,0) (0.25,0.25)
(0.5,0.25) (0.5,0.5) (0.75,0.5)
(0.75,0.75) (1,0.75) (1,1)
(1.25,1) (1.25,1.25) (1.5,1.25)
(1.5, 1.5) (1.75, 1.5) (1.75,1.75)
(2,1.75) (2,2) (3,2)
}
;
\end{axis}
\end{tikzpicture}\caption{(bleu) $f_1$; (rouge) $f_2$}
\end{center}
\end{figure}

La suite $(f_n)_{n\geq1}$ est une suite de fonctions étagées positives.
\begin{itemize}
\item qui converge vers $f$. En effet,
\begin{itemize}
\item Si $\om\in\{f=+\infty\}$ alors $f(\om)=+\infty$ et $\underset{n\rightarrow+\infty}{\lim} f_n=\underset{n\rightarrow+\infty}{\lim} n = +\infty$
\item Si $\om\in\{f<+\infty\}$ alors pour $\epsilon>0$, il existe $N$ tel que $\frac{1}{2^{N}}<\epsilon$ et $f(x)<N$ donc pour $n\geq N$ il existe $k\in\{0,1,\ldots, n2^{n}-1\}$ pour lequel $\frac{k}{2^{n}}\leq f<\frac{k+1}{2^{n}}$. Par suite
$$
0\leq f(\om)-f_{n}(\om)=f(\om)-\frac{k}{2^{n}}<\frac{1}{2^{n}}<\frac{1}{2^{N}}<\epsilon.
$$
\end{itemize}
\item qui est croissante, c'est à dire que pour tout $n\in\N$ et tout $\om\in\Om$, $f_{n}(\om)\leq f_{n+1}(\om)$
\begin{itemize}
\item Si $f_{n}(\om)=0$ alors le résultat est trivial,
\item Si $f_{n}(\om)>0$ alors
\begin{itemize}
\item Si $\om\in\{f=\infty\}$ alors $f_n(\om)=n<n+1=f_{n+1}(\om)$
\item Si $\om \in\{\frac{k}{2^{n}}\leq f<\frac{k+1}{2^{n}}\}$ pour un $k\in\{0,1,2,\ldots,n2^{n}-1\}$ alors
$$
f_{n}(\om)=\frac{k}{2^{n}}
\begin{cases}
=f_{n+1}(\om),&\text{ si }\om\in\left\{\frac{k}{2^{n}}\leq f< \frac{2k+1}{2^{n+1}}\right\}\\
 <\frac{2k+1}{2^{k+1}}=f_{n+1}(\om),&\text{ si }\om\in\left\{\frac{2k+1}{2^{n+1}}\leq f< \frac{k+1}{2^{n}}\right\}.
\end{cases}
$$
\end{itemize}
\end{itemize}
\end{itemize}
$\square$
\end{frame}


\subsection{Intégrale des fonctions mesurables positives}
\begin{frame}[allowframebreaks]
\underline{2. Intégrale des fonctions mesurables positives}\\
L'introduction des fonctions étagées permet de définir l'intégrale d'une fonction $f\in\mathcal{M}_{+}$.
\begin{definition}[Par les fonctions étagées positives]
L'intégrale d'une fonction $f\in\mathcal{M}_{+}$ par rapport à $\mu$ sur $B\subset\Omega$ est définie par
$$
\int_B f\text{d}\mu=\sup\left\{\int_B g\text{d}\mu\text{ ; }g\in\mathcal{E}^{+}\text{ , }g\leq f\right\}
$$
\end{definition}
\begin{prop}
Soit $f\in \mathcal{M}^+$ et $(h_n)_{n\in \N^\ast}$ une suite croissante d'éléments de $\mathcal{E}^+$, telle que $\lim h_n = f$. Alors 
$$
\int f\text{d} \mu = \underset{n\rightarrow \infty}{\lim}\int h_n\text{d} \mu.
$$
\end{prop}
\underline{preuve:}\\
Par définition de $\int f\text{d} \mu $, on a 
$$
\int f\text{d} \mu \geq \int h_n\text{d} \mu,\text{ pour tout }n>0,
$$
et en particulier $\int f\text{d} \mu \geq \underset{n\rightarrow\infty}{\lim} \int h_n\text{d} \mu$. Soit $h\in \mathcal{E}^+$ telle que $h\leq f$. On définit 
$$
g_n = min (h_n, h),\text{ }n\in \N^{\ast}.  
$$
On remarque que $(g_n)_{n\geq 1}$ est une suite croissante de fonctions étagées positives qui converge vers $h$, donc d'après Beppo-Lévi
$$
\underset{n\rightarrow \infty}{\lim}\int h_n\text{d}\mu\geq \underset{n\rightarrow \infty}{\lim} \int g_n\text{d}\mu = \int h\text{d}\mu
$$  
et ce pour tout $h\leq f$. On en déduit que 
$$
\underset{n\rightarrow \infty}{\lim}\int h_n\text{d}\mu \geq \sup \left\{\int h\text{d}\mu\text{; }h\in \mathcal{E}^+\text{, }h\leq f\right\} = \int f\text{d}\mu
$$

$\square$\\


\begin{prop}\label{prop:IntegralProperties}
Soit $f,g\in\mathcal{M}_{+}$ et $a, b>0$
\begin{enumerate}
    \item
    $$
    \int (af+ bg)\text{d}\mu = a\int f\text{d}\mu + b\int g\text{d}\mu 
    $$ 
    \item $f\leq g \Rightarrow  \int f \text{d}\mu\leq \int g\text{d}\mu$
% \item[(i)] Si $f\leq g$ alors $0\leq \int_B f\text{d}\mu\leq \int_B g\text{d}\mu$
% \item[(ii)] Si $A\subset B$ et $f\geq0$, alors $\int_A f\text{d}\mu\geq\int_B f\text{d}\mu$
% \item[(iii)] Pour $c\geq0$, on a $\int_B cf\text{d}\mu=c\int_B f\text{d}\mu$
% \item[(iv)] Si $f=0$ alors $\int_B f\text{d}\mu=0$
% \item[(v)] Si $\mu(B)=0$ alors $\int_{B}f\text{d}\mu=0$
% \item[(vi)] Si $f\geq 0$ alors $\int_{B}f\text{d}\mu\geq0$
% \item[(vii)] Si $f\geq 0$ et $\int_B f\text{d}\mu=0$, alors $\mathbb{I}_B f=0$ $\mu$-p.p.
\end{enumerate}
\end{prop}
\underline{Preuve:}\\
\begin{enumerate}
    \item Soient $(f_n)_{n\geq 1}$ et $(g_n)_{n\geq 1}$ deux suites croissantes de $\mathcal{E}^+$ qui convrege respectivement vers $f$ et $g$. On en déduit que la suite définie par $af_n+bg_n$ pour $n\geq1$ converge vers $af+bg$. De plus, on a 
    $$
    \int (af_n+bg_n)\text{d}\mu = a\int f_n\text{d}\mu +b\int g_n\text{d}\mu 
    $$
    ce qui est équivalent à 
    $$
    \int (af+bg)\text{d}\mu = a\int f\text{d}\mu+b\int g\text{d}\mu
    $$
    après passage à la limite. 
    \item Si $f\leq g$ alors $h\leq f\Rightarrow h\leq g$ pour toutes fonction $h\in \mathcal{E}^+$ et 
    $$
    \int f\text{d}\mu=\sup\left\{\int h\text{d}\mu\text{; }h\in \mathcal{E}^+,\, h\leq f\right\}\leq \sup\left\{\int h\text{d}\mu\text{; }h\in \mathcal{E}^+,\, h\leq g\right\} = \int g\text{d}\mu 
    $$
     
\end{enumerate}
$\square$
\end{frame}
\begin{frame}[allowframebreaks]
\begin{theorem}[Beppo-Lévi]
Si $(f_n)_{n\in\N}$ est une suite croissante de fonctions mesurables positives convergeant simplement ($\mu$-p.p.) vers $f$ alors
$$
\underset{n\rightarrow+\infty}{\lim}\int f_n\text{d}\mu=\int f\text{d}\mu.
$$
\end{theorem}
\underline{preuve:}\\

Soit $g\in \mathcal{E}^+$ telle que $f\geq g$, par exemple
$$
g=\sum_{i\in I} \inf\{f(\om)\text{ ; }\om\in A_i\}\mathbb{I}_{A_i},
$$
où $(A_i)_{i\in I}$ forme une partition de $\Om$. Soit
$$
E_n=\{f_n\geq\alpha g\},\text{ }n\in\N,
$$
avec $\alpha\in[0,1]$. $(E_n)_{n\in N}$ est une suite croissante d'éléments de $\mathcal{A}$. Comme $\lim f_n = f\geq \alpha g$ alors on peut trouver $n$ assez grand tel que $\om\in E_n$ et donc $\cup_{n\in N}E_n=\Om$. On en déduit que $(\mathbb{I}_{E_n}\alpha g)_{n\in\N^\ast}$ est une suite croissante de fonction étagées positives telle que $\underset{n\rightarrow \infty}{\lim} \mathbb{I}_{E_n}\alpha g = \alpha g$ et 
$$
\underset{n\rightarrow \infty}{\lim}\int \mathbb{I}_{E_n}\alpha g\text{d}\mu=\int \alpha g\text{d}\mu,
$$
par application de la Proposition \ref{prop:BL1}. On a
$$
f_n\geq f_n\mathbb{I}_{E_n}\geq \mathbb{I}_{E_n} \alpha g
$$
d'où
$$
\underset{n\rightarrow \infty}{\lim}\int f_n\text{d}\mu\geq\alpha \int \mathbb{I}_{E_n} \alpha g\text{d}\mu=\alpha \int  g\text{d}\mu
$$
On obtient
$$
\lim\int f_n\text{d}\mu\geq \int f\text{d}\mu
$$
en faisant tendre $\alpha$ vers $1$ et en prenant le sup sur les fonction étagées positives.
\end{frame}
\begin{frame}[allowframebreaks]

\begin{definition}
Une propriété $\Pi$ est vraie $\mu$-presque partout si et seulement si l'ensemble $\{\omega\in\Omega\text{ ; }\Pi(\omega)\text{  est fausse}\}$ est $\mu$-négligeable au sens où
$$
\exists B\in \mathcal{A}\text{, }\{\omega\in\Omega\text{ ; }\Pi(\omega)\text{  est fausse}\}\subset B\text{ et }\mu(B)=0.
$$
\end{definition}
\begin{example}
Soit $f$ et $g$ deux applications mesurables. Dire que $f\leq g$ $\mu$-presque partout est équivalent à 
$$
\mu(\{\omega\in \Omega\text{ ; }\{f(\omega)>g(\omega)\}) = \mu(\{f>g\}) = 0
$$
\end{example}
\begin{prop}
Soit $f$ une application mesurable à valeurs dans $\overline{R}^{+}$. Alors 
$$
\int f\text{d}\mu = 0 \Leftrightarrow f=0\text{ }\mu\text{-p.p.}
$$
\end{prop}
\underline{preuve:}\\
$\Leftarrow$ Supposons que $f\in \mathcal{E}^+$ alors si $f = \sum_{i = 1}^k\alpha_i\mathbb{I}_{A_i} =0$ $\mu$-pp, cela signifie que soit $\alpha_i = 0$ ou $\mu(A_i) = 0$ pour $i = 1,\ldots k$ puis 
$$\int f\text{d}\mu = \sum_{i = 1}^{k}\alpha_i\mu(A_i) = 0.$$
Si $f\in \mathcal{M}^{+}$ alors $f$ est limite d'une suite  croissante de fonctions étagées positives $(f_n)_{n\in\N^{\ast}}$, nulles $\mu$-pp. on a 
$$
\int f \text{d}\mu  = \underset{n\rightarrow\infty}{\lim} \int f_n\text{d}\mu = 0
$$
par Beppo-Lévi. \\
$\Rightarrow$ Supposons que $\int f\text{d}\mu$ = 0. Pour tout $n\in\N^{\ast}$, on a 
$$
\mathbb{I}_{\left\{f\geq \frac{1}{n}\right\}}\leq n.f,
$$
et 
$$
0\leq \mu\left(\left\{f\geq \frac{1}{n}\right\}\right) = \int \mathbb{I}_{\{f\geq \frac{1}{n}\}}\text{d}\mu\leq n.\int f\text{d}\mu =0.
$$
En remarquant que 
$$
\{f \neq 0\} = \bigcup_{n\in\N^{\ast}}\left\{f\geq \frac{1}{n}\right\},
$$
puis $\mu\left(\{f \neq 0\}\right) \leq \sum_{n\in\N^\ast}\mu\left(\left\{f\geq \frac{1}{n}\right\}\right) =  0.$\\
$\square$
\end{frame}
\subsection{Intégrale des fonctions mesurables}
\begin{frame}[allowframebreaks]
\underline{3. Intégrale des fonctions mesurables}\\
On note $\mathcal{M}$ l'ensemble des applications mesurables de $(\Omega, \mathcal{A})$ vers $(\overline{\R}, \mathcal{B}_{\overline{\R}})$ et $\mu$ une mesure sur $(\Omega, \mathcal{A})$. Pour tout $f\in \mathcal{M}$, on a 
$$
\{f\geq 0\},\{f< 0\}\in \mathcal{A}
$$ 
et 
$$
f^+ = \mathbb{I}_{\{f\geq 0\}}.f\text{, }f^- = -\mathbb{I}_{\{f< 0\}}.f
$$
sont des applications mesurables positives. Il ne faut pas oublier que 
$$
f = f^+ - f^-\text{ et }|f| = f^+ + f^-.
$$
On peut donc définir 
$$
\int f^+\text{d}\mu\text{ et }\int f^-\text{d}\mu
$$
et par suite introduire le concept de fonction $\mu$-intégrable.  
\begin{definition}[Fonction intégrable]
Une application $f\in \mathcal{M}$ est $\mu$-intégrable si et seulement si 
$$
\int f^{+}\text{d}\mu<\infty\text{ et }\int f^{-}\text{d}\mu<\infty
$$
et on pose
$$
\int f\text{d}\mu=\int f^{+}\text{d}\mu-\int f^{-}\text{d}\mu.
$$
On note $\mathcal{L}^{1}(\Omega, \mathcal{A},\mu) := \mathcal{L}^{1}(\mu) $ l'ensemble des fonctions intégrables.
\end{definition}
\begin{remark}[Critère important]
$f\in \mathcal{M}$ est $\mu$-intégrable si et seulement si |f| est $\mu$-intégrable, avec 
$$
\int |f|\text{d}\mu < \infty.
$$
\end{remark}
\begin{prop}
$\mathcal{L}^{1}(\Omega, \mathcal{A},\mu)$ est un espace vecoriel sur $\R$, et l'application $f\mapsto\int f\text{d}\mu$ est une forme linéaire.
\end{prop}
\begin{prop}
Soient $f,g\in \mathcal{M}$.  
\begin{enumerate}
    \item 
    $$
    \left|\int g\text{d}\mu\right|\leq \int |g|\text{d}\mu 
    $$
    \item Supposons que $g$ soit $\mu-$intégrable. Si $|f|\leq g$ alors $f$ est $\mu$-intégrable.
\end{enumerate}
\end{prop}
\underline{preuve:}\\
\begin{enumerate}
    \item Inégalité triangulaire 
    \item On a $f^+\leq g$ et $f^-< g$, ce qui implique que  
    $$
    \int f^+\text{d}\mu<\infty\text{ et }\int f^-\text{d}\mu<\infty
    $$
    puis $f$ est $\mu$-intégrable 
\end{enumerate}
\end{frame}
\begin{frame}[allowframebreaks]
\begin{example}[Intégration par rapport à la mesure de Dirac]
Soit $(\Omega, \mathcal{A})$ un espace mesurable, la mesure de Dirac en $x\in \Omega$ est définie par 
$$
\delta_x(A) = \mathbb{I}_A(x),\text{ pour tout }A\in \mathcal{A}. 
$$
On veut montrer que toute fonction $f\in \mathcal{M}$ est $\delta_x$-intégrable.\\ 
% \begin{equation}\label{eq:int_dirac}
% \int f\text{d}\delta_x = f(x), \text{ pour tout }f\in\mathcal{M}.
% \end{equation}
Pour une fonction étagée positive $f = \sum_{i = 1}^{n}\alpha_i\mathbb{I}_{A_i}$, où $A_1,\ldots, A_n$ forment une partition de $\Omega$, on a 
$$
\int f\text{d}\delta_x = \sum_{i = 1}^{n}\alpha_i\delta_x(A_i)  =  \sum_{i = 1}^{n}\alpha_i\mathbb{I}_{A_i}(x) = f(x).
$$
Pour une fonction mesurable positive, on peut définir une suite croissante $(f_n)_{n\in\N^\ast}$ de fonction étagées positives qui converge vers $f$ alors 
$$
\int f\text{d}\delta_x = \underset{n\rightarrow +\infty}{\lim}\int f_n\text{d}\delta_x  =  \underset{n\rightarrow +\infty}{\lim}f_n(x) = f(x)
$$
Pour une application mesurable $f\in \mathcal{M}$, on écrit $f = f^+-f^-$ et 
$$
\int f\text{d}\delta_x = \int f^+\text{d}\delta_x -\int f^-\text{d}\delta_x = f^+(x)-f^-(x) = f(x)
$$
cela prouve l'intégrabilité de $f$.
\end{example}
\begin{example}[Intégration par rapport à la mesure de comptage]
Si $\Omega = \{\omega_1,\omega_2,\ldots\}$ est un ensemble dénombrable, la mesure de comptage est donnée par 
$$
\mu(A) = \sum_{i = 1}^\infty p_i\delta_{\omega_i}(A),\text{ }A\in \mathcal{A}.
$$
D'après l'exemple précedente, l'application $f\in \mathcal{M}$ est $\mu$-intégrable si et seulement si 
$$
\sum_{i = 1}^{\infty}p_i f^+(\omega_i)<\infty\text{ }\sum_{i = 1}^{\infty}p_i f^-(\omega_i)<\infty
$$
ce qui est équivalent à 
$$
\sum_{i = 1}^{\infty}p_i|f(\omega_i)|<\infty.
$$
Il faut que la série de terme générale $(p_i f(\omega_i))_{i\geq1}$ soit absolument convergente.
\end{example}
\begin{prop}\label{prop:integrale_ensemble_mesure_nulle}
Soit $f$ une application mesurable, si $\mu(A) = 0$ alors 
$$
\int_A f\text{d}\mu = 0.
$$
\end{prop}
\underline{preuve:}\\
On a $\{f.\mathbb{I}_A\neq 0\}\subset A$ et $\mu(A) = 0$. Ainsi $f.\mathbb{I}_A=0$ $\mu$-p.p. ce qui implique $|f|.\mathbb{I}_A=0$ $\mu$-p.p. puis 
$$
\int |f|.\mathbb{I}_A\text{d}\mu = 0
$$
Ce qui implique que 
$$
\int f^+.\mathbb{I}_A\text{d}\mu = 0\text{ et }\int f^-.\mathbb{I}_A\text{d}\mu = 0
$$
et finalement 
$$
\int_A f\text{d}\mu = \int f.\mathbb{I}_A\text{d}\mu = 0.
$$
$\square$
\begin{prop}
Soient $f$ et $g$ deux applications mesurables à valeurs dans $\overline{\R}$, égales $\mu$- presque partout. Si $f$ est intégrable, alors $g$ aussi et $\int f\text{d}\mu = \int g\text{d}\mu$.
\end{prop}
\underline{preuve:}\\
Soit $A = \{f\neq g\}$ alors 
$$
\int f\text{d}\mu = \int_A f\text{d}\mu + \int_{A^c} f\text{d}\mu = \int_{A^c} g\text{d}\mu  =  \int g\text{d}\mu
$$
\begin{prop}\label{prop:integrale_finie_fonction_finie}
Toute application $\mu$-intégrable à valeurs dans $\overline{\R}$ est finie $\mu$-pp. 
\end{prop}
\underline{preuve:}\\
On peut montrer que $|f|$ est finie, soit $M = \left\{|f| = \infty \right\}$. On note que 
$$
f\geq n.\mathbb{I}_M\text{ pour tout }n\in \N^{\ast},
$$ 
il vient alors 
$$
n\mu(M) \leq \int f\text{d}\mu <\infty,
$$
ce qui implique que $\mu(M) = 0$.
\end{frame}
\subsection{Intégrale de fonctions à valeurs complexes}
\begin{frame}[allowframebreaks]
\underline{4. Intégrale de fonctions à valeurs complexes}\\
Nous avons vu qu'une application $f:(\Omega,\mathcal{A})\mapsto (\mathbb{C},\mathcal{B}_\mathbb{C})$ est mesurable si $\Re(f)$ et $\Im(f)$ sont mesurables. 
\begin{definition}
$f$ est dite intégrable si $\Re(f)$ et $\Im(f)$ sont intégrables et dans ce cas
$$
\int f \text{d}\mu = \int\Re(f)\text{d}\mu+i\int \Im(f)\text{d}\mu.
$$
\end{definition}

On note 
$$
|f| = \sqrt{\Re(f)^2+\Im(f)^2}.
$$

\begin{prop}
$|f|$ est intégrable si et seulement si $\Re(f)$ et $\Im(f)$ sont intégrables. 
\end{prop}
\underline{preuve:}\\
On a $|\Re(f)|\leq |f|$ et $|\Im(f)|\leq |f|$ mais aussi $|f|\leq |\Re(f)|+|\Im(f)|$. 

\begin{prop}
L'ensemble $\mathcal{L}_{\mathbb{C}}^{1}(\Omega, \mathcal{A},\mu)$ des fonctions à valeurs complexes intégrables est un espace vecoriel sur $\mathbb{C}$, et l'application $f\mapsto\int f\text{d}\mu$ est une forme linéaire.
\end{prop}
\begin{prop}
Pour $f\in \mathcal{L}_{\mathbb{C}}^{1}(\Omega, \mathcal{A},\mu)$, on a 
$$
\left|\int f\text{d}\mu\right|\leq \int |f|\text{d}\mu
$$
\end{prop}
\underline{preuve:}\\
Posons $\int f\text{d}\mu = re^{i\theta}$, on note que 
$$
\left|\int f\text{d}\mu\right| = r =e^{-i\theta}\int f\text{d}\mu = \int e^{-i\theta}f\text{d}\mu.
$$
Or 
$$
\int e^{-i\theta}f\text{d}\mu = \int \Re\left(e^{-i\theta}f\right)\text{d}\mu + i \int \Im\left(e^{-i\theta}f\right)\text{d}\mu
$$
puis 
$$
\int \Im\left(e^{-i\theta}f\right)\text{d}\mu = 0
$$
On en déduit que 
$$
\left|\int f\text{d}\mu\right| = \int \Re\left(e^{-i\theta}f\right)\text{d}\mu \leq \int \left|e^{-i\theta}f\right|\text{d}\mu = \int|f|\text{d}\mu.
$$
\end{frame}
\section{Théorème de convergence}
\begin{frame}[allowframebreaks]
\underline{II. Théorèmes de convergence}
\begin{lemma}[Lemme de Fatou]
Soit $(f_n)_{n\in\N}\in\mathcal{M}_{+}$, on a
$$
\int \liminf \, f_{n}\text{d}\mu\leq \liminf\int \, f_{n}\text{d}\mu.
$$
\end{lemma}
\underline{preuve:}\\
On pose $g_n=\underset{k\geq n}{\text{inf}}f_k$, pour $n\in\N$, ce qui définit une suite croissante de fonctions positives dont la limite est $\liminf f_n$, on a
$$
f_n\geq g_n \Rightarrow \int f_n\text{d}\mu\geq \int g_n\text{d}\mu
$$
d'où
$$
\liminf \int f_n\text{d}\mu\geq \liminf \int g_n\text{d}\mu
$$
puis par le théorème de Beppo-Lévi, il vient
$$
\liminf \int g_n\text{d}\mu=\lim\int g_n\text{d}\mu=\int \liminf f_n\text{d}\mu.
$$
\flushright$\square$
\flushleft

\begin{remark}[Moyen Mnémotechnique]
Soit
$$
f_n(x)=\mathbb{I}_{[n,n+1]}, \text{ pour }n\geq0,
$$
alors $\liminf f_n=0$ et $\int f_n(x)\text{d}\lambda(x)=1$ donc
$$
0=\int \liminf f_n\text{d}\lambda < \liminf \int f_n \text{d}\lambda=1.
$$
\end{remark}
\begin{example}
Soit
$$
f_n(x)=n\sin^2\left(\frac{\sqrt{x}}{n^{1/3}}\right),\text{ pour }n\geq0\text{ et }x\in \left]0,1\right[
$$
On a $\liminf f_n=+\infty$ puis
$$
+\infty=\int \liminf f_n\text{d}\lambda<\liminf\int f_n\text{d}\lambda
$$
puis $\lim\int f_n\text{d}\mu=\infty$
\end{example}
\begin{theorem}[Théorème de convergence dominée]
Soit $(f_n)_{n\in\N}\in\mathcal{M}$ convergeant presque partout vers $f$, et telle qu'il existe une fonction $g\in\mathcal{L}^{1}(\mu)$ vérifiant $|f_n|\leq g$, pour tout $n\in\N$ alors
$$
\underset{n\rightarrow+\infty}{\lim}\int f_n\text{d}\mu=\int f\text{d}\mu.
$$
\end{theorem}
\underline{preuve:}\\
Les fonctions $f_n$ sont intégrables car dominées par $g$. Par suite, les fonctions $g+f_n$ et $g-f_n$ sont intégrables et positives: On peut leur appliquer le lemme de Fatou, ce qui donne
$$
\int \liminf (g+f_n)\text{d}\mu\leq \liminf \int (g+f_n)\text{d}\mu\Leftrightarrow\int g\text{d}\mu+\int \liminf f_n\text{d}\mu\leq \int g\text{d}\mu+\liminf \int f_n\text{d}\mu.
$$
d'une part et
$$
\int \liminf (g-f_n)\text{d}\mu\leq \liminf \int (g-f_n)\text{d}\mu\Leftrightarrow\int g\text{d}\mu-\int \limsup f_n\text{d}\mu\leq  \int g\text{d}\mu + \liminf \left(-\int f_n\text{d}\mu\right).
$$
d'autre part. On en déduit que
$$
\int f\text{d}\mu \leq \liminf \int f_n\text{d}\mu\text{  et } \int f\text{d}\mu \geq \limsup \int f_n\text{d}\mu
$$
puis
$$
\int f_n\text{d}\mu\rightarrow \int f\text{d}\mu.
$$
\flushright$\square$
\begin{example}
Soit
$$
f_n(x)=\frac{\sin^n(x)}{x^2},\text{ pour }n\in\N \text{ et }x\in \left]1,+\infty\right[.
$$
On a $|f_n(x)|\leq g(x)=\frac{1}{x^2}$ intégrable. Soit $N=\pi/2 + \pi\mathbb{N}$, pour $x\in \left]1,+\infty\right[/N$ on a $\underset{n\rightarrow+\infty}{\lim} f_n(x)=0$, comme $N$ est de mesure nulle alors on a convergence de $(f_n)$ vers $0$ $\lambda$ presque partout. Puis par application du théorème de convergence dominée, on a
$$
\underset{n\rightarrow+\infty}{\lim}\int_{\left]1,+\infty\right[} f_n\text{d}\lambda=\int_{\left]1,+\infty\right[} \underset{n\rightarrow+\infty}{\lim} f_n\text{d}\lambda=0.
$$
\end{example}
\end{frame}
\section{Intégrale et série d'applications}
\begin{frame}[allowframebreaks]
\underline{III. Intégrale et série d'applications}\\
\begin{prop}[Intégrale d'une série de fonctions mesurables positives]\label{prop:serie_app_mesurable_positive}
Soit $(f_n)_{n\in\N}$ une suite de fonctions de $\mathcal{M}_+$ et soit $f=\sum_{n\in\N}f_n$. Alors
$$
\int f\text{d}\mu=\sum_{n\in\N}\int f_n\text{d}\mu.
$$
\end{prop}
\underline{preuve:} examen.
\begin{prop}
Supposons que pour tout $\omega$, $(f_n(\omega))_{n\N^\ast}$ soit une suite alternée tel que $(|f_n(\omega)|)_{n\in\N^\ast}$ décroisse vers $0$. Si $f_1$ est intégrable alors 
$$
\int\sum_{n\in\N^\ast}f_n\text{d}\mu
=\sum_{n\in\N^\ast}\int f_n\text{d}\mu.
$$
\end{prop}
\underline{preuve:} Par hypothèse, $\sum_{n\in\N^\ast}f_n$ converge (Critère de convergence des série alternée). On a également 
$$
\left|\sum_{k=1}^{n}f_k\right| \leq |f_1|, \text{ }n\geq1. 
$$
On applique ensuite le théorème de convergence dominée sur la suite $\left(\sum_{k=1}^n f_k\right)_{n\in\N^\ast}$.\\
$\square$\\
\begin{prop}
Soit $(f_n)_{n\in \N^\ast}$ une suite d'applications intégrables qui vérifient 
$$
\sum_{n\in \N^{\ast}}\int|f_n|\text{d}\mu<\infty.
$$
Alors la série $\sum_{n\in\N^{\ast}}f_n$ converge $\mu$-pp, sa somme $f$ est intégrable et
$$
\int\sum_{n\in\N^\ast}f_n\text{d}\mu = \sum_{n\in\N^\ast}\int f_n\text{d}\mu.
$$
\end{prop}
\underline{preuve:}\\
La suite $(|f_n|)_{n\in\N^{\ast}}$ est une suite de fonction mesurable positives, par application de la Proposition \ref{prop:serie_app_mesurable_positive} on a 
$$
\int \sum_{n\in\N^\ast}|f_n|\text{d}\mu = \sum_{n\in\N^\ast}\int |f_n|\text{d}\mu.
$$
On en déduit que $\sum_{n\in\N^\ast}|f_n|$ est fini $\mu$-p.p. (Proposition \ref{prop:integrale_finie_fonction_finie}) et que la suite $(f_n)_{n\in\N^\ast}$ est absolument convergente. Sa somme, $f$, vérifie 
$$
|f|\leq \sum_{n\in\N^\ast}|f_n|
$$ 
et est donc intégrable. On vérifie également que 
$$
\left|\sum_{k=1}^n f_k\right|\leq\sum_{n\in\N^\ast} |f_n|
$$
puis il vient 
$$
\int \sum_{n\in\N^\ast} f_n \text{d}\mu=\sum_{n\in\N^\ast}\int f_n\text{d}\mu,
$$
par application du théorème de convergence dominée. 
\begin{prop}
Si $(f_n)_{n\in\N^\ast}$ est une suite d'applications mesurables dont la série converge $\mu$-p.p., on a l'inégalité suivante 
$$
\int\left|\sum_{n = 0}^{+\infty}f_n\right|\text{d}\mu\leq \sum_{n = 0}^{+\infty}\int|f_n|\text{d}\mu
$$
\end{prop}
\underline{preuve:}\\
Soit 
\begin{itemize}
\item $\sum_{n = 0}^{+\infty}\int|f_n|\text{d}\mu=\infty$ et l'inégalité est vérifiée
\item $\sum_{n = 0}^{+\infty}\int|f_n|\text{d}\mu<\infty$ alors comme $\left|\sum_{n=0}^{\infty}f_n\right|\leq\sum_{n=0}^{\infty}|f_n| <\infty$ et 
$$
\int\left|\sum_{n = 0}^{+\infty}f_n\right|\text{d}\mu\leq \int\sum_{n = 0}^{+\infty}|f_n|\text{d}\mu =\sum_{n = 0}^{+\infty}\int|f_n|\text{d}\mu. 
$$
\end{itemize}
\end{frame}
\section{Intégrale de Lebesgue et de Riemann}
\begin{frame}[allowframebreaks]
\underline{IV. Intégrale de Lebesgue et de Riemann}\\
Soit $(\Omega, \mathcal{A},\mu) = (\R, \mathcal{B}_\R,\lambda)$ où $\lambda$ désigne la mesure de Lebesgue. Soit $f:(\R, \mathcal{B}_\R)\mapsto (\overline{\R}, \mathcal{B}_{\overline{\R}})$ mesurable. Nous allons examiner le lien netre intégrale de Riemann et de Lebesgue. 


L'intégrale de Riemann étudie les fonctions continues sur un intervalle compact. Soit $f$ une fonction continue par morceaux sur un intervalle fermé $[a,b]$. Soit $P$ une partition de $[a,b]$:$a = x_0 <x_1<\ldots<x_n = b$. On pose
$$
S_p = \sum_{i = 1}^{n}(x_i-x_{i-1})M_i\text{ et }s_p = \sum_{i = 1}^{n}(x_i-x_{i-1})m_i,
$$
où
$$
M_i= \underset{x\in [x_{i-1}, x_{i}]}{\sup} f(x)\text{ et }m_i= \underset{x\in [x_{i-1}, x_{i}]}{\inf} f(x).
$$
$f$ est intégrable sur $[a,b]$ si pour tout $\epsilon >0$ il existe une partition $P$ telle que 
$$
S_P-s_p<\epsilon.
$$
Si $f$ est Riemann intégrable alors $\underset{P}{\inf} S_P = \underset{P}{\sup} s_P$ .\\

Soit $f$ une application définie sur $\left[a,b\right[$, on dit que $\int_{a}^bf(x)\text{d}x$ est convergente, pour tout $c<b$, l'intégrale $\int_{a}^c f(x)\text{d}x$ est convergente et si $\underset{c\rightarrow b}{\lim}\int f(x)\text{d}x$ existe. On note alors $\int_{a}^bf(x)\text{d}x$ et on l'appelle intégrale impropre de Riemann. On procède de la même manière pour définir l'intégrale impropre sur $\left]a,b\right[$ avec $-\infty\leq a<b\leq +\infty$.
\begin{remark}[Fonction Riemann intégrable mais pas Lebesgue intégrable]
Soit 
$$
f = \mathbb{I}_{\mathbb{Q}_1},
$$
où $\mathbb{Q}_1$ désigne l'ensemble des nombres rationels dans $[0,1]$. $f$ est Lebesgue intégrable, avec 
$$
\int f(x) \text{d}\lambda(x) = \int \mathbb{I}_{\mathbb{Q}_1}(x)\text{d}\lambda(x) = \mu(\mathbb{Q}_1) = 0.
$$
mais pas Riemann intégrable puisque 
$$S_P = 1\text{ et } s_P = 0,$$
pour toute partition $P$ de $[0,1]$
\end{remark}
\begin{theorem}\label{theo:Riemann-Lebesgue1}
Si $f$ est Riemann intégrable sur $[a,b]$, elle est Lebesgue intégrable sur $[a,b]$, et les deux intégrales coincident. 
\end{theorem}
\underline{preuve:}\\
Soit $P$ une partition quelconque de $[a,b]$, on pose 
$$
g = \sum_{i = 1}^{n-1}M_i\mathbb{I}_{\left[x_{i-1}, x_i\right[}+ M_n\mathbb{I}_{\left[x_{n-1}, x_n\right]}\text{ et }h = \sum_{i = 1}^{n-1}m_i\mathbb{I}_{\left[x_{i-1}, x_i\right[}+ m_n\mathbb{I}_{\left[x_{n-1}, x_n\right]}.
$$
On note que $g$ et $h$ sont des fonctions étagées qui vérifient $h\leq f\leq g$ et 
$$
S_P = \int_{[a,b]} g\text{d}\lambda\text{ et }s_P = \int_{[a,b]} h\text{d}\lambda
$$
si $f$ est mesurable, alors $\int_{[a,b]} f\text{d}\lambda$ existe et coincide avec $\int_a^b f\text{d}\lambda = \underset{P}{\sup} s_P$. 
\begin{theorem}
Soit $\left]a,b\right[$ un intervalle ouvert $(-\infty\leq a<b\leq\infty)$. Si $\int_a^bf(x)\text{d}x$ est absolument convergente, alors $f$ est Lebesgue intégrable sur $\left]a,b\right[$ et les deux intégrales coincident. 
\end{theorem}
\underline{preuve:}\\
$\forall a,b,c,d$ tels que $-\infty\leq a <c <d<b\leq +\infty$, on a d'après le Théorème \ref{theo:Riemann-Lebesgue1}, 
$$
\int_{[c,d]} |f|\text{d}\lambda  = \int_c^d|f(x)|\text{d}x <\int_a^b|f(x)|\text{d}x<\infty
$$
D'autre part, en vertu du théorème de Beppo-Lévi
$$
\int_{[a,b]} |f|\text{d}\lambda =\underset{d\rightarrow b}{\underset{c\rightarrow a}{\lim}}\int_{[c,d]} |f|\text{d}\lambda (x)
$$
puis $\int_{[a,b]} |f|\text{d}\lambda=\int_a^b |f(x)|\text{d}x$.

\end{frame}
\begin{frame}{Bibliographie}
Mes notes se basent sur les documents suivants \cite{Ca09,le2006integration,GaKu11}
\bibliographystyle{plain}
\bibliography{Integration_notes}
 \end{frame}

\end{document}
